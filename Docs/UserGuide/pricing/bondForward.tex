\subsubsection{Forward Bond} \label{forwardBond}


A forward bond is a derivative contract entered at time $t$ with the agreement to purchase at time $T_f$ an underlying bond contract whose maturity is $T$. Clearly at inception $t\leq T_f\leq T$. $P(t_1,t_2)$, $t_1\leq t_2$ is a discount factor from $t_2$ to $t_1$. For the purpose of this description notional is set to $1$.

\begin{enumerate}
\item Default-free forward bond in multi-curve framework:

\begin{enumerate}
	\item If $t<T_f$ forward contract is derivative. Hence the OIS curve is used for discounting of the derivative product.
	\item If $t>T_f$ ordinary bond is hold but the buyer of the bond. Hence the BondReferenceYield (BRY) curve is used for discounting. A \emph{LiquiditySpread} / \emph{ValuationSpread} can be applied on top of the BRY curve.
	\item  For compounding a compounding curve (COM) is used and discounting is done via the Bond Reference Yield Curve (BRY)
\end{enumerate}
\textbf{Forward bond valuation:}
\begin{itemize}
	\item Today's price of \emph{restricted} bond, with cashflows only counted for $t_i>T_f$:
	$$V(t)=\sum_{ T_f<t_i<T}c_{t_i}{P_{BRY}}(t,t_i)+P_{BRY}(t,T).$$
	The sum is over cashflows that occur after the date of entry into a forward contract. It reflects the coupons of the restricted bond. The term $P_{BRY}(t,T)$ reflects the redemption payment, $c_{t_i}$ reflects a coupon payment at time $t_i$.
	\item Cash value at maturity of forward (in return to bond):
	$$Cash(T_f)=V(t)*P_{COM}(t,T_f)^{-1}$$
	Notice that this value is compounded  according to a $COM$ curve with respective discount factor $P_{COM}(t,T_f)$.
	\item Let $K$ be the agreed price of the bond (to be paid at $T_f$). This price can be given either as a \emph{dirty} or \emph{clean} price. The difference between dirty and clean price is taken account of by appropriate subtraction of an accrual factor in case of the clean price.\\
	Value of forward contract at $t$:
	$$F(t,T_f)=\left(Cash(T_f)-K\right)*P_{OIS}(t,T)$$
	$P_{OIS}(t,T)$ is used here for discounting as risk of the derivative (forward bond) is assumed to be covered.
	\item In praxis both $P_{COM}$ and $P_{BRY}$ will often be chosen as RePo curve.
\end{itemize}

\item Forward on default-able bond:

The basic pricing is as described under point \emph{1)} but now we account for the fact that at $T_f$ a bond is delivered that \emph{might default}. Default might occur before or after maturity $T_f$ of the forward. In the first case a defaulted bond would be delivered at $T_f$.\\

\textbf{Forward on default-able bond valuation proposal:}
%
%
\begin{enumerate}
	\item Conditioned on the assumption that no default event occurs before forward maturity:
	\begin{itemize}
		\item Today's price of \emph{restricted} default-able bond, with cashflows only counted for $t_i>T_f$:
		$$V(t)=\sum_{ T_f<t_i<T}c_{t_i}{\bar{P}_{BRY}}(t,t_i)+\bar{P}_{BRY}(t,T)+R(T_f,T)$$
		\begin{itemize}
			\item $R$ reflects face-value recovery between $T_f$ and $T$: $$R(T_f,T)=R\int_{T_f}^T\rho(s){P}_{BRY}(T_f,s)*Spread\ d s,$$ where $\rho$ is the time-density of default conditioned on the event that no default occurs before $T_f$.
			\item $\bar{P}_{BRY}=P_{BRY}*Spread*CondSurvivalProb$ accounts for potential default 
			\item Conditional survival probabilities are given by $$CondSurvivalProb(t,t_i)=S(t,t_i)/S(t,T_f)$$
			\item $S(t_1,t_2)$ being the survival probability between $t_1$ and $t_2$.
			\item $BRY$: base reference curve, $COM$: compounding curve
		\end{itemize}
		\item As before we set
		 \begin{align*}
		Cash(T_f):=\frac{V(t)}{{P}_{COM}(t,T_f)}
		\end{align*}
	\end{itemize}
	
	\item Taking account of possible default before $T_f$. If default occurs before $T_f$ we receive a defaulted bond. In other words we only receive recovery value, but we will have to still pay $K$. The NPV of the derivative under face value recovery is therefore
	\begin{align*}
	F(t,T_f)=&\int_{t}^{T_f}(RN-K)*P_{OIS}(t,s)\rho(s)d s\\
	&+ (Cash(T_f)-K)*S(t,T_f)*P_{OIS}(t,T_f)
	\end{align*}
	Notice that now the $OIS$ curve is used for discounting as we value a derivative.
	
\end{enumerate}

\end{enumerate}

The cashflow reports provides the relevant future cashflows of the underlying bond with the corresponding discount factors from the BRY and 
the forward cashflows at expiry date with the discount factor from the OIS curve.
The sum of the present values of the foward cashflows is equal to the Forward NPV.

The additional results provided together with the cashflows and NPV are described in table
\ref{tab:additional_results_bond_forward}.

\begin{table}[H]
\begin{center}
\begin{tabular}{|p{5cm}|p{10cm}|}
  \hline
  Result Label & Description \\
  \hline
  bondCashflow & The amounts of the future cashflows of the underlying bond, not discounted as a vector.\\
  \hline
  bondCashflowPayDates & The corresponding payment dates of the underlying bond cashflows. \\
  \hline
  bondCashflowSurvivalProbabilities & The survival probabilities corresponding to the underlying bond cashflows. \\
  \hline
  bondCashflowDiscountFactors & The discount factors from the BRY corresponding to the underlying bond cashflows. \\
  \hline
  bondRecovery & The expected recovery from the underlying bond in case of a default. \\
  \hline
  forwardBondCashflow & The amounts of the contract Value $F(t,T_f)=\left(Cash(T_f)-K\right)$ and premium payments\\
  \hline
  forwardBondCashflowPayDates & The corresponding payment dates of the forward cashflows. \\
  \hline
  forwardBondCashflowSurvivalProbabilities & The survival probabilities corresponding to the forward cashflows. \\
  \hline
  forwardBondCashflowDiscountFactors & The discount factors from the OIS corresponding to the forward cashflows. \\
  \hline
  forwardBondRecovery & The expected payoff of the forward in case of a default before $T_f$. \\
  \hline
\end{tabular}
\end{center}
\caption{Additional Results Bond Forward}
\label{tab:additional_results_bond_forward}
\end{table}
