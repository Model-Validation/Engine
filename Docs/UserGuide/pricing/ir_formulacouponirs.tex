\subsubsection{Interest Rate Swap with Formula Based Coupon}
\label{pricing:ir_formulacouponirs}

The present value of an Interest rate Swap leg with formula-based
coupons can be written as follows
$$
\NPV_{\form} = \sum_{i=1}^{n} N_i \, c_i^{\form} \,\delta_i\, P(t_i)
$$
where each formula based coupon is evaluated as an expectation in the
T-forward measure associated with the payment time $T=t_i$
$$
c_i^{\form} = \E^{t_i}\left[ f_i(R_1, \dots, R_m)\right]
$$
and $f_i(\cdot)$ is a function of IBOR and/or CMS rates as described in
section \ref{ss:formulalegdata}.
 

\subsubsection*{Pricing Model}

For non-callable structured coupons, we generalise the model in
\cite{Brigo_Mercurio_2006}, 13.16.2. We assume the rates
to evolve with a shifted lognormal dynamics under the $T$-forward
measure in the respective currency of $R_i$ as

\begin{equation}\label{lndyn} dR_i = \mu_i (R_i + d_i) dt + \sigma_i
(R_i + d_i) dZ_i
\end{equation}

with displacements $d_i>0$, drifts $\mu_i$ and volatilities $\sigma_i$
or alternatively with normal dynamics

\begin{equation}\label{ndyn} dR_i = \mu_i dt + \sigma_i dZ_i
\end{equation}

for $i=1,\ldots,m$ where in both cases $Z_i$ are correlated Brownian
motions

\begin{equation*} 
dZ_i dZ_j = \rho_{i,j} dt
\end{equation*}

with a positive semi-definite correlation matrix $( \rho_{i,j})$. The
drifts $\mu_i$ are determined using given convexity adjustments

\begin{equation*} 
c_i = E^T(R_i(t)) - R_i(0)
\end{equation*}

where the expectation is taken in the $T$-forward measure in the
currency of the respective rate $R_i$. Slightly abusing notation $c_i$
can be computed both using a model consistent with \ref{lndyn}
resp. \ref{ndyn} (i.e. a Black76 style model) or also a different
model like e.g. a full smile TSR model to compute CMS
adjustments. While the latter choice formally introduces a model
inconsistency it might still produce more market consistent prices at
the end of the day, since it centers the distributions of the $R_i$
around a mean that better captures the market implied convexity
adjustments of the rates $R_i$ entering the structured coupon payoff.

Under shifted lognormal dynamics the average drift is then given by

\begin{equation*} 
\mu_i = \frac{\log( (R_i(0)+d_i+c_i) / (R_i(0)+d_i))}{t}
\end{equation*}

and under normal dynamics

\begin{equation*} 
\mu_i = c_i / t
\end{equation*}

The NPV $\nu$ of the coupon is now given by

\begin{equation}\label{struccpnnpv} 
\nu = P(0,T) \tau E^T ( f(R_1(t),\ldots, R_n(t)) )
\end{equation}

where $P(0,T)$ is the applicable discount factor for the payment time
in the domestic currency and the expectation is taken in the
$T$-forward measure in the domestic currency. To adjust \ref{lndyn}
resp. \ref{ndyn} for the measure change between the currency of $R_i$
and the domestic currency (if applicable), we apply the usual Black
quanto adjustment

\begin{equation*}
  \mu_i \rightarrow \mu_i + \sigma_i \sigma_{i,X} \rho_{i,X}
\end{equation*}

where $\sigma_{i,X}$ is the volatility of the applicable FX rate and
$\rho_{i,X}$ is the correlation between the Brownian motion driving
the FX process and $Z_i$.

\subsubsection*{Evaluation}

To evaluate the expectation in \ref{struccpnnpv} we use Monte Carlo
simulation to generate samples of the distribution of
$(R_1, \ldots, R_n)$, evaluate $f$ on these samples and average the
results.

Using Monte Carlo simulation for each coupon has an impact on 
pricing performance (speed). The formula based coupon should therefore
only be used in situations where explicit analytical or
semi-analytical formulas are not available.


