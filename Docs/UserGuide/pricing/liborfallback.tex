This section describes the handling of coupons referencing a Libor rate after its cessation date.

\subsection{Base Line}

The publication of fixings for a number of Libor indices will cease at certain dates. In addition we expect that market
quotes that we use to build forward curves for these indices will loose liquidity and eventually no longer provided as
well. The system contains fallbacks addressing both points to enable seamless processing of coupons without the
nececssity to change the trade representation from ``Ibor'' to the correpsonding ``OIS'' legs encoding the fallback
conventions.

References: \cite{isda_entrypoint}, \cite{isda_user_guide}, \cite{isda_product_table}, \cite{bbg_1}, \cite{isda_1},
\cite{fca_1}, \cite{isda_2}, \cite{isda_3}.

\subsection{Standard Libor Coupon Pricing}

As long as the sytem's evaluation date lies strictly before an Libor index cessation date, nothing changes, i.e. we use
the historical fixings from the usual data feeds and forward curve built from the usual market quotes. As soon as the
system's evaluation date lies on or after an Libor index cessation date, the following fallback handling kicks in.

\begin{itemize}
\item If the coupon fixing date lies earlier than the index cessation date, the usual ``old'' published Libor fixing is
  used.
\item If the coupon fixing date lies on or after the index cessation date, a fallback fixing is computed over the
  relevant interest rate period of the original Libor index with a lookback of 2 business days. The fallback fixing is
  derived from daily fixings of the fallback overnight index (e.g. SOFR for USD Libor, SONIA for GBP Libor etc.) and a
  defined fallback spread specific to the index.
\end{itemize}

Notice that due to the nature of the fallback rate calculation the value is only known for sure at the end of the
interest rate period. This is opposed to the old Libor fixing which is known in advance of the interest rate period of
the coupon on the fixing date.

Before the fallback rate is fully determined at the end of the interest rate period it comprises both known, historical
overnight fixings and projected overnight fixings. Since the projections are computed on the associated index forward
curve they will change every day and therefore the estimation of the fallback fixing will change as well from day to
day, even if the original Libor index fixing date lies in the past.

\subsection{Non-Standard Instrument Pricing}

For all non-standard instruments referencing Libor, e.g. swaps referencing Libor in-arrears, caps / floors on Libor,
swaptions on Libor, exotics like RPAs on Libor swaps we recommend to change the trade representation so that it
references the fallback overnight index directly and includes the fallback spread. This ensures that all features of the
trade are accurately taken into account in the pricers.

If the trade representation is not amended, the instrument pricing will still work, but an approximation error will be
introduced. As an example consider a cap / floor:

\begin{itemize}
\item Without amending the trade representation: The cap payoff is known at the Libor fixing date, before that the
  projected Libor rate based on the overnight curve and the spread together with the ``classic'' Black pricer is used to
  price the cap, the variance is measured from eval date to Libor fixing date.
\item With amended trade representation: The cap payoff is known at the end of the Libor interest rate period. The
  pricing is done as it is implemented for OIS coupons with global cap within the specialized pricer for that.
\end{itemize}

\subsection{Sensitivity Analysis}

Trades referencing a Libor rate will show sensitivities either on the Libor tenor or the underlying overnight fallback
index depending on the system configuration. The default configuration is the former, i.e. sensitivities stay on the
original Libor tenor although the rate estimation is done on the fallback overnight index forward curve.

Trades that are amended, i.e. referencing the fallback overnight index directly will show sensitivities on the overnight
index in any case, just as any other ``usual'' trade referencing overnight indices.



