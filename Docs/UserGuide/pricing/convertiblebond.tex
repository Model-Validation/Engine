\subsubsection{Convertible Bond}
\label{pricing::convertible_bond}

A convertible bond is priced in a jump diffusion model, see \cite{Andersen_Buffum_2002}.

\underline{Basics}

The model dynamics for the stock price $S(t)$ is given by

\begin{equation}\label{formula:convertible_bond_sde}
  dS / S(t^-) = (r(t) - q(t) + \eta h(t, S(t^-))) dt + \sigma(t) dW(t) - \eta dN(t)
\end{equation}

with a risk free rate $r(t)$, a continuous dividend yield $q(t)$, a default intensity $h(t,S)$, a volatility
$\sigma(t)$, a default loss fraction for the equity $\eta \in [0,1]$ and a Cox process $N(t)$ with

\begin{equation}
  E_t(dN(t)) = h(t,S(t^-)) dt
\end{equation}

The notation $S(t^-)$ is shorthand for $\lim_{\epsilon\downarrow 0} S(t-\epsilon)$. The first jump of $N(t)$ represents
the default of the equity. See equation (1) in \cite{Andersen_Buffum_2002}. We support a local default intensity of the
form

\begin{equation}
h(t,S(t)) = h_0(t) \left( \frac{S(0)}{S(t)} \right)^p
\end{equation}

with a deterministic function $h_0(t)$ that is independent from $S(t)$ and a parameter $p \geq 0$. The parameters $p$
and $\eta$ can be set in the pricing engine configuration.

Both $h_0(t)$ and $\sigma(t)$ are calibrated to market data, i.e. to a market default curve and a market equity
volatility surface. In general this requires the numerical evolution of the density of $S_t$ via its Fokker-Planck PDE.
The corner case $p=0$ allows for a simplified treatment and faster calibration of the model.

The value of a derivative $V = V(t,S)$ is computed by solving the PDE corresponding to the SDE
\ref{formula:convertible_bond_sde}. We do this in terms of the log equity price, i.e. we introduce a new variable
$z = \ln(S)$, and write $v(t,z) = V(t,e^z)$, see also (6) in \cite{Andersen_Buffum_2002}:

\begin{eqnarray}\label{formula:convertible_bond_pde}
v_t + \left( r(t)-q(t)+\eta h(t,e^z)-\frac{1}{2}\sigma(t)^2 \right) v_z + \frac{1}{2}\sigma(t)^2 v_{zz}  \\
- (r(t)+h(t,e^z)) v + h(t,e^z) R(t,e^z) = 0
\end{eqnarray}

Here, $R(t,e^z)$ is a recovery term paid in case of the equity default, which may depend on time $t$ and the {\em
  pre-default} stock price $e^z$.

\underline{Non-Exchangeables Recovery Term}

For non-exchangeables, the default of the equity is the same as the default of the bond issuer. We assume a
deterministic, constant recovery rate $\rho$ for the bond in terms of the (oustanding) nominal $N_t$ . We assume that in
case of the issuer default the investor has a claim on the higher of $\rho N_t$ and the conversion value on the
recovered equity price $C^R_t S(t) (1-\eta)$, i.e. we have

\begin{equation}
R(t,e^z) = \max \{ \rho N_t, C^R_t e^z (1-\eta) \}
\end{equation}

For $\eta = 1$ this simplifies to $R = R(t) = \rho N_t$, independent of the equity price $S = e^z$. Conversion is always
possible in the case of a default, even if no voluntary conversion right could be exercised e.g. due to a contingent
conversion clause at the time of default.

\underline{Exchangeables Recovery Term}

For exchangeables, we have to consider two sources of default risk, the bond issuer default risk and the equity issuer
default risk. We denote the associated default intensities by $h^B(t,e^z)$ for the bond issuer and $h^S(t,e^z)$ for the
equity issuer. The pricing PDE \ref{formula:convertible_bond_pde} is modified to

\begin{eqnarray}\label{formula:convertible_bond_pde_exchangeables}
  v_t + \left( r(t)-q(t)+\eta h^S(t,e^z)-\frac{1}{2}\sigma(t)^2 \right) v_z + \frac{1}{2}\sigma(t)^2 v_{zz} \\
  - (r(t)+h^S(t,e^z) + h^B(t,e^z)) v \\
+ h^S(t,e^z) R^S(t,e^z) + h^B(t,e^z)R^B(t,e^z) = 0
\end{eqnarray}

For non-secured exchangeables,

\begin{itemize}
\item in case of the bond issuer default, the investor has a claim on the bond recovery value only, i.e.
  $$R^B(t,e^z) = \rho N_t$$
\item in case of the equity issuer default, the convertible bond transaction terminates, but the investor has still a claim on
  the full notional of the bond, i.e.
  $$R^S(t,e^z) = N_t$$
\end{itemize}

For secured exchangeables,

\begin{itemize}
\item in case of the bond issuer default, the investor has a claim on the (non-defaulted) equity conversion value plus
  the recovery value on the bond notional less the conversion value, i.e.
  $$R^B(t,e^z) = C^R_t e^z + \max( \rho(N_t - C^R_t e^z),0) $$
\item in case of the equity issuer default, the convertible bond transaction terminates, but the investor has still a claim on
  the full notional of the bond, i.e.
  $$R^S(t,e^z) = N_t$$
\end{itemize}

\underline{Calibration of $h_0$ and $\sigma$ to market data}

We fix a set of calibration times $t_1, \ldots, t_n$ and assume $h_0$ and $\sigma$ to be piecewise flat w.r.t. this time
grid. Our aim is to bootstrap the functions $h_0$ and $\sigma$ so that market values for defaultable bonds and ATM
equity options are matched by model prices for these instruments, i.e. for the defaultable bond we have the condition

\begin{equation}\label{formula:convertible_bond_defbond_match}
  P^{mkt}_D(0,t_i) = P^{mdl}_D(0,t_i) = E \left( e^{-\int_0^t r(s) ds} 1_{\tau > t_i} \right)
\end{equation}

for $i=1,\ldots,n$, where $P^{mkt}_D(0,t_i)$ is the market value of a defaultable bond paying $1$ at $t_i$ and
$P^{mdl}(0,t_i)$ is the model value of this bond with $\tau$ denoting the model default time (first jump time). The
market value $P^{mkt}_D(0,t_i)$ is computed as the product of the risk free zero bond price and the survival probability
up to time $t_i$. Both the zero bond price and the survival probability are computed on the rate and credit curves
stripped from market instruments in the usual way. The condition for the equity option reads

\begin{equation}\label{formula:convertible_equity_option_match}
C^{mkt}(K,t_i) = C^{mdl}(K,t_i) = E \left( e^{-\int_0^t r(s) ds} 1_{\tau > t_i} (S - K)^+ \right)
\end{equation}

where $C^{mkt}(K,t_i)$ is the price of an equity call option at strike $K$ and with maturity $t_i$, computed on the
market equity forward curve and volatility surface. $C^{mdl}(K,t_i)$ is the corresponding model price.

In general we follow the calibration procedure outlined in \cite{Andersen_Buffum_2002}, section 4.2 with a 2-dimensional
minimization of the sum of squares of relative pricing errors for defaultable zero bonds and ATM equity options at each
calibration time $t_i$. The Fokker-Planck equation for the transition density $p(t,z,s,y)$ for times $s\geq t$ is

\begin{eqnarray}\label{formula:convertible_fokker_planck}
  p_s - \frac{1}{2} p_{yy} \sigma(s,e^y)^2 + \\
        \left( r(s)-q(s)+\eta h(s,e^y)-\frac{1}{2} \sigma(s,e^y)^2 \right) p_y + \\
        \left( r + h(s,e^y) + h_y(s,e^y) \right) p = 0
\end{eqnarray}

also compare {\cite{Andersen_Buffum_2002}, equation 11, where we assume $\sigma$ independent of $y$ in our setup (at
  least for the time being). For our parametrisation

\begin{equation}
  h(s,e^y) = h_0(s) ( S(0) / e^y )^p
\end{equation}

we have

\begin{equation}
  h_y(s,e^y) = -p h(s,e^y)
\end{equation}

For $p=0$ we can use a simplified approach for the bootstrap: In this case, the piecewise hazard rate $h_0$ can be
directly set to

\begin{equation}\label{formula:convertible_bondmatch_p0}
  h_0(t) = \frac{ -\ln (S(t_i) / S(t_{i-1}))}{t_i-t_{i-1}}
\end{equation}

for $t_{i-1} \leq t < t_i$. The model volatility $\sigma$ can be set to

\begin{equation}\label{formula:convertible_equitymatch_p0}
  \sigma(t) = \sqrt{ V^2 t_i - \sum_{j=0}^{i-1} \sigma(t_j)^2 (t_{j+1} - t_j) }
\end{equation}

for $t_{i-1} \leq t < t_i$. Here, $V$ is computed as the implied Black volatility of the market option premium
$C(K,t_i)$ weighted with the market survival probability $S(0,t_i)$ and using an adjusted forward level $F^*$

\begin{equation}
  F^* = \frac{F}{S(0,t_i)^\eta}
\end{equation}

where $F$ is the market equity forward level.

Notice that for \underline{exchangeables}, the equity-linked market default curve is used to calibrate
$h(s,e^y) = h^S(s,e^y)$ in \ref{formula:convertible_fokker_planck}. The bond-linked term $h^B(t,e^z)$ in the pricing PDE
\ref{formula:convertible_bond_pde_exchangeables} on the other hand is directly derived from the bond-linked market
default curve and has no impact on the model calibration.

For \underline{cross-currency} the model calibration is done in the target equity currency, i.e. {\em after} switching
from $S^*$ to $S$, see the next section on this.

\underline{Cross Currency}

If the stock denominates in a different currency ccy1 than the bond ccy2, we proceed as follows. Let $S^*$ be the
original equity price process in ccy1, i.e. (compare \ref{formula:convertible_bond_sde}):

\begin{equation}
  dS^* / S^*(t^-) = (r_{ccy1}(t) - q(t) + \eta h^*(t, S^*(t^-))) dt + \sigma(t) dW(t) - \eta dN(t)
\end{equation}

where we index $r(t)$ by the equity currency for clarity. We assume the FX rate process $X$ converting one unit of ccy1 into
ccy2 to follow a lognormal dynamics

\begin{equation}
  dX / X = (r_{ccy2}(t) - r_{ccy1}(t)) dt + \sigma_X(t) dW_X(t)
\end{equation}

with risk free rates $r_{ccy2}$ and $r_{ccy1}$ in ccy2 (domestic currency) resp. ccy1 (foreign currency). The
instantenous correlation between FX and equity process is assumed to be $\rho$, i.e.

\begin{equation}
  dW dW_X = \rho dt
\end{equation}

 We now set $S:=XS^*$, which is a synthetic equity denominated in ccy2 and work out its dynamics. By Ito,

\begin{equation}
  dS = d(XS^*) = XdS^* + S^*dX + dXdS^*
\end{equation}

so after some simplifications

\begin{equation}
  dS / S = (r_{ccy2} - q + \rho \sigma \sigma_X + \eta h(t, S(t^-))) dt + \sqrt{\sigma^2 + \sigma_X^2 + 2
    \rho\sigma\sigma_X} d\tilde{W} - \eta dN(t)
\end{equation}

Note that we replaced $h^*(t,S^*)$ with a new function $h(t,S)$. We will parametrise this new $h(t,S)$ as
$ h(t,S) = h_0(t) (S_0/S)^p$ to obtain the model for $S$. This dynamics is still in the ccy1 measure. Changing the
measure to ccy2 gives the final dynamics

\begin{equation}\label{formula:convertible_bond_xccy_dynamics}
  dS / S = (r_{ccy2} - q + \eta h(t, S(t^-))) dt + \sqrt{\sigma^2 + \sigma_X^2 + 2\rho\sigma\sigma_X} d\tilde{W} - \eta dN(t)
\end{equation}

i.e. we can proceed an in the single currency case with a modified equity volatility
$\sqrt{\sigma^2 + \sigma_X^2 + 2\rho\sigma\sigma_X}$.

\underline{Perpetuals}

Perpetuals are priced by cutting off the maturity at the evaluation date plus a period specified in the pricing engine
parameters of the Bond trade type under ``OpenEndDateReplacement''. A typical setting for this value is 30Y or
50Y. Notice that the parameter is read from the Bond trade type configuration rather than the ConvertibleBond trade type
configuration, even for convertible bonds.

A perpetual convertible bond does not pay any final redemption at its virtual maturity date.

\underline{Dividend Handling}

Currently, we support a continuous dividend model. The absolute dividend amount $D$ paid between $0<t_1<t_2$ as seen
from $t_2$ is estimated as

\begin{equation}
  D = S_{t_2} \left( e^{\int_{t_1}^{t_2} q(s) ds} - 1 \right)
\end{equation}

Historical dividends with ex-dividend date before the evaluation date are added to this value, as appropriate.

\underline{N-of-M triggers}

N-of-M triggers are priced following the implied barrier aprroach proposed in Jasper Anderluh and Hans van der Weide:
Parisian Options – The Implied Barrier Concept, M. Bubak et al. (Eds.): ICCS 2004, LNCS 3039, pp. 851–858,
2004. Springer-Verlag Berlin Heidelberg 2004.

\underline{Curves used in practice (benchmark curve, equity forecast)}

In practice we use a benchmark curve $b$ to discount the bond value instead of the equity forecast curve $r$. We modify
the pricing PDE \ref{formula:convertible_bond_pde}, \ref{formula:convertible_bond_pde_exchangeables} in a
straightforward way by replacing $r$ with $b$ in the discounting term, i.e. the former PDE becomes

\begin{eqnarray}\label{formula:convertible_bond_pde_with_benchmark_curve}
v_t + \left( r(t)-q(t)+\eta h(t,e^z)-\frac{1}{2}\sigma(t)^2 \right) v_z + \frac{1}{2}\sigma(t)^2 v_{zz}  \\
- (b(t)+h(t,e^z)) v + h(t,e^z) R(t,e^z) = 0
\end{eqnarray}

and likewise for the latter PDE. The benchmark curve also includes a security spread, if defined. Notice that we do not
modify the Fokker-Planck PDE \ref{formula:convertible_fokker_planck} for model calibration, i.e. in the calibration step
we use $r$ as the discounting curve always.

Moreover, for the cross-currency case we have a drift term for the original equity $S^*$

\begin{equation}
  r - q + \eta h
\end{equation}

Here, $r$ (equity forecast curve) is not necessarily identical to $r_{ccy1}$ (xccy-consistent discounting curve for
ccy1). Therefore the drift for the currency-changed $S$ is given by

\begin{equation}
  r + ( r_{ccy2} - r_{ccy1}) - q + \eta h
\end{equation}

If $r=r_{ccy1}$ this expression obviously collapsed to \ref{formula:convertible_bond_xccy_dynamics}.

\underline{Model Flavour Selection}

There are some pricing engine parameters that control the model behaviour on a fundamental level, see table
\ref{tab:convertiblebond_model_flavours}. These flags can be used to mimick the behaviour of other vendor models.

\begin{table}[h]
  \begin{tabular}{p{5cm}|p{11cm}}
    Paramteter & Description \\
    \hline
    AdjustEquityForward & If false, the term $\eta h(t,e^z)$ in the coefficient of
    $v_z$ in \ref{formula:convertible_bond_pde} is set to zero, i.e. the hazard rate $h$ is still used in the
    discounting term, but the equity drift is not corrected upwards accordingly. The default value is true. \\ \hline

    AdjustEquityVolatility & If false, the market equity volatility input is not adjusted, but directly used in the
    pricing model. This setting is only possible if $p=0$. It will then set the weighting with the market survival
    probability $S(0,t_i)$ in the context of formula \ref{formula:convertible_equitymatch_p0} to zero, i.e. $V$ is taken
    as the market implied volatility without adjustment. The default value is true. \\ \hline

    AdjustDiscounting & If false, the adjustment of the discounting rate $r$ to the benchmark curve $b$ is suppressed,
    i.e. the change of the PDE \ref{formula:convertible_bond_pde} to
    \ref{formula:convertible_bond_pde_with_benchmark_curve} is {\em not} made (see section ``Curves used in practice''
    above). The default value is true. \\ \hline

    ZeroRecoveryOverwrite & If true, the recovery rate $\rho$ of the convertible
    bond is overwritten with zero. This option is usually used in conjunction with AdjustCreditSpreadToRR set to true
    (see below). The default value is false. \\ \hline

    AdjustCreditSpreadToRR & If true, the credit curve $h(\cdot)$ is adjusted by a factor $\frac{1-R}{1-\rho}$ where $R$
    is the recovery rate associated to the market default curve and $\rho$ is the recovery rate of the bond. Usually,
    $R=\rho$, i.e. the bond recovery rate is the same as the recovery rate of the associated credit curve. In this case,
    the flag has no effect, since the multiplier is $1$. However, when $\rho$ is overwritten with zero due to the flag
    ZeroRecoveryOverwrite set to true, the flag AdjustCreditSpreadToRR should also be set to true. The default value is
    false. \\ \hline

    TreatSecuritySpreadAsCreditSpread & If true, the security spread is not incoroporated into the benchmark curve $b$
    as described above, but rather added as a spread on top of the credit curve, i.e. it is added to $h(t,e^z)$. For
    exchangeables, the security spread is added to {\em both} $h^B$ and $h^S$, i.e. it simultaneously increases the credit
    pread of both the equity and the bond component. Since the security spread is understood as an effective discounting spread,
    it is scaled by $s \rightarrow s / (1-\rho)$ before it is added to $h$, where $\rho$ is the recovery rate of the bond.
    The default value is false. \\ \hline
  \end{tabular}
  \caption{Convertible bond model flavours}
  \label{tab:convertiblebond_model_flavours}
\end{table}

\underline{Numerical Implementation of the Calibration}

The model is calibrated on a configurable set of times

\begin{equation}\label{formula:convertible_calgrid}
t_1,\ldots,t_m
\end{equation}

specified under \verb+Bootstrap.CalibrationGrid+ in the pricing engine configuration. All tenors from the specified grid
before the maturity date of the convertible bond are kept and the maturity date itself is added to the resulting grid to
avoid calibration for times beyond the bond maturity and at the same time ensuring that we do not need to extrapolate
model functions beyond the last calibration point in the pricing.

The case $p=0$ does not require any particular numerical methods, the piecewise flat model functions $h_0$ and $\sigma$
can be directly computed as described in formulae \ref{formula:convertible_bondmatch_p0} and
\ref{formula:convertible_equitymatch_p0}.

For $p>0$ the solution of the Fokker-Planck PDE is numerically solved using a discretisation of the state variable
$y(t)=\ln(S(t))$

\begin{equation}
  y_1, \ldots, y_n
\end{equation}

The grid geometry is determined by the following parameters (configurable in the pricing engine configuration with
parameter labels given in brackets):

\begin{itemize}
\item the number $n$ of grid points (\verb+Bootstrap.StateGridPoints+)
\item an optional mesher concentration $m_c$ to generate a non-uniform grid \\ (\verb+Bootstrap.MesherConcentration+)
\item a mesher epsilon $m_\epsilon$ (\verb+Bootstrap.MesherEpsilon+)
\item a mesher scaling $m_s$ (\verb+Bootstrap.MesherScaling+)
\end{itemize}

The minimum value $y_1$ and maximum value $y_n$ covered by the grid are determined as follows. First, the time original
calibration time grid \ref{formula:convertible_calgrid} is enriched by additional points to ensure that a certain given
number of time steps per year (specified as \verb+Boostrap.TimeStepsPerYear+) is used for the numerical solution of the
PDE later on, i.e.

\begin{equation}
  t_1, \ldots, t_m \rightarrow t'_1, \ldots, t'_M
\end{equation}

with the refined time grid $t'_i$ and $M\geq m$. On each grid point $t'_i$ we compute the forward equity price $F(t'_i)$
for $i=1,\ldots,M$ and set

\begin{eqnarray}
y_1 = \ln \left( \min_{i=1}^M F(t_i) \right) - m_S \sigma_M(t_M)\sqrt{t_M} \Phi^{-1}(1 - m_\epsilon) \\
y_n = \ln \left( \max_{i=1}^M F(t_i) \right) + m_S \sigma_m(t_M)\sqrt{t_M} \Phi^{-1}(1 - m_\epsilon)
\end{eqnarray}

where $\sigma_M(t_M)$ is the market volatility for an ATMF call option with expiry $t_M$.

If the mesher concentration $m_c$ is not given, a uniform grid ${y_i}_{i=1,\ldots,n}$ will be built. Otherwise a grid
concentrated around and contaning $\ln(S)$ as a grid point will be contructed following the approach described in
\cite{Clark_FXBook}, 7.11.3., i.e. using an $\sinh(\cdot)$-transformation for the uniform grid points.

To each state variable grid point $y_i$ we associate the width

\begin{equation}
\Delta y_i = \frac{y_{i+1} - y_{i-1}}{2}
\end{equation}

where we set $y_0 =y_1 - (y_2 - y_1) / 2$ and $y_{n+1} = y_n + (y_n - y_{n-1}) / 2$. The initial Dirac condition
$\delta(y-y(0))$ is numerically approximated as

\begin{eqnarray}
  p_j = \alpha / \Delta y_j \\
  p_{j+1} = (1-\alpha) / \Delta y_{j+1}
\end{eqnarray}

where $j$ is the unique index for which $y_j \leq y(0) < y_{j+1}$ and $\alpha = (y_{j+1} - y(0)) / (y_{j+1}-y_j)$
(notice that $1 \leq j < n - 1$). This ensures

\begin{equation}
  \sum_{i=1}^n p_i \Delta y_i = 1
\end{equation}

The Fokker-Planck PDE is solved numerically using the Crank-Nicholson scheme. The first $d$ steps are evolved using an
implicit Euler scheme to account for the singular initial condition, with $d=5$. The calibration proceeds as follows
(also cf. section 4.2 of \cite{Andersen_Buffum_2002}):

\begin{enumerate}
\item Set $i=1$.
\item \label{algo:convertible_cal_2} Evolve the PDE for the discretised probability $p$ from $t_{i-1}$ to $t_i$ on the
  refined time grid $t'_j$, i.e. stepping over all $t_{i-1} \leq t'_j \leq t_i$ using preliminary parameters $h_0^*$ and
  $\sigma^*$ for the piecewise model functions $h_0$ and $\sigma$ on the interval $[t_{i-1}, t_i)$ (where we set
  $t_0 :=0$). For $i>1$ start with the parameters found for $i-1$. For $i=1$ start with parameters that are calculated
  for $p=0$ as a start value.
\item Compute the model prices for a defaultable zero bond with maturity $t_i$ as
  $$P^{mdl}_D(0,t_i) = \sum_{i=1}^n p_i \Delta y_i$$
  and for an equity option with expiry $t_i$ and ATMF strike $K$ \footnote{for the smallest $i$ s.t. $y_i>\ln(K)$ we
    replace $\Delta y_i$ with $(y_i - \ln(K)) + (y_{i+1}-y_i) / 2$ to account for the fact that $\ln(K)$ in general does
    not lie on the midpoint between $y_{i-1} \leq \ln(K) < y_i$.}
  $$C^{mdl}(K,t_i) = \sum_{i \text{ s.t. } y_i > \ln(K)} p_i \Delta y_i ( e^{y_i} - K) $$
\item Minimize $T(h_0^*,\sigma^*)$ using a Levenberg-Marquardt optimizer by repeating step \ref{algo:convertible_cal_2}
  until the optimizer has converged to a solution. The minimization is either done (chosen via the pricing configuration
  parameter \verb+Bootstrap.Mode+)
  \begin{itemize}
  \item Simultaneously: The target function is defined as
    $$T(h_0^*,\sigma^*) = \left( \frac{P^{mdl} - P^{mkt}}{P^{mkt}} \right)^2  + \left( \frac{C^{mdl} - C^{mkt}}{C^{mkt}} \right)^2$$
    and minimized in two variables simultaneously.
  \item Alternating: The target function
    $$T(h_0^*) = \left( \frac{P^{mdl} - P^{mkt}}{P^{mkt}} \right)^2$$
  is minimized with fixed $\sigma^*$. The found solution $h^*$ for $h_0$ is kept and the target function
    $$T(\sigma^*) = \left( \frac{C^{mdl} - C^{mkt}}{C^{mkt}} \right)^2$$
    is minimized with this fixed $h^*$. The found solution $\sigma^*$ for $\sigma$ is kept and the two optimizations in
    $h^*$ and $\sigma^*$ and repeated until the change in both solutions is below a threshold.
  \end{itemize}
\item Set the piecewise flat model functions $h_0$ and $\sigma$ on $[t_{i-1},t_i)$ to the solution found in the
  optimisation for step $i$.
\item Increase $i$ by $1$ and go to \ref{algo:convertible_cal_2} if $i<m$, otherwise stop.
\end{enumerate}

After the procedure above has finished we have calibrated model functions $h_0(t)$ and $\sigma(t)$ so that market
defaultable zero bonds and ATMF equity optinos are repriced by the model on the calibration grid $t_1,\ldots,t_m$. For
$t > t_m$ we extrapolate the model functions flat (notice this is never used in the context of convertible bond pricing
due to the construction of the calibration grid, see the start of this section).

\underline{Numerical Implementation of the Pricing}

For the pricing of the convertible bond, we build a set of event times

\begin{equation}
  t_1, \ldots, t_m
\end{equation}

associated to pricing events

\begin{itemize}
\item underlying bond flow payment
\item issuer call date
\item investor put date
\item voluntary conversion date
\item mandatory conversion date
\item contingent conversion trigger observation date
\item conversion ratio reset (due to a conversion reset or dividend protection with conversion ratio adjustment)
\item dividend pass through (due to dividend protection with pass through)
\end{itemize}

The set of event times is enriched to ensure a minimum number of grid points per year (taken from parameter
\verb+Pricing.TimeStepsPerYear+ in the pricing engine configuration) for the numerical PDE solution, similar to what is
done for the calibration step:

\begin{equation}
  t_1, \ldots, t_m \rightarrow t'_1, \ldots, t'_M
\end{equation}

The refined times $t'_i$ are also used to approximate American call / puts and conversion rights, i.e. it is assumed
that on each such grid point the respective right can be exercised as an approximation to daily exercises of these
options.

The state variable $y=\ln(S)$ is discretised on a uniform grid

\begin{equation}
  y_1, \ldots, y_n
\end{equation}

where the number of grid points is taken from the pricing configuration parameter \verb+Pricing.StateGridPoints+. The
boundaries are determined in a similar way to the calibration step, i.e.

\begin{eqnarray}
y_1 = \ln \left( \min_{i=1}^M F(t_i) \right) - m_S \sigma_M(t_M)\sqrt{t_M} \Phi^{-1}(1 - m_\epsilon) \\
y_n = \ln \left( \max_{i=1}^M F(t_i) \right) + m_S \sigma_m(t_M)\sqrt{t_M} \Phi^{-1}(1 - m_\epsilon)
\end{eqnarray}

with $m_\epsilon$ taken from \verb+Pricing.MesherEpsilon+, $m_S$ from \verb+Pricing.MesherScaling+.

The PDE is solved using a Crank-Nicholson scheme. We use the following vectors during the rollback. The vector $v$ is
initialised with $0$ at $t=t_m$. The vector $w$ is left uninitialised for the time being:

\begin{itemize}
\item $(v_i)_{i=1,\ldots,n}$: the value of the convertible bond at $y_i$
\item $(w_i)_{i=1,\ldots,n}$: the value of the convertible bond at $y_i$ assuming no conversion is possible during
  conversion periods with contingent conversion triggers observed at the start of these periods
\end{itemize}

The vector $w$ is used to model contingent conversion rights where the trigger is observed at the start of a conversion
period. In this case, both $v$ and $w$ need to be rolled back and on the observation date it is decided how to update
the value vector $v$, see below for details.

In the presence of either conversion reset events or dividend protection events with conversion ratio adjustments,
instead of $v$ and $w$ we need several parallel versions of $v$ and $w$ for a discretised set of conversion ratios
$cr_1,\ldots,cr_K$. The conversion ratios are discretised by setting

\begin{equation}
  cr_k = CR \cdot m_k
\end{equation}

where the multipliers $m_k$ for $k=1,\ldots,K$ are taken from the pricing engine parameter
\verb+Pricing.ConversionRatioDiscretisationGrid+ and $CR$ is the initial conversion ratio from the term sheet. We then
have to deal with $2K$ vectors instead of only $2$:

\begin{itemize}
\item $(v_i^k)_{i=1,\ldots,n}^{k=1,\ldots,K}$
\item $(w_i^k)_{i=1,\ldots,n}^{k=1,\ldots,K}$
\end{itemize}

As before, the vectors $v^k$ are initialised with $0$ while the $w^k$ are left uninitialised during periods without
contingent conversion rights. To ease notation we will write $v^k_i$ for $v_i$ in the following if no discretization of
the conversion rate is required, i.e. we set $k=K=1$ in this case.

The pricing algorithm proceeds as follows:

\begin{enumerate}
\item Set $i := M$
\item \label{algo:convertible_rollback_1} If on $t_i$ there is a conversion right contingent on the observation of a
  trigger at an earlier $t_{i'}$, $i'<i$ and $w^k$ is not initialised, initialise $w^k$ with $v^k$ (for $k=1,\ldots,K$)
\item Update $v_j^k$ according to voluntary (contingent) conversion rights:
  \begin{enumerate}
  \item compute the exercise value $E_j = cr_k e^{y_j} N_t / N_0$
  \item if there is a contingent conversion trigger check on $t_i$:
    \begin{enumerate}
    \item if the check is on the start of period check for future contingent converion rights: if the conversion right
      is not triggered, i.e.
      $$cr_k e^{y_j} \leq B$$
      set $v_j^k$ to $w_j^k$ (future conversion is not possible), otherwise do not update $v_j^k$ (future conversion is
      possible). Uninitialise all $w^k$'s.
    \item if the check is a spot check and the conversion right is triggered, i.e.
      $$cr_k e^{y_j} > B$$
      set $v_j^k$ to $\max(v_j^k, E_j)$.
    \end{enumerate}
  \item if there is no contingent conversion trigger check on $t_i$, update $v_j^k$ to $\max(v_j^k, E_j)$
  \end{enumerate}
\item If on $t_i$ there is a conversion ratio reset (due to a conversion reset feature or a dividend protection or a
  reset of the conversion ratio to a fixed value)
  \begin{enumerate}
  \item  for all $k=1,\ldots,K$, $j=1,\ldots,n$:
  \item interpolate $v_j^k$ linearly from the $v_j^\kappa$ on the adjusted conversion ratio $cr'=cr'(cr_k, y_j)$ which
    is always computable as a function of the assumed conversion ratio for $v^k$ and the log equity price $y_j$ for the
    involved reset and dividend protection features.
  \item interpolate $w_j^k$ from $w_j^\kappa$ analoguously, if $w$ is initialised
  \end{enumerate}
\item If no conversion resets occur on $t_i$ or prior to $t_i$, collapse the $v^k$ to one $v$ by interpolating the $v$
  from the $v^k$ on the initial conversion ratio, and do the same with $w^k$ (if initialised)
\item If there is a mandatory conversion right on $t_i$ set $v^k_j$ (and $w^k_j$ if initialised) to the mandatory
  conversion payoff, which is a function of $y_j$.
\item If there is an issuer call right on $t_i$ and no conversion (voluntary or mandatory) has been exercised in the
  above steps on the same $t_i$, update
  $$v_j^k \rightarrow \min(\max(c, f), v_j^k)$$
  where $f$ is the forced conversion value
  $$f = cr_k e^{y_j} N_t/N_0$$
  and $c$ is the amount to be paid by the issuer in case of a call. If the call is soft, only update
  $v_j^k$ if the soft call is triggered, i.e.
  $$e^{y_j} > T N_0 / cr_k$$
  If $w$ is initialised, update $w$ analogously. Notice that a forced coversion can be exercised regardless of a
  contingent conversion clause active.
\item If there is an investor put on $t_i$, update
  $$v_j^k \rightarrow \max( v_j^k, p)$$ where $p$ is the put price to be paid by the issuer in case of a put exercise by
  the investor. Notice that if a call was exercised before, the put overrides the call. Also, a put overrides a
  conversion (voluntary or mandatory) on the same date, because the investor will exercise the put or conversion
  whatever is more valuable, and in case of a mandatory conversion and a put on the same date (unlikely in a real term
  sheet) we favor the put over the mandatory conversion.
\item If there is a dividend protection pass through on $t_i$ add the relevant amount to $v_j^k$ (and $w_j^k$ if
  initialised).
\item If a bond cashflow is paid on $t_i$, add the amount to $v_j^k$ (and $w_j^k$ if initialised)
\item Roll back $v_j^k$ (and $w_j^k$ if initialised) from $t_i$ to $t_{i-1}$. If $i>1$ go to step
  \ref{algo:convertible_rollback_1}.
\end{enumerate}

After the algorithm has finished, the NPV of the convertible is interpolated from $v_j$.

\underline{Additional pricing engine results}

The following additional results are provided by the pricer:

Results concerning the processing of events on the FD grid: 

\begin{itemize}
\item time: the time associated to the FD grid point
\item date: an event date associated to the time
\item notional: the current notional of the underlying bond
\item accrual: the current accrual of the underlying bond
\item flow: bond flow payments
\item call: issuer call events:
  \begin{itemize}
  \item \verb+@1.0+ hard call with price $1.0$
  \item \verb+@1.0 s@1.3+ soft call with price $1.0$ and trigger $1.3$
  \end{itemize}
\item put: investor put events:
  \begin{itemize}
  \item \verb+@1.0+ put with price $1.0$
  \end{itemize}
\item conversion: conversion event:
  \begin{itemize}
  \item \verb+@0.03+ voluntary conversion with conversion ratio $0.03$
  \item \verb+@0.03 c@1.3+ contingent conversion with conversion ratio $0.03$ and coco barrier $1.3$
  \item \verb+@0.03 c@1.3b+ as before, but check is on the next date in the past without being marked with \verb+b+
  \item \verb+peps(0.03,0.04)+: mandatory conversion with PEPS barriers $0.03$ and $0.04$ 
  \end{itemize}
\item CR\_reset: conversion ratio reset event:
  \begin{itemize}
  \item \verb+0.8@0.9/CP0+ conversion ratio reset with gearing $0.8$ and threshold $0.9$, reference price is the initial
    conversion price
  \item \verb+0.8@0.9/CPT+ as before, but reference price is the current conversion price
  \item \verb+DP(i,0.23)@0.17+ dividend protection conversion ratio reset, relevant period starts at time index $i+1$,
    dividend yield over relevant period (without historical dividends) is $0.23$, threshold is $0.17$
  \end{itemize}
\item div\_passth: dividend passthroughs:
  \begin{itemize}
    \item \verb+@0.17+: dividend pass-through event with threshhold $0.17$
  \end{itemize}
\item curr\_cr: the current conversion ratio associated to the time index, if there are conversion ratio reset events or
  dividend protection events with conversion ratio reset active, the number is displayed with the suffix \verb+s+ for
  stochastic (in which case several PDE planes for different discretized conversion ratios are rolled back)
\item fxConv: the fx conversion factor from Equity to Bond currency for the time index.
\item eq\_fwd: the equity forward price at the time index (value is output just for info purposes)
\item div\_amt: the expected dividend amount relevant for dividend pass through or conversion rate adjustments based on
  the dividend protection feature (calculated using the equity forward price, value is output just for info purposes)
\item conv\_val: the expected conversion value at the time index based on the equity forward and current conversion
  ratio (excluding stochastic adjustments, value is output just for info purposes)
\item conv\_prc: the expected conversion price at the time index based on the equity forward and current conversion
  ratio (excluding stochastic adjustments, value is output just for info purposes)
\end{itemize}

Results concerning the pricing:

\begin{itemize}
\item BondFloor: the bond floor value
\item trade.tMax: the maximum time relevant for the trade
\item market.discountRate(tMax): the benchmark curve discount rate at tMax
\item market.creditSpread(tMax): the hazard rate at tMax, for exchangeables this is the hazard rate associated to the
  equity credit risk
\item market.exchangeableBondSpread(tMax): the hazard rate at tMax associated to the bond credit risk (only filled for
  exchangeables)
\item market.discountinngSpread: the additional (security) spread applied to discounting
\item market.recoveryRate: the recovery rate applied in the model (in case of a bond default)
\item market.equitySpot: the equity spot
\item market.equityForward(tMax): the equity forward at tMax
\item market.equityVolatility(tMax): the equity market volatility at tMax
\item model.fdGridSize: the number of grid points for the FD solve (in time direction)
\item model.eta: the model parameter eta (equity default loss ratio)
\item model.p: the model parameter p (credit-equity linkage)
\item model.calibrationTimes: the times on which the model is calibrated
\item model.h0: the calibrated model parameter $h_0$ on the calibration time grid
\item model.sigma: the calibrated model parameter $\sigma$ on the calibration time grid
\item conversionIndicator: the pv of a hypothetical instrument paying $1$ when the conversion is exercised
\end{itemize}

Results concerning the processing of events before or on the evaluation date:

\begin{itemize}
\item historicEvents.initialConversionRatio: The initial conversion ratio of the bond.
\item historicEvents.crReset: Data concerning historic conversion ratio resets:
  \begin{itemize}
  \item crReset\_DATE\_S: the share price on the reset date
  \item crReset\_DATE\_threshold: the threshold
  \item crReset\_DATE\_referenceCP: the reference conversion price
  \item crReset\_DATE\_gearing: the gearing
  \item crReset\_DATE\_floor: the floor
  \item crReset\_DATE\_globalFloor: the global floor
  \item crReset\_DATE\_currentCr: the conversion ratio before the reset event
  \item crReset\_DATE\_adjustedCr: the conversion ratio after the reset event
  \end{itemize}
\item historicEvents.crReset\_DP: Data concerning historic dividend protection conversion ratio adjustments:
  \begin{itemize}
  \item crReset\_DP\_DATE\_div\_DATE1\_DATE2: the dividends for the adjustment period date1/2
  \item crReset\_DP\_DATE\_S: the share price on the reset ddate
  \item crReset\_DP\_DATE\_threshold: the threshold
  \item crReset\_DP\_DATE\_currentCr: the conversion ratio before the reset event
  \item crReset\_DP\_DATE\_adjustedCr: the conversion ratio after the reset event
  \end{itemize}
\item historicEvents.coco: Data concerning contingent conversion with start of period observation:
  \begin{itemize}
  \item coco\_DATE\_S: the share price on the observation date
  \item coco\_DATE\_cocoBarrier: the barrier
  \item coco\_DATE\_currentCr: the conversion ratio on the observation date
  \item coco\_DATE\_triggered: whether the barrier is triggered (i.e. conversion is allowed)
  \end{itemize}
\item historicEvents.accruedDividends\_DATE1\_DATE2: dividends between the last dividend protection date before the
  evaluation date and the evaluation date, which enter the next (future) dividend reset event or pass through
\end{itemize}



