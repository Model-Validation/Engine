\subsubsection{Bond Option}
\label{pricing:ir_bondoption}

Bond Options are European options and usually valued using the Black's model, which assumes that the value of the
underlying bond at exercise time $T$ in the future has a lognormal distribution. The pricing formula for a Bond Option
that {\em knocks out} if the bond defaults before the option expiry is

\begin{eqnarray*}
NPV^{call}_{\text{knock-out}}&=& (1-p) P(0,T)\left[F_BN(d_1)-KN(d_2)\right] \\
NPV^{put}_{\text{knock-out}}&=& (1-p) P(0,T)\left[KN(-d_2)-F_BN(-d_1)\right] \\
d_1&=&\frac{\ln(F_B/K)+\sigma_B^2T/2}{\sigma_B\sqrt{T}}\\ 
d_2&=&d_1-\sigma_B\sqrt{T} 
\end{eqnarray*}

where $F_B$ is the forward bond price, $\sigma_B$ is the price volatility, $K$ is the strike price of the bond option,
$T$ is its time to option maturity, $P(0,T)$ is the (risk-free) discount factor for maturity $T$, and $p$ is the default
probability of the bond until the option expiry. If a yield volatility $\sigma_y$ is given, the price volatility is
derived as

\begin{equation}
\sigma_B = \sigma_Y  D y
\end{equation}

if $\sigma_y$ is a lognormal volatility and

\begin{equation}
\sigma_B = \sigma_Y  D  (y + s)
\end{equation}

if $\sigma_y$ is a shifted lognormal volatility with shift $s$ and

\begin{equation}
\sigma_B = \sigma_Y  D
\end{equation}

if $\sigma_Y$ is a normal volatility. Here $D$ denotes the forward modified duration and $y$ the forward yield of the
bond w.r.t. $F_B$.

The forward bond price $F_B$ is computed as for a plain bond but a) only taking cashflows into account that are paid
after the option expiry and b) using forward discount factors on the underlying's reference curve (including a security
spread, if given) and forward survival probabilities computed from the underlying's credit curve as of the option expiry
date, i.e. the forward price is computed {\em conditional on survival} of the underlying until the option expiry.

If the option {\em does not knock out} if the bonds defaults before the option expiry, the forward bond price used to
calculate the option price in the Black formula above is replaced by the weighted sum of the forward bond price
conditional on survival until the option expiry weighted with the survival probability of the bond until the option
eypiry and the recovery value of the bond weighted with the default probability of the bond until option expiry, i.e. in
this case we have

\begin{eqnarray*}
F_{B, \text{no knock-out}} &=& (1-p) F_B + p R \\
NPV^{call}_{\text{no knock-out}}&=& P(0,T)\left[F_{B, \text{no knock-out}}N(d_1)-KN(d_2)\right] \\
NPV^{put}_{\text{no knock-out}}&=& P(0,T)\left[KN(-d_2)-F_{B, \text{no knock-out}}N(-d_1)\right] \\
d_1&=&\frac{\ln(F_{B, \text{no knock-out}}/K)+\sigma_B^2T/2}{\sigma_B\sqrt{T}}\\ 
d_2&=&d_1-\sigma_B\sqrt{T} 
\end{eqnarray*}

where $R$ is the recovery value of the bond in case of a default before option expiry. The forward modified duration $D$
and forward yield $y$ is computed w.r.t. $F_{B, \text{no knock-out}}$ in this case.

The cashflow reports provide the relevant future cashflows of the underlying bond and the non-discounted option npv at expiry date. 
The sum of the column present value of the underlying bond cashflows is equal to the price of the forward Bond $F_B$.

The additional results provided together with the cashflows and NPV are described in table
\ref{tab:additional_results_bond_option}.

\begin{table}[H]
\begin{center}
\begin{tabular}{|p{5cm}|p{10cm}|}
  \hline
  Result Label & Description \\
  \hline
  knockOutProbability & The probability $p$, that the bond defaults before option expiry . \\
  \hline
  CashStrike & The strike in terms of cash. \\
  \hline
  FwdCashPrice & The forward cash price of the bond, $(1-p) F_B + p R$. \\
  \hline
  PriceVol & The volatility used. \\
  \hline
  ExpectedBondRecovery & The expected recovery from the underlying bond in case of defaulting before expiry ($p R$). \\
  \hline
\end{tabular}
\end{center}
\caption{Additional Results Bond Option}
\label{tab:additional_results_bond_option}
\end{table}