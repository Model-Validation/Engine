\subsubsection{Cap/Floor}
\label{pricing:ir_capfloor}

Interest rate caps, floors, and collars are supported for underlyings of type

\begin{itemize}
\item Ibor (e.g. USD-LIBOR-3M)
\item USD-SIFMA
\item forward looking RFR term rate, e.g. USD-SOFR-3M
\item backward looking RFR rates, e.g. USD-SOFR compounded over 3M periods
\end{itemize}

We use the Black model for log-normal and shifted log-normal volatilities, and the Bachelier model for normal
volatilities. To price a cap which is a series of caplets we price the individual caplets and add up the NPVs of the
caplets\footnote{Here and in what follows we often refer to ``cap'' and ``caplet'' but mean both caps / floors resp. caplets /
floorlets}.

\subsubsection{Pricing of an Ibor or forward looking RFR term rate caplet}

A caplet on an Ibor or forward looking RFR term rate $F(t, t_s, t_e)$ as seen from $t$ with fixing time $t_f > t \geq
0$, index rate projection period $[t_s, t_e]$ and pay time $t_p$ has the price $\nu_F$

\begin{equation}
\label{pricing:ir_capfloor_ibor_caplet}
\nu_F = N \cdot P(0,t_p) \cdot \text{Black/Bachelier}(K, F(0, t_s, t_e), \sigma(t_f) \sqrt{t_f}, \omega)
\end{equation}

where ``Black/Bachelier'' refers to the well-known pricing formulas \ref{models:black} resp. \ref{models:bachelier} and

\begin{itemize}
\item $N$ is the notional of the caplet
\item $P(0,t_p)$ is the applicable discount factor, usually the discount factor on an overnight curve corresponding to
  the collateralization of the cap
\item $K$ is the strike of the caplet
\item $F0,t_s,t_e)$ is the projected forward rate as seen from the valuation time $t=0$
\item $\sigma = \sigma(t_f)$ is the volatility of the caplet. The volatility is read from the interpolated caplet
  volatility surface for the underlying Ibor / term rate index at time $t_f$ and strike $K$. The caplet volatility
  surface is either
   \begin{itemize}
   \item bootstrapped from quoted market volatilities as described in \ref{pricing:ir_capfloor_bootstrap} or
   \item proxied by a another caplet volatility surface as described in \ref{pricing:ir_capfloor_proxying}. This is
     useful if no direct market quotes are available for the underlying index. For example, we might use liquid market
     quotes to boostrap a volatility surface for backward looking USD-SOFR caplets and use the resuling caplet surface
     to proxy caplet volatilities for forward looking USD-SOFR-3M term rates.
   \end{itemize}
\item $\sigma^2 t_f$ is the accrued variance from $t=0$ to the fixing time $t_f$ of the caplet. The square root of the
  accrued variance is the input to the Black / Bachelier model (\ref{models:black}, \ref{models:bachelier}) and called
  ``terminal volatility'' in the context there
\item $\omega \in \{-1,+1\}$ is $1$ for a caplet and $-1$ for a floorlet
\end{itemize}

We mention that if the caplet uses a non-standard timing, i.e. a fixing time $t_f$ such that $t_e \neq t_p$, the forward
rate $F(0,t_s,t_e)$ entering the Black / Bachelier formula \ref{pricing:ir_capfloor_ibor_caplet} is replaced by a
timing-adjusted forward rate. We do not go into further details on this point here.

\subsubsection{Pricing of an backward looking RFR caplet}

The general framework to price backward looking RFR caplet is outlined in \cite{Lyashenko_Mercurio_2019}. In the Forward
Market Model (FMM) we consider the term rate $R$ spanning the accrual period $[S,T]$ with length $\tau$

\begin{equation}
R(S,T) := \frac{1}{\tau} \left( e^{\int_S^T r(u) du} - 1 \right)
\end{equation}

$R(S,T)$ is the continuous version of the daily compounded rate. Denote $R(T_{j-1}, T_j)$ as seen from $t$ as $R_j(t)$
on a forward rate grid $T_0, \ldots, T_n$, then $R_j(t)$ is a martingale in the $T$-forward measure and gives rise to
the (normal) FMM dynamics

\begin{equation}\label{pricing:ir_capfloor_FMM}
dR_j(t) = \sigma_j(t) g_j(t) dW_j(t)
\end{equation}

where

\begin{itemize}
\item $\sigma_j(\cdot)$ is an adapted process
\item $g$ is a scaling function that accounts for the decreasing volatility of $R_j$ in $[T_{j-1},T_{j}]$:
  \begin{itemize}
  \item $g$ is piecewise differentiable (technical requirement)
  \item $g(t) = 1$ for $t < T_{j-1}$, i.e. no scaling before the start of the accrual period of $R(t,T_j)$
  \item $g(t)$ is monotonically decreasing in $[T_{j-1}, T_j]$
  \item $g(t) = 0$ for $t \geq T_j$
  \item In the following we use a linear decay: $g_j(t) = \min \left( \frac{(T_j-t)^+}{T_j-T_{j-1}}, 1 \right)$
  \end{itemize}
\end{itemize}

We note that It is straightforward to modify \ref{pricing:ir_capfloor_FMM} to use lognormal or shifted lognormal
dynamics. We do not go into further details here.

In the FMM framework, a (backward looking) cap with strike $K$ has payoff

\begin{equation}
\max( \omega (R_j(T_j) - K) , 0)
\end{equation}

Assuming $\sigma_j(\cdot)$ to be deterministic we get a Bachelier formula for the caplet price $\nu_B$

\begin{equation}
\nu_B = N \cdot P(0,T_j) \cdot \text{Black/Bachelier}( K, R_j(0), v, \omega )
\end{equation}

where the variance $v^2$ is given by

\begin{equation}
v^2 = \int_0^{T_j} \sigma_j(s)^2 g(s)^2 ds
\end{equation}

We use the explicit formula for $v^2$ from \cite{Lyashenko_Mercurio_2019} in section 6.3

\begin{equation}
v^2 = \sigma_j^2 \left[ T_{j-1}^+ + \frac{1}{3} \frac{(T_j - T_{j-1}^+)^3}{(T_j - T_{j-1})^2} \right]
\end{equation}

Summarizing, this leads to the following price $\nu_B$ for a backward looking RFR caplet:

\begin{equation}
\label{pricing:ir_capfloor_rfr_caplet}
\nu_B = N \cdot P(0,t_p) \cdot \text{Black/Bachelier}(K, F(0, t_s, t_e), v, \omega)
\end{equation}

where

\begin{itemize}
\item $N$ is the notional of the caplet
\item $P(0,t_p)$ is the applicable discount factor for the payment time $t_p$
\item $K$ is the strike of the caplet
\item $F(0,t_s,t_e)$ is the projected forward rate as seen from $t=0$ for the compounding period $[t_s, t_e]$.
\item $v^2$ is the accrued variance from $t=0$. As above, the square root of the accrued variance, i.e. $v$ is the input
  to the Black / Bachelier model (\ref{models:black}, \ref{models:bachelier}) and called ``terminal volatility''
  there. Following the derivation above, we have
\begin{equation}
v^2 = \sigma(t_0)^2 \left[ t_0^+ + \frac{1}{3} \frac{(t_1 - t_0^+)^3}{(t_1 - t_0)^2} \right]
\end{equation}
with
  \begin{itemize}
  \item $t_0$, $t_1$ denoting the first and last relevant fixing times of the overnight fixings underlying the compounded
    RFR rate. Notice that $t_0$ might be negative, i.e. part of the underlying overnight fixings are already known.
  \item $\sigma(t_0)$ is the volatility of the RFR caplet. This is read from the interpolated caplet volatility surface
    for the underlying Ibor / term rate index at time $t_0$ and strike $K$ (we set $\sigma(t) = \sigma(0)$ for $t<0$).
  \end{itemize}
\item $\omega \in \{-1,+1\}$ is $1$ for a caplet and $-1$ for a floorlet
\end{itemize}

Notice the important difference between $\sigma(t_0)$ and the effective Black / Bachelier caplet volatility
$\sigma_\text{eff}(t_1)$ which can be defined by

\begin{equation}\label{pricing:ir_capfloor_effvol}
\sigma_\text{eff}(t_1)^2 t_1 = v^2
\end{equation}

and which allows to price an RFR caplet as if it was a standard Ibor caplet with caplet volatility
$\sigma_\text{eff}(t_1)$ and fixing time $t_1$. We do not use $\sigma_\text{eff}$ to represent caplet volatilities in
bootstrapped or proxied caplet volatility surfaces for RFR underlyings. We also do not report $\sigma_\text{eff}$ as the
pricing vol in the cashflow report. Instead we use $\sigma(t_0)$ in all cases (where we set $\sigma(t_0) := \sigma(0)$
for $t_0 < 0$).

\subsubsection{Pricing of a SIFMA caplet}

The methodology to price a SIFMA caplet is similar to a backward looking arithmetic average overnight caplet.

\subsubsection{Bootstrap of caplet volatilities}
\label{pricing:ir_capfloor_bootstrap}

Quoted par market volatilities for caps are not directly used for pricing. Instead we build an interpolated caplet
surface with nodes

\begin{equation}
  \sigma(t_i, K_j)
\end{equation}

for caplet expiries $t_i$, $i=1,\ldots,n$ and strikes $K_j$, $j=1,\ldots,m$. The caplet volatilities $\sigma(t_i, K_j)$
are determined in a bootstrap procedure such that the premiums of the quoted market caps are reproduced. The pricing of
the caplets both on the interpolated caplet surface and to generate the benchmark market premiums follow the methods
outlined in \ref{pricing:ir_capfloor_ibor_caplet} and \ref{pricing:ir_capfloor_rfr_caplet} for Ibor / forward looking
term rates and backward looking RFR rates. We further note:

\begin{itemize}
\item for Ibor / forward looking term rates the first caplet of a cap is excluded for the purpose of the bootstrap
\item for backward looking RFR rates the first caplet of a cap is included for the purposes of the bootstrap
\item the bootstrapped volatility for an Ibor / forward looking term rate caplet is stored at node $(t_f, K)$ where
  $t_f$ is the fixing time of the caplet and $K$ is the strike.
\item the bootstrapped volatility for a backward looking RFR caplet is stored at node $(t_0, K)$ where $t_0$ is the
  fixing time of the first relevant overnight fixing entering the compounded RFR rate and $K$ is the strike. The stored
  volatility is the value $\sigma(t_0)$ in the sense of the RFR caplet pricing model description above (i.e. it is {\em
    not} the effective volatility $\sigma_\text{eff}$ defined in \ref{pricing:ir_capfloor_effvol}).
\end{itemize}

\subsubsection{Volatility Proxies for caps}
\label{pricing:ir_capfloor_proxying}

For underlyings for which no direct market volatility quotes exist, a proxy caplet volatility surface can be
defined. The proxy surface is based on a bootstrapped caplet volatility surface for another underlying. For example,

\begin{itemize}
\item a USD-SOFR-3M term rate caplet volatility surface can be proxied by a USD-SOFR backward looking RFR caplet surface
  with 3M compounding periods, where the latter is bootstrapped from market quotes
\item a USD-SOFR backward looking RFR caplet surface with 3M compounding periods can be proxied by a USD-LIBOR-3M Ibor
  caplet volatility surface
  \item etc. ... all combinations of Ibor / forward looking and RFR backward looking surface types are allowed to derive
    a proxy caplet surface from a bootstrapped surface
\end{itemize}

When deriving a proxy surface, the base surface is bootstrapped first and the proxy surface is based on the resulting
caplet volatilities of the base surface. The volatility $\sigma_P(t, K)$ for a time $t$ and strike $K$ on the proxy
surface is then given as

$$
\sigma_P(t, K) = \sigma_B(t, K + \Delta K)
$$

where $\sigma_B(\cdot)$ is the caplet volatility function on the base surface and $\Delta K$ is a strike adjustment. The
strike adjustment is computed as

\begin{equation}
F_B(t) - F_P(t)
\end{equation}

where $F_B$ and $F_P$ are the fair forward rates for the base resp. proxy underlying at the ``anchor time'' $t$, which
is given as

\begin{itemize}
\item the fixing time of the unique fixing date for an Ibor / forward looking RFR term rate
\item the time of the first relevant overnight fixing for an backward looking RFR rate
\end{itemize}

For example, if the base surface is a USD-SOFR backward looking RFR caplet volatility surface (with 3M compounding
periods) and the proxy surface is a USD-SOFR-3M term rate surface, the volatility $\sigma_T$ for a USD-SOFR-3M term rate
caplet for strike $K$ with and fixing time $t_f$ and fair rate $F_T$ will be given as

\begin{equation}
\sigma_R(t_f, K + (F_R - F_T))
\end{equation}

where $F_R$ is the fair backward looking USD-SOFR rate for the period $[t_f, t_f + 3M]$ and $\sigma_R(\cdot)$ is the
volatility on the backward looking RFR caplet surface.
