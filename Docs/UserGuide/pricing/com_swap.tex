\subsubsection{Commodity Swap}
\label{pricing:com_swap}

As in Section \ref{pricing:com_forward}, the commodity forwarding
curve is used for respective underlying commodity.
 The net present value of each calculation period in a Commodity Swap 
can be computed following \cite{Clark_2014}:
$$
\NPV_{Period} = \mbox{Quantity}\cdot \omega\cdot \left(\frac{1}{n}\sum_{i=1}^n F(0,T_i) - K\right) \cdot P(T)
$$
where:
\begin{itemize}
\item Quantity: number of units of the underlying commodity
\item $K$: strike price
\item the sum runs over all fixing dates $t_i$ in the calculation
  period, $i=1, ..., n$
\item $F(0,T_i)$: the forward commodity price for maturity $T_i$ seen at
  time 0 (valuation date), $T_i$ is the prompt futures expiry date
  associated with the fixing date $t_i$ if the Swap references Futures
  prices, or it is the fixing date $t_i$ if the  Swap references
  Commodity spot prices.
\item $P(T)$: the discount factor for settlement date $T$ typically 5
  business days after the last fixing date in the period
\item $\omega$: +1 for a long Commodity Swap, -1 otherwise 
\end{itemize}

As the payoff can also be rolled up into a single payment we
generalize the pricing formula, taking the sum over all fixings dates
in all calculation periods
$$
\NPV_{Total} = \mbox{Quantity}\cdot \omega\cdot \frac{1}{N}\sum_{i=1}^N\left( F(0,T_i) - K\right) \cdot P(T^{Set}_i)
$$
where $T^{Set}_i$ denotes the settlement date associated with each fixing,
possibly linked to the following period end or swap maturity.