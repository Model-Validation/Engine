\subsubsection{Generic Total Return Swap / Contract for Difference (CFD)}

A generic total return swap is priced using the accrual method, i.e.

\begin{itemize}
\item The return leg NPV is computed as the difference between the price of the underlying asset today and the price at
  the valuation start date of the current return period. If the asset currency is not equal to the swap currency, both
  prices are converted to the swap currency before taking the difference, using the FX rates at the valuation start date
  and today, respectively. If the underlying asset has paid cashflows (dividends, coupons, amortisation payments) in the
  current return period, the sum of these cashflows is added to the return leg NPV (converted to the swap currency using
  today's FX rate).
\item The funding leg NPV is computed as the accrued interest of the current coupon period.
\item If any additional future cashflows are given, their NPV is the sum of their amounts, converted to the swap
  currency using today's FX rate.
\end{itemize}

The swap NPV is the difference between the return and the funding leg (if the return is received) or vice versa (if the
return is paid). The additional results provided together with the NPV are described in table
\ref{tab:additional_results_generic_trs}. In terms of these result labels, the NPV of the swap is the sum of
assetLegNpv, fundingLegNpv and additionalCashflowLegNpv.

\begin{table}[H]
\begin{center}
\begin{tabular}{|p{5cm}|p{10cm}|}
  \hline
  Result Label & Description \\
  \hline
  underlyingCurrency & The currency of the underlying. \\
  \hline
  returnCurrency & The currency of the return payments. \\
  \hline
  fundingCurrency & The currency of the funding payments. \\
  \hline
  pnlCurrency & The currency in which the pnl is expressed, this is always the funding currency. \\
  \hline
  s0 & The value of the underlying at the start of the current valuation period, in underlyingCurrency. \\
  \hline
  s1 & The value of the underlying today, in underlyingCurrency. \\
  \hline
  fx0 & The FX conversion factor from underlyingCurrency to returnCurrency at the start of the current valuation period.
        If initialPrice is given and in returnCurrency, the value will be 1 instead\\
  \hline
  fx1 & The FX conversion factor from underlyingCurrency to returnCurrency at the end of the current valuation period.  \\
  \hline
  assetLegNpv & The npv of the return leg in return currency. This is s1 x fx1 - s0 x fx0 + underlyingCashflows x fx1 if the leg is received. \\
  \hline
  fundingLegNpv & The npv of the funding leg in funding currency . This is the accrued amount on the funding leg (for a unit notional) x
                  fundingLegNotionalFactor x fundingLegFxFactor, if the leg is received. \\
  \hline
  fundingLegFxFactor & The FX conversion factor to the funding currency at the start of the funding leg period.  \\
  \hline
  fundingLegNotionalFactor & If an initialPrice is given for this funding period, initialPrice x underlyingMultiplier. Otherwise, the underlying price at the start of this funding leg period multiplied by UnderlyingMultiplier. \\
  \hline
  underlyingCashflows & Sum of underlying cashflows paid in the current valuation period (up to and including today), in
                        underlyingCurrency. These can be bond cashflows (for bond underlyings) or dividends (for equity
                        basket underlyings). \\
  \hline
  underlyingMultiplier & For (convertible) bond underlyings, the number of bonds (BondNotional), for equity (option)
                         baskets the quantity. \\
  \hline
  additionalCashflowLegNpv & The sum of future additional cashflows in additional cashflows currency. \\
  \hline
  currentNotional & The current period's notional in pnl currency. This is s0 x fx0 x fx(returnCcy $\rightarrow$ pnlCcy). \\
  \hline
  fxConversionAssetLegNpvToPnlCurrency & Today's FX conversion factor from return currency to pnl currency. \\
  \hline
  fxConversionAdditionalCashflowLegNpvToPnlCurrency & Today's FX conversion factor from additional cashflows currency to pnl currency \\
  \hline
\end{tabular}
\end{center}
\caption{Additional Results Generic Total Return Swap Pricing}
\label{tab:additional_results_generic_trs}
\end{table}


\underline{Rationale for the accrual method}

If the swap contains the right to unwind the position on a daily basis, the accrual method is an obvious choice. In the
following we give a rationale for this method even ignoring these unwind rights to highlight the differences to a full
discounting valuation approach:

Consider one return period with pricing dates $t_1 < t_2$. The return leg pays (assuming it is the receiver leg)

\begin{equation}
  S(t_2) - S(t_1)
\end{equation}

on $t_2$. Here, $S(t)$ denotes the underlying asset price, including all income (e.g. dividends) and - in case of a
composite swap - converted to the return currency using the prevailing market FX spot rate $X(t)$. The funding leg pays
(the minus sign indicating that it is the payer leg)

\begin{equation}
  - \left( S(t_1) e^{\int_{t_1}^{t_2} r(s) ds} - S(t_1) \right)
\end{equation}

on $t_2$ where $r(s)$ is the interest short rate. Notice that the funded amount is $S(t_1)$ due to the resetting nature
of a total return swap. Furthermore, we assume $r(s)$ is determinisitc and the funding rate is consistent with the
market rate, i.e. in practical terms we fund ``at Libor flat'' and there is no fixing risk.

Let's add cashflows to the return leg

\begin{equation}
  -S(t_1) \text{ at }  t_1, S(t_1) \text{ at } t_2
\end{equation}

and to the funding leg

\begin{equation}
 S(t_1) \text{ at }  t_1, -S(t_1) \text{ at } t_2
\end{equation}

which cancel out, i.e. they do not change the trade. The return leg now has cashflows

\begin{equation}
  -S(t_1) \text{ at } t_1, S(t_2) \text{ at } t_2
\end{equation}

and the funding leg has cashflows

\begin{equation}
  S(t_1) \text{ at } t_1, - S(t_1) e^{\int_{t_1}^{t_2} r(s) ds} \text{ at } t_2
\end{equation}

Both legs are obviously worth zero at $t_1$. The valuation of a total return swap therefore reduces to the valuation of
the cashflows for the current return period with pricing dates $t_1 < t < t_2$.

On the return leg we have a future payment $S(t_2)$ at $t_2$ which evaluates to $S(t)$. On the funding leg we have a
future payment $- S(t_1) e^{\int_{t_1}^{t_2} r(s) ds}$ which evaluates to $-S(t_1)e^{\int_{t_1}^{t} r(s) ds}$.
cashflows $-S(t_1)$ to the return leg and $S(t_1)$ to the funding leg, both paid at $t$ we see that the return leg value
can be written as

\begin{equation}
  S(t) - S(t_1)
\end{equation}

and the funding leg value as

\begin{equation}
-\left( S(t_1)e^{\int_{t_1}^{t} r(s) ds} - S(t_1) \right)
\end{equation}

The latter expression represents the accrued amount on the funding leg between $t_1$ and $t$, which is apprxoimated by
the linear accrued amount of the current funding coupon in the practical computation. This shows that under the
assumptions stated above the valuation of a total return swap under the accrual method is equivalent to the usual
no-arbitrage valuation.
