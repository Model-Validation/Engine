\subsubsection{Generic Barrier Option}
\label{pricing:generic_barrieroption}
Generic Barrier Options are priced using a finite-difference solver in a Black-Scholes model. The final payoff $V(T,S(T))$
at the option expiry $T$ dependent on the underlying value $S(T)$ is easily generated, including an additional
transatlantic barrier condition. We use the notation as in table \ref{tab:generic_barrier_pricing_notation}

\begin{table}[h]
  \begin{tabular}{p{3cm}|p{11cm}}
    Variable & Meaning \\ \hline
    $V_{V}(t,S)$ & Value of the option conditional on $(t,S)$, this excludes rebate payments and assumes barrier monitoring from $t$ on\\ \hline
    $V_{NA}(t,S)$ & Value of the final payoff at $t$ conditional on no knock events happening for times $s$ with $s \geq t$ \\ \hline
    $V_{KI}(t,S)$ & Value of the final payoff at $t$ conditional on a KI event but no KO event happening for a time $s$ with $s \geq t$ \\ \hline
    $V_{KO}(t,S)$ & Value of the final payoff at $t$ conditional on a KO event but no KI event happening for a time $s$ with $s \geq t$ \\ \hline
    $V_{KIKO}(t,S)$ & Value of the final payoff at $t$ conditional on a KI event at time $u$ and a KO event at $v$ with $t\leq u<v$) \\ \hline
    $V_{KOKI}(t,S)$ & Value of the final payoff at $t$ conditional on a KO event at time $u$ and a KI event at $v$ with $t\leq u<v$) \\ \hline
  \end{tabular}
  \caption{Values in FD backward solver}
  \label{tab:generic_barrier_pricing_notation}
\end{table}

The idea is that we track future Knock events. For a series of future knock events we collapse neighboring events of the
same type ``in'' and ``out'' into a single event (because only the first is relevant). Also we only look two events into
the future, since this is sufficient for the update rules below. Notice that for this purpose we ignore whether the
events as allowed in this order according to the term sheet, i.e. we track a sequence of KI then KO events in $V_{KIKO}$
even if KoBeforeKi is defined, i.e. the KO event is excluded by the termsheet in this case. Also notice, that $V_{x}$
for $x \in \{ NA, KI, KO, KIKO, KOKI \}$ is the value of the final payoff, i.e. without considering KI or KO events in
the value itself. This is different for $V_{V}$, where we account for exluded barrier events and include the barrier
events in the value.

At expiry we set $V_{V} = V$ if no KI barrier is defined and $V_{V} = 0$ if a KI barrier is defined. Furthermore we set
$V_{NA} = V$ and $V_x = 0$ for $x \in \{ KI, KO, KIKO, KOKI \}$. The rollback from $t_{i+1}$ to $t_i$ works as follows:

If there is no KI or KO event at $t_i$, all values are rolled back without any update.

If there is a KI event at $t_i$, we apply the following updates after the roll back:

\begin{itemize}
\item $V_{KIKO} := V_{KO} + V_{KIKO} + V_{KOKI}$
\item $V_{KOKI} := 0$
\item $V_{KI} := V_{NA} + V_{KI}$
\item $V_{KO} := 0$
\item $V_{NA} := 0$
\item $V_{V} := V_{KI}$, add $V_{KIKO}$ if KoBeforeKi
\end{itemize}

If there is a KO event at $t_i$, we apply the following updates after the roll back:

\begin{itemize}
\item $V_{KOKI} := V_{KI} + V_{KOKI} + V_{KIKO}$
\item $V_{KIKO} := 0$
\item $V_{KO} := V_{NA} + V_{KO}$
\item $V_{KI} := 0$
\item $V_{NA} := 0$
\item $V_{V} := 0$ if KoAlways or KoBeforeKi
\end{itemize}

If the trade is seasoned we continue the rollback into the past to account for past KI and KO events properly. This
rollback does not involve the computation of conditional expectations though, we only update the values according to the
rules above.

Rebates are handled as follows:

The Transatlantic Barrier Rebate is incorporated into the final option payoff, i.e. for
the nodes where the Transatlantic Barrier deactivates the option, a payoff amount equal to the rebate amount is
generated.

If only KO-Barriers are present, the rebate NPV is computed in the PDE by setting $V_{V}$ to the rebate amount (instead
of $0$) on nodes where a KO event is detected. The payment timing is accounted for by appropriate discounting of that
amount.

If at least one KI-Barrier barrier is present, only a unique rebate amount is allowed which is always paid on the option
settlement date. We can roll back this amount as $R_x$ in complete analogy to the $V_x$ for $x \in \{ NA, KI, KO, KIKO,
KOKI \}$. The variables $R_x$ are initialised in the same way as $V_x$, but with the final option payoff replaced with
the rebate payment $R$. The rebate NPV is then computed as the discounted value of $R$ at $t=0$ minus $R_V$ (because
$R_V$ represents an option that pays $R$ at maturity if the original barrier option is active).

When evaluated via Monte-Carlo simulation, the exercise probability and the trigger probability are 
given as additional results. Additional to this, the knock-out probability of the transatlantic feature is given as well if relevant. 
Therein, there are the following definitions active:

\begin{table}[h]
  \begin{tabular}{p{4.5cm}|p{8cm}}
      Additional Result&Description\\ 
      \hline
      ExerciseProbability&The estimated probability of the exercise event. The latter means that the payoff 
      is non-zero at maturity (i.e. the plain vanilla part of the payoff is positive and no knock-out event 
      has accured and necessary knock-in events have occured until then).\\ \hline
      TriggerProbability&The estimated probability of the exercise event. The latter means that the knock-in 
      or the knock-out event (or one of them if there are multiple ones) has been triggered.\\ \hline      
      TransatlanticTriggered&The estimated probability of the transatlantic trigger event. The latter means 
      that the transatlantic knock-out event (or at least one of them if there are multiple ones) has been triggerd.\\ \hline
  \end{tabular}
  \caption{Additional results output labels for the Generic Barrier Option}
  \label{tab:AdditionalResultsGenericBarrier}
\end{table}


