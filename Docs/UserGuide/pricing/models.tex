\section{Pricing Models}
\label{sec:models}

%--------------------------------------------------------------------------------
\subsection{Bachelier Model}
\label{models:bachelier}

Under the Bachelier model it is assumed the asset price S is normally distributed 
and follows an Arithmetic Brownian Motion process (Haug, 1997) Ch 1.3.1:
$$
dS=\sigma(t)\,dW(t)
$$
where $dW(t)$ is a Wiener process. \\

Then, the Bachelier model gives the following analytical solution for the price 
at expiry of a call/put option with strike $K$, on an underlying asset with price $S$, 
and normal volatilities $\sigma$:
$$
\mbox{Bachelier}(K,S,v,\omega)=v\, \phi\left(\omega\,\frac{S-K}{v}\right)
+\omega\, (S-K) \,\Phi\left(\omega\,\frac{S-K}{v}\right)
$$
where
\begin{itemize}
\item $K$: strike price
\item $S$: price of the underlying asset
\item $T$: time to expiration in years
\item  terminal volatility $v$: $\displaystyle v^2(T)=\int_0^T\sigma^2(t)\,dt$ 
\item $\sigma(t)$: instantaneous volatility of the underlying asset price
\item $\omega$: 1 for a call option and -1 for a put option
\item $\Phi(\cdot)$: the cumulative standard normal distribution
\item $\phi(\cdot)$: the standard normal density function
\end{itemize}

%--------------------------------------------------------------------------------
\subsection{Black Model, Shifted Black Model}
\label{models:black}

In Black's model it is assumed that variable $F$ for future times t follows a Geometric 
Brownian Motion:
$$
dF/F=\mu(t)+\sigma(t)\,dW(t)
$$
where $\mu(t)$ is the drift and $dW(t)$ is a Wiener process. If $F$ is a forward price 
or yield, and the Wiener process is assumed in the T-forward measure, then the Geometric 
Brownian Motion is drift-free, $\mu(t)\equiv 0$.

This leads to the Black76 analytic formula, see (Lichters, Stamm, Gallagher, 2015) 
Appendix C.1, for the price at expiry of a call or put option on underlying asset $F$ 
with strike $K$ and log-normal volatilities $\sigma$: 
$$
\mbox{Black}(K,F ,v,\omega)=\omega\,\left\{F\,\Phi(\omega\, d_+) - K\,\Phi(\omega \, d_-)\right\}
$$
where
\begin{itemize}
\item $K$: strike rate
\item $F$: the underlying asset forward price $F$
\item $\sigma(t)$: the instantaneous log-normal volatility of $F$
\item  terminal volatility $v$: $\displaystyle v^2=\int_0^T\sigma^2(t)\,dt$ 
\item $\omega$: 1 for call, -1 for put
\item $\Phi(\cdot)$: the cumulative standard normal distribution
\item $\displaystyle d_+ = \frac{1}{v}\left(\ln\frac{F}{K}+ \frac{1}{2}v^2\right)$
\item $d_-= d_+ - v$
\end{itemize}

%--------------------------------------------------------------------------------
\subsubsection*{Shifted Black model}

To apply the Black model in low- or negative-rate scenarios, one can assume displaced 
Geometric Brownian Motion, introducing a displacement parameter $\alpha\geq 0$ into 
the dynamics
$$
d(F + \alpha) = (F + \alpha)\,\sigma(t)\,dW
$$
This parameter is the same for the entire volatility surface.

The present value of a call or put (assuming displaced GBM) uses 
the same analytical solution as the previous section (Black formula), but with 
forward rate $F$ and strike $K$ amended to include the displacement parameter 
$\alpha$:
\begin{align*}
F_{\mbox{shifted}} &= F + \alpha \\
K_{\mbox{shifted}} &= K + \alpha
\end{align*}

%--------------------------------------------------------------------------------
\subsection{Linear Terminal Swap Rate model (LTSR)}
\label{models:LTSR}

CMS-linked cash flows (in CMS Swaps and CMS Cap/Floors) are priced using Terminal 
Swap Rate Models to replicate the payoff with quoted market Swaptions, i.e.
\begin{align*}
V(0) &= A(0)\,\E(\alpha(S(t))\,f(S(t))) \\
&= A(0)\,\alpha(S(0))\,f(S(0)) +\int_{-\infty}^{S(0)} f'' (k)\,p(k)\,dk 
+ \int_{S(0)}^\infty f''(k)\,c(k)\,dk
\end{align*}
where
\begin{itemize}
\item $V(0)$: the NPV at time zero of a CMS-linked cash flow
\item $\E(\dots)$: the expectation in the physical annuity measure
\item $S(t)$: the relevant Swap rate observed at fixing time $t$
\item $A(t)$: the NPV of the annuity (w.r.t. physical settlement) at fixing 
time $t$
\item $P(t,T)$: the discount factor between fixing time t and payment time $T$
\item $\alpha(s)=\E(P(t,T)/A(t) | S(t) = s)$: the annuity mapping function
\item $f(S(t))$: the payoff depending on $S(t)$ and paid at time $T$
\item $p(k)$: the NPV of a put (receiver Swaption with physical settlement), 
struck at strike $k$
\item $c(k)$: the NPV of a call (payer Swaption with physical settlement), 
struck at strike $k$
\end{itemize}
See (Andersen \& Piterbarg, 2010), Ch. 16.3 and Ch. 16.6. 

The LTSR is specified by the annuity mapping function $\alpha(s)=a\,s+b$, 
where the coefficient $b$ is determined by a ``no arbitrage'' condition and $a$ 
is linked to a one-factor model mean reversion parameter $\kappa$, see 
(Andersen \& Piterbarg, 2010), formulas 16.27 and 16.28, for details. It 
supports normal, log-normal, and shifted log-normal volatilities. Besides the 
mean reversion, the lower and upper integrations bounds are the parameters of 
this model.

For log-normal volatilities, the Hagan model can also be used. Hagan model 
parameters are Mean Reversion and Yield Curve Model. The Yield Curve Model 
parameter can take the following values: Standard, ExactYield, ParallelShifts, 
NonParallelShifts. Each of these models corresponds to a certain specification 
of the annuity mapping function $\alpha(s)$ \cite{Hagan_2003}.

%--------------------------------------------------------------------------------
\subsection{Bivariate swap rate model BrigoMercurio}
\label{models:BrigoMercurio}

CMS Spread Options with payoff $\max(\varphi(S_1(t)-S_2(t)-K),0)$ are priced using 
the the model \cite{Brigo_Mercurio_2006} Ch 13.6.2, optionally with partial smile
adjustment \cite{Berrahoui_2004}. This model supports log-normal swap rate dynamics 
and has been extended to support shifted log-normal and normal dynamics as well
\cite{Caspers_2015}.

The model is given by:
\begin{align*}
dS_1 &= \mu_1 \,(S_1+d_1)\,dt+\sigma_1\, (S_1+d_1)\,dW_1 \\
dS_2 &= \mu_2 \,(S_2+d_2)\,dt+\sigma_2\, (S_2+d_2)\,dW_2 \\
dW_1 \,dW_2 & =\rho \,dt
\end{align*}
in the shifted lognormal case ($d_1=d_2=0$ in the lognormal case), and by
\begin{align*}
dS_1 &=\mu_1 dt + \sigma_1\, dW_1 \\
dS_2 &=\mu_2 dt +\sigma_2\, dW_2 \\
dW_1\,dW_2 &=\rho\,dt
\end{align*}
in the normal case. All equations are in the T-forward measure. 

Here:
\begin{itemize}
\item $t$: fixing Time
\item $T$: payment Time
\item $S_1,S_2$: Swap Rates
\item $K$: strike
\item $\varphi$: +1 for a call, -1 for a put
\item $\mu_1,\mu_2$: implied drifts, see below
\item $\sigma_1,\sigma_2$: swap rate volatilities
\item $\rho$: instantaneous correlation
\end{itemize}
The drifts are determined as deterministic, constant values such that the 
T-forward expectation matches the convexity-adjusted swap rate with identical 
fixing and payment time as the CMS Spread payoff.  The convexity adjustment 
for the single swap rates are computed using one of the models in Section 
\ref{models:LTSR}.

The swap rate volatilities are either taken as the market ATM volatilities of 
the swap rates with the option expiry equal to the fixing time of the CMS Spread 
payoff, or (optionally) as smile volatilities with strikes $S_2+K$ (for $S_1$) and 
$S_1-K$ (for $S_2$), see \cite{Berrahoui_2004} for the rationale of the latter.

The instantaneous correlation is an external input to the model, which can be 
implied from quoted CMS Spread option premiums.

The (shifted) lognormal model allows for a semi-analytical solution involving 
a one-dimensional numerical integration which can be carried out using a 
Gauss-Hermite integration scheme, see \cite{Brigo_Mercurio_2006} Ch 13.6.2 for 
the details.

The normal model can be solved using the Bachelier option pricing formula, see \cite{Caspers_2015}. In this special case
the NPV $\nu$ is given by:

\begin{equation}
P(0,T) \mathcal{B}(\phi, K, \mu, \sigma\sqrt{t})
\end{equation}

Here, $\mathcal{B}$ denotes the Bachelier option pricing formula with strike $K$, forward $\mu$, standard deviation
$\sigma\sqrt{t}$ and $\phi=1$ for a call, $-1$ for a put. The forward is given by

\begin{equation}
\mu = (S_1(0)+\mu_1t) - (S_2(0) + \mu_2t);
\end{equation}

and the variance is given by

\begin{equation}
\sigma^2 = \sigma_1^2t + \sigma_2^2t - 2 \rho\sigma_1\sigma_2t
\end{equation}


%--------------------------------------------------------------------------------
\subsection{One-Factor Linear Gauss Markov model (LGM)}
\label{models:LGM}

The starting point for the one-factor LGM model is the one-factor Hull-White (HW)
model with time-dependent parameters in the bank account measure. 
$$
dr_t = (\theta_t-\lambda_t \,r_t)\,dt + \sigma_t\, dW_t
$$

where:
\begin{itemize}
\item $\theta_t$: long-term mean interest rate
\item $\lambda_t$: mean reversion parameter
\item $r_t$: short interest rate 
\item $\sigma_t$: volatility of the short rate
\item $dW_t$: a Wiener process
\end{itemize}

The LGM model is closely related to the HW model, and it can e.g. be 
obtained from the HW model by means of a change of variables and change of measure. 
In the LGM, both numeraire and zero bond price are closed-form functions 
\begin{align*}
N(t) &= \frac{1}{P(0,t)}\exp\left\{H_t\, z_t + \frac{1}{2}H^2_t\,\zeta_t \right\} \\
P(t,T,z_t)
&= \frac{P(0,T)}{P(0,t)}\:\exp\left\{ -(H_T-H_t)\,z_t - \frac{1}{2} \left(H^2_T-H^2_t\right)\,\zeta_t\right\}.
\end{align*}
of a single Gaussian random variable $z_t$,
$$
dz_t=\alpha_t\, dW_t.
$$
The LGM parameters $H_t$ and $\zeta_t$ can be related to the HW 
model's mean reversion speed $\lambda_t$ and volatility $\sigma_t$ above.
The model's volatility functions are calibrated to Swaptions, choosing 
a limited range of Swaption expiries and underlying Swap maturities from the 
ATM surface or adapted to the deal strike. %(Open Source Risk, 2017)
%
%\medskip
%The full volatility smile (cube) is taken into account where available. LGM can 
%handle both normal and log-normal volatilities 
For details refer to \cite{LSG_2015} Ch. 11.1.

\medskip
Pricing with LGM is done, depending on product type, either
\begin{itemize}
\item semi-analytically using the closed form expressions for numeraire and 
zero bond above
\item using a backward induction approach on a one-dimensional 
LGM lattice, or
\item using Monte Carlo simulation.
\end{itemize}

%--------------------------------------------------------------------------------
\subsection{Barone-Adesi and Whaley Model}
\label{models:barone-adesi}

In the Barone-Adesi and Whaley model, the price of an American option is subdivided 
into a European option component (priced with Black76) and an early exercise premium 
component. Summed up, the prices of these two components give the American option price.   

The early exercise premium is approximated with a free boundary value problem for an 
ordinary differential equation, which in turn, is solved numerically by quadratic 
approximation. 

The starting point for the approximation is the Black-Scholes partial differential 
equation for the value of an FX Option with price $V$:
$$
\frac{1}{2} \sigma^2\, S^2 \frac{\partial^2 V}{\partial S^2} + 
(r_d -r_f)\, S\, \frac{\partial V}{\partial S} - r_d\, V
+\frac{\partial V}{\partial t} = 0
$$
where:
\begin{itemize}
\item $V$: the option price
\item $S$: the underlying spot FX rate between domestic and foreign currencies
\item $\sigma$: the volatility of the FX rate
\item $r_d$: the domestic interest rate
\item $r_f$: the foreign interest rate
\end{itemize}

For American options, there is no known analytic solution to the differential equation 
above. Barone-Adesi and Whaley decomposes the American option price into the European 
price and the early exercise premium:
$$
C_{Amer} (S,T)= C_{Euro} (S,T) + E_c (S,T)
$$
where:
\begin{itemize}
\item $C_{Amer} (S,T)$: the price of an American FX call option on underlying FX 
rate $S$ at expiry $T$
\item $C_{Euro} (S,T)$: the price of a European FX call option on underlying FX 
rate $S$ at expiry $T$
\item $E_c (S,T)$: the early exercise premium of the American FX call option
\end{itemize}

Using that $E_c (S,T)$ also satisfies the Black-Scholes partial differential equation, 
and removing terms involving $\partial V / \partial t$ that are negligible, it is found 
that:
$$
C_{Amer} (S,T)=\left\{ \begin{array}{lcl}
C_{Euro}(S,T) + A_2\left(\frac{S}{S^*}\right)^{q_2} &;& S < S^*\\
S-K &;& S\geq S^* 
\end{array}
\right.
$$
where:
\begin{itemize}
\item $\displaystyle A_2=\frac{S^*}{q_2} \left(1-e^{-r_f (T-t)}\right)\,\Phi(d_1\,S^*)$
\item $\displaystyle q_2=\frac{1}{2}\left(-(\beta-1)+\sqrt{(\beta-1)^2+4\alpha/h}\right)$
\item $\displaystyle \alpha =\frac{2\,r_d}{\sigma^2}$ 
\item $\displaystyle \beta=2\frac{r_d-r_f}{\sigma^2}$ 
\item $\displaystyle h(T)=1-e^{-r_d (T-t)}$
\item $\displaystyle d_1(S)=\frac{1}{\sigma\sqrt{T-t}}\,
	\left(\ln\frac{S}{K}+\left(r_d - r_f + \frac{\sigma^2}{2}\right)(T-t)\right)$
\item $S^*$: the FX rate below which the option should be exercised
\item $K$: the strike FX rate
\item $S^*$ solves:
$\displaystyle S^*-K = C_{Euro}(S^*,T) + \frac{S^*}{q_2}  \left(1-e^{-r_f (T-t)}\right) \Phi(d_1 \,S^*)$
\end{itemize}

For more details refer to the original Barone-Adesi \& Whaley paper \cite{Barone-Adesi_1987}.

