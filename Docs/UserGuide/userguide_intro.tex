\section{Introduction}

The {\em Open Source Risk Project} \cite{ORE} aims at providing a transparent platform for pricing and risk analysis
that serves as
%\medskip
\begin{itemize}
\item a benchmarking, validation, training, and teaching reference,
\item an extensible foundation for tailored risk solutions.
\end{itemize}

Its main software project is {\em Open Source Risk Engine} (ORE), an application that provides
\begin{itemize}
\item a Monte Carlo simulation framework for contemporary risk analytics and value adjustments
\item simple interfaces for trade data, market data and system configuration
\item simple launchers and result visualisation in Jupyter, Excel, LibreOffice
\item unit tests and various examples.  
\end{itemize}
ORE is open source software, provided under the Modified BSD License. It is based 
on QuantLib, the open source library for quantitative finance \cite{QL}.

%\medskip
\subsubsection*{Audience}
The project aims at reaching quantitative risk ma\-nage\-ment practitioners (be it in financial institutions, audit
firms, consulting companies or regulatory bodies) who are looking for accessible software solutions, and quant
developers in charge of the implementation of pricing and risk methods similar to those in ORE. Moreover, the project
aims at reaching academics and students who would like to teach or learn quantitative risk management using a freely
available, contemporary risk application. And in the meantime, as ORE is used in risk services at industrial scale since 2018
with over 150 users, ORE is aimed at firms that consider the replacement of third party licensed software. 

\subsubsection*{Contributions}
Quaternion Risk Management \cite{QRM} has been committed to sponsoring the Open Source Risk project through ongoing project
administration, through providing an initial release and a series of subsequent releases in order to achieve a wide
analytics, product and risk factor class coverage. Since Quaternion's acquisition by Acadia Inc. in February 2021,
Acadia \cite{acadia} has been committed to continue the sponsorship. The Open Source Risk project work continues with
former Quaternion operating as Acadia's Quantitative Services unit. And with Acadia's acquisiton by London Stock
Exchange Group (LSEG) \cite{lseg} in 2023, the journey continues under the roof of LSEG Post Trade Solutions.

The community is invited to contribute to ORE, through feedback, discussions and suggested enhancements. Our forum 
for that has moved to github, \url{https://github.com/OpenSourceRisk/Engine/discussions}. Issues can be reported at
\url{https://github.com/OpenSourceRisk/Engine/Issues}. And contributions to the source code can be submitted via
pull requests at \url{https://github.com/OpenSourceRisk/Engine/pulls}.
See also the FAQ section on the ORE site \cite{ORE} on how to get involved.

\subsection{Scope}

ORE currently provides portfolio pricing, cash flow generation, market risk analysis and a range of contemporary derivative
portfolio analytics. The latter are based on a Monte Carlo simulation framework which yields the evolution of various exposure measures:
\begin{itemize}
\item EE aka EPE (Expected Exposure or Expected Positive Exposure)
\item ENE (Expected Negative Exposure, i.e. the counterparty's perspective)
\item 'Basel' exposure measures relevant for regulatory capital charges under internal model methods 
\item PFE (Potential Future Exposure at some user defined quantile)
\end{itemize}
and derivative value adjustments (xVA)
\begin{itemize}
\item CVA (Credit Value Adjustment)
\item DVA (Debit Value Adjustment)
\item FVA (Funding Value Adjustment)
\item COLVA (Collateral Value Adjustment)
\item MVA (Margin Value Adjustment)
\end{itemize}
for portfolios with netting, variation and initial margin agreements. 

\medskip
The market risk framework provides
\begin{itemize}
\item sensitivity analysis, also in the ``par'' domain
\item stress testing, also in the ``par'' domain
\item several parametric VaR versions (Delta VaR, Delta-Gamma Normal VaR, Delta-Gamma VaR with Cornish-Fisher expansion and Saddlepoint method)
\item historical simulation VaR
\item P\&L and P\&L explain
\end{itemize}
across all asset classes and products. 

\medskip
Thanks to Acadia's open-source strategy, ORE's financial instrument scope was extended beyond the initial vanilla scope with
a series of quarterly releases since version 7 to cover
\begin{itemize}
\item "First Generation" Equity and FX Exotics, released September with ORE v7
\item Commodity products (Swaps, Basis Swaps, Average Price Options, Swaptions), released December 22 with ORE v8
\item Credit products (Index CDS and Index CDS Options, Credit-Linked Swaps, Synthetic CDOs), released March 23 with ORE v9
\item Bond products and Hybrids (Bond Options, Bond Repos, Bond TRS, Composite Trades, Convertible Bonds, Generic TRS with mixed basket underlyings, CFDs), released in June 23 with ORE v10
\item a Scripted Trade framework in October 23 with ORE v11: This allows the modelling of complex hybrid payoffs
  such as Accumulators, TARFs, PRDCs, Basket Options, etc, across IR, FX, INF, EQ, COM classes.  Scripted Trades are fully
  integrated into the market risk and exposure simulation frameworks, supported by American Monte Carlo methods for pricing
  and exposure simulation. The user can now extend the instrument scope conveniently by adding payoff scripts (embedded into
  the trade XML or in separate script "library" XML) and without recompiling the code base.
\item Formula-based legs, Callable Swaps, Flexi Swaps, Balance Guaranteed Swaps and American Swaptions in May 24 with ORE v12
\end{itemize}

These contributions were accompanied by analytics extensions to enhance ORE usability
\begin{itemize}
\item Exposure simulation for xVA and PFE, adding Commodity to the asset class coverage, and adding American Monte Carlo for Exotics, released in December 22 with ORE v8
\item Market Risk including multi-threaded sensitivity analysis, par sensitivity conversion, parametric delta/gamma VaR with Cornish-Fisher expansion and Saddlepoint method, released in March 23 with ORE v9
\item Portfolio Credit Model, released in June 23 with ORE v10
\item ISDA's Standard Initial Margin Model (SIMM), released in June 23 with ORE v10
\item Historical Simulation VaR, P\&L and P\&L Explain, released in May 24 with ORE v12
\end{itemize}

Recent analytics additions are
\begin{itemize}
\item XVA Risk with sensitivities, P\&L and P\&L explain
\item Regulatory Capital for CCR and CVA Risk (SA-CCR, BA-CVA, SA-CVA)
\item Dynamic SIMM based on path-wise sensitivities computed with Algorithmic Differentiation (AAD)
\end{itemize}

\medskip 
The product coverage of the latest release of ORE is sketched in Table \ref{tab_coverage}.
\begin{table}[hbt]
\scriptsize
\begin{center}
\begin{tabular}{|l|p{1.5cm}|p{1.5cm}|p{1.2cm}|p{1.5cm}|}
\hline
Product & Pricing and Cashflows & Sensitivity Analysis & Stress Testing & Exposure Simulation \& XVA\\
\hline
Fixed and Floating Rate Bonds/Loans & Y & Y & Y & N \\
\hline
Interest Rate Swaps & Y & Y & Y & Y\\
\hline
Caps/Floors & Y & Y & Y & Y\\
\hline
Swaptions, Callable Swaps & Y & Y & Y &Y \\
\hline
Constant Maturity Swaps, CMS Caps/Floors & Y & Y & Y & Y\\
\hline
FX Forwards and Average Forwards & Y & Y & Y & Y \\
\hline
Cross Currency Swaps & Y & Y & Y & Y \\
\hline
FX European and Asian Options & Y & Y & Y & Y\\
\hline
FX Exotic Options (see below) & Y & Y & Y & Y\\
\hline
Equity Forwards & Y & Y & Y & Y\\
\hline
Equity Swaps & Y & Y & Y & N\\
\hline
Equity European and Asian Options & Y & Y & Y & Y \\
\hline
Equity Exotic Options (see below)  & Y & Y & Y & Y \\
\hline
Equity Future Options & Y & Y & Y & Y \\
\hline
Commodity Forwards and Swaps & Y & Y & Y & Y\\
\hline
Commodity European and Asian Options & Y & Y & Y & Y \\
\hline
Commodity Digital Options & Y & Y & Y & Y \\
\hline
Commodity Swaptions & Y & Y & Y & Y\\
\hline
CPI Swaps & Y & Y & N & Y \\
\hline
CPI Caps/Floors & Y & Y & N & N\\
\hline
Year-on-Year Inflation Swaps & Y & Y & N & Y \\
\hline
Year-on-Year Inflation Caps/Floors & Y & Y & N & N\\
\hline
Credit Default Swaps, Options & Y & Y & N & Y \\
\hline
Index Credit Default Swaps, Options & Y & Y & N & Y \\
\hline
Credit Linked Swaps & Y & Y & N & Y \\
\hline
Index Tranches, Synthetic CDOs & Y & Y & N & Y \\
\hline
Composite Trades & Y & Y & Y & Y \\
\hline
Total Return Swaps and Contracts for Difference & Y & Y & Y & Y \\
\hline
Convertible Bonds & Y & Y & Y & N \\
\hline
ASCOTs & Y & Y & Y & Y \\
\hline
Scripted Trades & Y & Y & Y & Y \\
\hline
Flexi Swaps and Balance Guaranteed Swaps & Y & Y & Y & Y \\
\hline
\end{tabular}
\caption{ORE product coverage. FX/Equity Exotics include Barrier, Digital, Digital Barrier (FX only), Double Barrier, European Barrier, KIKO Barrier (FX only), Touch and Double Touch Options, Outperformance options and Pairwise Variance Swaps. Scripted Trades cover single and multi-asset products across all asset classes except Credit (so far), see Example\_52 and the separate documentation in Docs/ScriptedTrade.}
\label{tab_coverage}
\end{center}
\end{table}

\medskip The simulation models applied in ORE's risk factor evolution implement the models discussed in detail in {\em
  Modern Derivatives Pricing and Credit Exposure Analysis} \cite{Lichters}: The IR/FX/INF/EQ risk factor evolution is based on
a cross currency model consisting of an arbitrage free combination of Linear Gauss Markov models for all interest rates
and lognormal processes for FX rates and EQ prices, Dodgson-Kainth (or Jarrow-Yildirim) models for inflation. The model components are calibrated to cross currency discounting and forward curves, Swaptions, FX Options, EQ Options and CPI caps/floors. With the 8th release, Commodity simulation has been added, as well as the foundation for a multi-factor Hull-White based IR/FX/COM simulation model. 

\subsection{ORE in Python or Java}
\label{sec:intro_wheels}

ORE is written in C++ and comes with a command line executable {\tt ore.exe} that supports batch processes. 
But since early versions of ORE we also provide language bindings following QuantLib's example using SWIG, in ORE's case with focus on Python and Java modules. 
The ORE SWIG module extends (contains) the QuantLib SWIG module and offers moreover access to a part of ORE's functionality.
Since ORE v9, Python {\em wheels} are provided for each release, so that users can install the most recent ORE Python module by calling

\medskip
\centerline{\tt pip install open-source-risk-engine}
 
\medskip
See section \ref{example:orepython} on how to use ORE-Python. 

\medskip
Note that, technically, the ORE SWIG module source code has moved into the ORE Engine repository with release 13, see directory ORE-SWIG there. And we have removed the separate repository formerly located at https://github.com/OpenSourceRisk/ORE-SWIG.

\subsection{Roadmap}

ORE grows with community contributions and the demand of clients who utilise ORE to replace existing risk applications with the help of the Quant Services team at Acadia/LSEG.

Moreover, it is generally planned that subsequent ORE releases will extend the scope of the {\bf Regulatory Capital} analytics in ORE, with
\begin{itemize}
\item broader product scope of SA-CCR and BA-CVA
\item addition of Market Risk capital (FRTB-SA)
\end{itemize}

{\bf Performance:} ORE v12 contains applications of {\bf AAD} for sensitivity analysis, CVA sensitivity (proof-of concept stage),
as well as applications of {\bf GPUs} to parallelize computations (see legacy Examples 56 and 61), with significant
speed-ups. Both areas are active work in progress, and further enhancements and tests should be released with the next versions.

{\bf ORE Python:} Moreover, there is demand among ORE users for extended coverage of the ORE-Python version, so that we also expect steady growth
of the Python wrapper around ORE.

{\bf ORE Service:} ORE v12 contains an implementation of a restful API around ORE, see folder ore/Api and the example therein.
This is written in Python, uses the ORE-Python module and the Flask web framework. This implementation is proof of concept,
for demonstration purposes and to encourage the community to extend and contribute alternatives.

\subsection{Further Resources}
\begin{itemize}
\item Open Source Risk Project site: \url{http://www.opensourcerisk.org}
\item Source code and releases: \url{https://github.com/opensourcerisk/engine}
% Relocated into the engine repository
%\item Language bindings: \url{https://github.com/opensourcerisk/ore-swig}
\item Frequently Asked Questions: \url{http://www.opensourcerisk.org/faqs}
% Relocated to github discusions
%\item Forum: \url{http://www.opensourcerisk.org/forum}
\item Follow ORE on Twitter {\tt @OpenSourceRisk} for updates on releases and events
\item ORE Product Catalogue \cite{products}
\item ORE Methodology \cite{methods}
\item ORE Academy on youtube: \url{https://www.youtube.com/channel/UCrCpkb1-s3pxKd7U-YgJulA}
\end{itemize}
 
\subsubsection*{Organisation of this document}

This document focuses on instructions how to use ORE to cover basic workflows from individual deal analysis to portfolio
processing. After an overview over the core ORE data flow in section \ref{sec:process} and installation instructions in
section \ref{sec:installation} we start in section \ref{sec:examples} with a series of examples that illustrate how to
launch ORE using its command line application, and we discuss typical results and reports.
%We then illustrate in section \ref{sec:visualisation} interactive analysis of resulting 'NPV cube' data. The final sections of this text document ORE parametrisation.

The description of products, pricing and trade representration in ORE XML has been carved out and moved to \cite{products}.

And finally, a summary of methodologies applied in ORE can be found in \cite{methods}.
