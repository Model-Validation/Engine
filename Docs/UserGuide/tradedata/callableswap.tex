\subsubsection{Callable Swap}\label{ss:callable_swap}

\ifdefined\IncludePayoff{{\bf Payoff}

A Callable Swap can be terminated by one of the parties on specific dates and thus can
be decomposed into a swap and a physically settled swaption, the latter representing the call
right. The underlying swap must be fixed versus float and may have varying notional, rates,
and spreads during its lifetime.

{\bf Input}}\fi

The \lstinline!CallableSwapData! node is the trade data container for the \emph{CallableSwap} trade type. A Callable Swap is a
 swap that can be cancelled at predefined dates by one of the counterparties. A Callable Swap must have at least one leg, each leg described by a \lstinline!LegData! trade component sub-node as described in section
\ref{ss:leg_data}.  

Unless MidCouponExercise is \emph{true}, there must be at least one full coupon period after the exercise date for European Callable Swaps, and after the last exercise date for Bermudan and American Callable Swaps. 

The \lstinline!CallableSwapData! node also contains an \lstinline!OptionData! node which describes
the exercise dates and specifies which party holds the call right, see \ref{ss:option_data}. An example structure of a
\lstinline!CallableSwapData! node is shown in Listing \ref{lst:callableswap_data}.

\begin{listing}[H]
%\hrule\medskip
\begin{minted}[fontsize=\footnotesize]{xml}
<CallableSwapData>
    <OptionData>
      <LongShort>Short</LongShort>
      <Style>Bermudan</Style>
      <Settlement>Physical</Settlement>
      <MidCouponExercise>true<MidCouponExercise>
      <ExerciseDates>
        <ExerciseDate>2031-10-01</ExerciseDate>
        <ExerciseDate>2032-10-01</ExerciseDate>
        <ExerciseDate>2033-10-01</ExerciseDate>
      </ExerciseDates>
      ...
    </OptionData>
  <LegData>
	<LegType>Fixed</LegType>
        <Payer>false</Payer>    
        <Currency>USD</Currency>	
	...
  </LegData>
  <LegData>
	<LegType>Floating</LegType>
        <Payer>true</Payer>     
        <Currency>USD</Currency>	
	...
  </LegData>
</CallableSwapData>
\end{minted}
\caption{Callable Swap data}
\label{lst:callableswap_data}
\end{listing}



The meanings and allowable values of the elements in the \lstinline!CallableSwapData!  node follow below.

\begin{itemize}

\item OptionData: This is a trade component sub-node outlined in section \ref{ss:option_data}. The exercise dates
  specify the dates on which one of the counterparties may terminate the swap. The counterpart holding the call right is
  specified by the {\tt LongShort} flag. The Settlement should be set to \emph{Physical} always. See also the OptionData
  node outlined for a Swaption - see \ref{ss:swaption}, which is identical for a CallableSwap with the exception of the
  requirement that Settlement must be \emph{Physical}, and that the leg directions on a CallableSwap are from the perspective of the client, whereas they are from the perspective of the party that is long on a Swaption. A callable swap can be marked as exercised as explained in
  \ref{ss:swaption} using the \lstinline!ExerciseData! node within OptionData.

\item LegData: This is a trade component sub-node described in section \ref{ss:leg_data} outlining each leg of the
  underlying Swap. A Callable Swap must have at least one leg on the underlying Swap, but can have multiple legs,
  i.e. multiple \lstinline!LegData! nodes.  The LegType elements must be of types \emph{Floating}, \emph{Fixed} or
  \emph{Cashflow}. All legs must have the same \lstinline!Currency!.
  
  Note that the direction of the legs, determined by the \lstinline!Payer! tag, is like for a Swap, from the perspective of the party to the trade. I.e. unlike for a Swaption where the direction of the legs is from the perspective of the party that is long. 

\end{itemize}
