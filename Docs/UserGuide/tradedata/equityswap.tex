\subsubsection{Equity Swap}
\label{ss:equity_swap}

\ifdefined\IncludePayoff{{\bf Payoff}

An equity swap is a swap where one of the legs has a floating rate with 
coupon payments linked to an equity price, either a single stock or equity 
index. 

\begin{itemize}
\item The non equity leg is either a fixed rate or a floating rate based off an 
IBOR index.
\item The notional amount for the equity leg can be either fixed, or resettable 
at specified dates - this equates to a fixed quantity of shares in the underlying 
equity.
\item There are two types of equity coupons, Price Return and Total Return. 
Price Return coupons pay the difference in equity price between the coupon 
start and end dates, while a Total Return coupon also includes dividend 
payments during the period.
\item The equity price can optionally be converted from the equity ccy to the equity leg ccy
  using the FX spot rate observed at the equity fixing dates.
\end{itemize}

{\bf Input}}\fi

An Equity Swap uses its own trade type  \emph{EquitySwap}, and is set up using a {\tt EquitySwapData} node with one leg of type  \emph{Equity} and one more leg - called Funding leg -  that can be either \emph{Fixed} or  \emph{Floating}. Listing \ref{lst:equityswap} shows an example. The
Equity leg contains an additional {\tt EquityLegData} block. See \ref{ss:equitylegdata} for details on the Equity leg specification.

Note that the  \emph{Equity} leg of an \emph{EquitySwap} can only include one single underlying equity name (that can be an equity index name). For instruments with more than one underlying equity name, TradeType \emph{TotalReturnSwap} (GenericTRS) should be used instead.

Cross currency \emph{EquitySwaps} are supported, i.e. the Equity and the Funding legs do not need to have the same currency. However, if the Funding leg uses \lstinline!Indexings! with \lstinline!FromAssetLeg! 
 set to \emph{true} to derive the notionals from the Equity leg, then the Funding leg must use the same currency as the Equity leg.
 
Note that pricing for an \emph{EquitySwap} is based on discounted cashflows, whereas pricing for a \emph{TotalReturnSwap} (GenericTRS) on an equity underlying uses the accrual method. The accrual method is common practice when daily unwind rights are present in the trade terms.


Also note that, unlike other leg types, the {\tt DayCounter} field is optional for an \emph{Equity} leg, and defaults to \emph{ACT/365} if left blank or omitted. The daycount convention for the equity leg of an equity swap does not impact pricing, only the accrued amount (displayed in cashflows).


\begin{listing}[H]
%\hrule\medskip
\begin{minted}[fontsize=\footnotesize]{xml}
    <EquitySwapData>
      <LegData>
        <LegType>Floating</LegType>
        <Payer>true</Payer>
        <DayCounter>ACT/365</DayCounter>
        ...
      </LegData>
      <LegData>
        <LegType>Equity</LegType>
        <Payer>false</Payer>
        <DayCounter>ACT/365</DayCounter>
        ...
        <EquityLegData>
        ...
        </EquityLegData>
      </LegData>
    </EquitySwapData>
\end{minted}
\caption{Equity Swap Data}
\label{lst:equityswap}
\end{listing}

If the equity swap has a resetting notional, typically the Funding leg's notional will be aligned with the equity leg's
notional. To achieve this, \lstinline!Indexings! on the floating leg can be used, see \ref{ss:indexings}. In the context of equity swaps the indexings can be defined in a simplified way by adding an \lstinline!Indexings! node with a subnode \lstinline!FromAssetLeg! set to \emph{true} to the Funding leg's \lstinline!LegData! node. The \lstinline!Notionals! node is not required  in the Funding leg's LegData in this
case. An example is shown in listing \ref{lst:equityswap_reset}.

\begin{listing}[H]
\begin{minted}[fontsize=\footnotesize]{xml}
    <EquitySwapData>
      <LegData>
        <LegType>Floating</LegType>
        <Currency>USD</Currency>
        ...
        <!-- Notionals node is not required, set to 1 internally -->
        ...
        <Indexings>
          <!-- derive the indexing information (equity price, FX) from the Equity leg -->
          <FromAssetLeg>true</FromAssetLeg>
        </Indexings>
      </LegData>
      <LegData>
        <LegType>Equity</LegType>
          <Currency>USD</Currency>
          ...
          <EquityLegData>
            <Quantity>1000</Quantity>
		<Underlying>
		 <Type>Equity</Type>
		 <Name>.STOXX50E</Name>
		 <IdentifierType>RIC</IdentifierType>
		</Underlying>
            <InitialPrice>2937.36</InitialPrice>
            <NotionalReset>true</NotionalReset>
            <FXTerms>
              <EquityCurrency>EUR</EquityCurrency>
              <FXIndex>FX-ECB-EUR-USD</FXIndex>
            </FXTerms>
          </EquityLegData>
          ...
      </LegData>
    </EquitySwapData>
\end{minted}
\caption{Equity Swap Data with notional reset and FX indexing}
\label{lst:equityswap_reset}
\end{listing}

\subsubsection{Dividend Swap}
\label{ss:dividend_swap}

An Dividend Swap uses its the trade type \emph{EquitySwap}, shown above \ref{ss:equity_swap}, and is set up using a {\tt EquitySwapData} node with one leg of type  \emph{Equity}, with \emph{ReturnType} equal to \emph{Dividend} and one more leg that can be either \emph{Fixed} or  \emph{Floating}. Listing \ref{lst:dividendswap} shows an example.

An example is shown in listing \ref{lst:equityswap_reset}.

\begin{listing}[H]
\begin{minted}[fontsize=\footnotesize]{xml}
    <EquitySwapData>
        <LegData>
            ...
        </LegData>
        <LegData>
            <Payer>false</Payer>
            <LegType>Equity</LegType>
            <Currency>EUR</Currency>
            <PaymentConvention>Following</PaymentConvention>
            <DayCounter>A360</DayCounter>
            <EquityLegData>
                <ReturnType>Dividend</ReturnType>
                <Underlying>
		 <Type>Equity</Type>
		 <Name>.STOXX50E</Name>
		 <IdentifierType>RIC</IdentifierType>
		</Underlying>
	        <Quantity>10000</Quantity>
            </EquityLegData>
            <ScheduleData>
                <Rules>
                    <StartDate>2018-12-31</StartDate>
                    <EndDate>2020-12-31</EndDate>
                    <Tenor>6M</Tenor>
                    <Calendar>EUR</Calendar>
                    <Convention>ModifiedFollowing</Convention>
                    <Rule>Forward</Rule>
                </Rules>
            </ScheduleData>
        </LegData>
    </EquitySwapData>
\end{minted}
\caption{Dividend Swap Data}
\label{lst:dividendswap}
\end{listing}
