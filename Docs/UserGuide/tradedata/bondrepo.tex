\subsubsection{Bond Repo}

In a bond repo transaction one party A receives a cash amount from a party B for a specified period. At the maturity of
the trade party A pays back the cash amount plus accrued interest to party B. Intermediate interest payments are also
possible. Party A delivers a bond to party B as a collateral for the received cash amount for the duration of the
trade. In exchange the interest to be paid by party A will be lower than for an uncollateralised borrowing transaction.

A bond repo trade is set up using the trade type \verb+BondRepo+ and a \verb+BondRepoData+ block as shown in listing
\ref{lst:bondrepodata}. The block contains two nodes

\begin{itemize}
\item \verb+BondData+, which specifies the underlying bond and its quantity, and
\item \verb+RepoData+, which specifies the cash leg of the repo
\end{itemize}

The \verb+BondData+ block contains the following fields

\begin{itemize}
\item SecurityId: The identified of the underlying security.\\
  Allowable values: A valid key, usually of the form ``ISIN::XY012345679''
\item BondNotional: The notional of the underlying bond. This is the effective notional used as collateral, i.e. it
  should include hair cuts. Usually the number Bond Notional x Bond Dirty Price x (1 - Haircut) will correspond to the
  nominal on the cash leg at trade inception.\\
  Allowable values: Any positive real number.
    \item CreditRisk [Optional] Boolean flag indicating whether to show Credit Risk on the Bond product. \\
      Allowable Values: \emph{true} or \emph{false} Defaults to \emph{true} if left blank or omitted.    
\end{itemize}

In this case the details of the underlying bond is read from the reference data. It is also possible to inline the
details in the trade, see \ref{ss:bond} for more details on this.

The \verb+RepoData+ block contains exactly one \verb+LegData+ subnode that describes the payments on the cash leg of the
repo, see \ref{ss:leg_data} for details on how to set this up. The \verb+Payer+ leg determines whether interest is paid
(regular repo) or received (reversed repo).

\begin{listing}[H]
%\hrule\medskip
\begin{minted}[fontsize=\footnotesize]{xml}
<BondRepoData>
  <BondData>
    <SecurityId>ISIN:US912828X703</SecurityId>
    <BondNotional>27807597.777444</BondNotional>
  </BondData>
  <RepoData>
    <LegData>
      <LegType>Fixed</LegType>
      <Payer>true</Payer>
      <Currency>USD</Currency>
      <Notionals>
        <Notional>28371510.00</Notional>
      </Notionals>
      <ScheduleData>
        <Rules>
          <StartDate>2020-01-06</StartDate>
          <EndDate>2020-04-07</EndDate>
          <Tenor>1Y</Tenor>
          <Calendar>US</Calendar>
          <Convention>MF</Convention>
          <TermConvention>MF</TermConvention>
          <Rule>Forward</Rule>
          <EndOfMonth/>
          <FirstDate/>
          <LastDate/>
        </Rules>
      </ScheduleData>
      <DayCounter>A360</DayCounter>
      <PaymentConvention>F</PaymentConvention>
      <FixedLegData>
        <Rates>
          <Rate>0.0178</Rate>
        </Rates>
      </FixedLegData>
    </LegData>
  </RepoData>
</BondRepoData>
\end{minted}
\caption{Bond Repo Data}
\label{lst:bondrepodata}
\end{listing}
