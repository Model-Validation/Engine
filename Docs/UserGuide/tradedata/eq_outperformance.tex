\subsubsection{Equity Outperformance Option}

\ifdefined\IncludePayoff{{\bf Payoff}

An Equity Outperformance option has a payoff that depends on the `outperformance' of two equity indices (i.e. the difference between 
their returns) against a strike return.
The buyer has the right but not the obligation to receive the outperformance in 
exchange for the strike rate at a predetermined time in the future.	

Payoff:
$$
N\cdot \max(0,R_1-R_2 - K)
$$
where $N$ is the notional amount, $R$ is the return of an index,
and $K$ is the strike.

A knockIn and knockOut price may be provided. 
The payoff is then only paid if on the exercise date the price of Index2 is above the knockIn price and below the knockOut price.

The pricing methodology was generalized from \cite{Brigo_Mercurio_2006}, 13.16.2.

{\bf Input}}

\else

An Equity Outperformance option has a payoff that depends on the `outperformance' of two equity indices (i.e. the difference between 
their returns) against a strike return. The buyer has the right but not the obligation to receive the outperformance in 
exchange for the strike rate at a predetermined time in the future.

The trade may optionally have a knockIn or knockOut price (or both). 
Only if the price of Underlying2 is above the knockIn value or below the knockOut value is the payoff paid.

\fi

The \lstinline!EquityOutperformanceOptionData! node is the trade data container for the \emph{EquityOutperformanceOption} trade type. The \lstinline!EquityOutperformanceOptionData!  node includes one  \lstinline!OptionData! trade component sub-node plus elements
specific to the Equity Outperformance Option. 

The structure of an example \lstinline!EquityOutperformanceOptionData! node for an Equity Outperformace Option is shown in Listing
\ref{lst:eqoutperformaceoption_data}.

\begin{listing}[H]
%\hrule\medskip
\begin{minted}[fontsize=\footnotesize]{xml}
        <EquityOutperformanceOptionData>
            <OptionData>
              <LongShort>Long</LongShort>
              <OptionType>Call</OptionType>
              <Style>European</Style>
              <Settlement>Cash</Settlement>
              <ExerciseDates>
                <ExerciseDate>2022-09-21</ExerciseDate>
              </ExerciseDates>
              ...
            </OptionData>
          <Currency>USD</Currency>
          <Notional>500000</Notional>
          <Underlying1>
            <Type>Equity</Type>
            <Name>RIC:.SPX</Name>
          </Underlying1>
          <Underlying2>
            <Type>Equity</Type>
            <Name>RIC:.NDX</Name>
          </Underlying2>
          <InitialPrice1>2140</InitialPrice1>
          <InitialPrice2>13000</InitialPrice2>
          <StrikeReturn>0.01</StrikeReturn>
          <KnockInPrice>12500</KnockInPrice>
          <KnockOutPrice>14000</KnockOutPrice>
        </EquityOutperformanceOptionData>
\end{minted}
\caption{Equity Outperformance Option Data}
\label{lst:eqoutperformaceoption_data}
\end{listing}

The Payoff is:
$$
N\cdot \max(0,R_1-R_2 - K)
$$
where:
\begin{itemize}
  \item $N$ is the notional amount
  \item $R_1$ is the return of \lstinline!Underlying1!
  \item $R_2$ is the return of \lstinline!Underlying2!
  \item $K$ is the \lstinline!StrikeReturn!.
\end{itemize}
The meanings and allowable values of the elements in the \lstinline!EquityOutperformaceOptionData!  node follow below.

\begin{itemize}
\item OptionData: This is a trade component sub-node outlined in section \ref{ss:option_data}. The relevant fields in the \lstinline!OptionData! node for an EquityOutperformanceOption are:

\begin{itemize}
\item \lstinline!LongShort! The allowable values are \emph{Long} or \emph{Short}.

\item \lstinline!OptionType! The allowable values are \emph{Call} or \emph{Put}.  \emph{Call} means that the holder has the right but not obligation to receive the Outperformance and pay the StrikeReturn. \emph{Put} means that the holder has the right but not obligation to pay the Outperformance and receive the StrikeReturn.

\item  \lstinline!Style! The allowable value is \emph{European}. Note that the Equity Outperformance Option type allows for \emph{European} option style only. 

\item  \lstinline!Settlement! The allowable values are \emph{Cash} or \emph{Physical}.

%\item \lstinline!PayoffatExpiry! The allowable values are \emph{true} for payoff at expiry, or \emph{false} for payoff at knockout. 
%(relevant for \emph{American} style FxOptions).

\item An \lstinline!ExerciseDates! node where exactly one ExerciseDate date element must be given. 

\item Premiums [Optional]: Option premium amounts paid by the option buyer to the option seller.

Allowable values:  See section \ref{ss:premiums}

\end{itemize}



\item Currency: The currency of the equity outperformance option.

Allowable values:  See Table \ref{tab:currency} \lstinline!Currency!.
    
\item Underlying1:  Specifies the first underlying equity.  This in turn defines the equity curve used for pricing. The \lstinline!Underlying! node is described in further detail in Section \ref{ss:underlying}. Note that the node name is \lstinline!Underlying1!.

\item Underlying2:  Specifies the second underlying equity. This in turn defines the equity curve used for pricing. The \lstinline!Underlying! node is described in further detail in Section \ref{ss:underlying}. Note that the node name is \lstinline!Underlying2!.

Also note that the equities in Underlying1 and Underlying2 must be quoted in the same currency.

\item InitialPrice1:  Specifies the initial price for first underlying equity.

Allowable values:  Any positive real number.	
    
\item InitialPrice2:  Specifies the initial price for second underlying equity.

Allowable values:  Any positive real number.	
    
\item StrikeReturn: The option strike return.
    
Allowable values:  Any positive real number.	
    
\item Notional: The notional amount for the trade.
    
Allowable values:  Any positive real number.

\item KnockInPrice[Optional]: The payoff is only paid if on the settlement date the price of underlying2 is above this value.
    
Allowable values:  Any positive real number.

\item KnockOutPrice[Optional]: The payoff is only paid if on the settlement date the price of underlying2 is below this value.
    
Allowable values:  Any positive real number.

\item InitialPriceCurrency1 [Optional]: Only relevant if InitialPrice1 is given in a currency other than Underlying1's currency.

Allowable values:  See Table \ref{tab:currency} \lstinline!Currency!.

\item InitialPriceCurrency2 [Optional]: Only relevant if InitialPrice1 is given in a currency other than Underlying2's currency.

Allowable values:  See Table \ref{tab:currency} \lstinline!Currency!.

\item InitialPriceFXTerms1 [Mandatory when InitialPriceCurrency1 is provided]: The node must be given if and only if the underlying currency is different from the initialPrice currency. The node contains the following sub nodes:
\begin{itemize}
  \item FXIndex: The fx index to use for the conversion, this must contain the underlying asset currency and the funding leg
    currency (in the order defined in table \ref{tab:fxindex_data}, i.e. it does not matter which one is the asset currency and which is the funding currency)
    
    Allowable values: see \ref{tab:fxindex_data}
	
\item InitialPriceFXTerms2 [Mandatory when InitialPriceCurrency2 is provided]: The node must be given if and only if the underlying currency is different from the initialPrice currency. Contains the same subnodes as InitialPriceFXTerms1.

    Allowable values: Any valid calendar, see Table \ref{tab:calendar} Calendar. Defaults to the \emph{NullCalendar} (no holidays) if left blank or omitted.
\end{itemize}
\end{itemize}
