\subsubsection{FX Digital Barrier Option}

\ifdefined\IncludePayoff{{\bf Payoff}

This product has a continuously monitored single barrier with an Cash-or-Nothing
digital underlying option.

At expiry a digital FX Barrier option pays either the payoff of a digital Cash-or-Nothing underlying option depending
upon an FX spot rate reaching a pre-set barrier level.
\begin{itemize}
\item Can be of the same four main types as a regular Barrier FX Option: 
Up-and-out, Down-and-out, Up-and-in, Down-and-in. 
\item A Knock-In or One-Touch option pays the underlying option payoff if the barrier is
breached, and no payout otherwise. 
\item A Knock-Out or No-Touch option has no payout if the barrier is breached, 
and pays the underlying option payoff otherwise.
\item The buyer of a Digital FX Barrier Option pays a premium to the seller.
\end{itemize}

{\bf Input}}

\else

An FX Digital Barrier Option pays a given cash amount in domestic currency at expiry, if the underlying fx rate has hit (or not hit) a continuously monitored barrier (as for the FxTouchOption) and the fx rate at the expiry date is above (call) or below (put) a given strike.

\fi

The \lstinline!FxDigitalBarrierOptionData!  node is the trade data container for the \emph{FxDigitalBarrierOption} trade type.   The
\lstinline!FxDigitalBarrierOptionData!  node includes one  \lstinline!OptionData! trade component sub-node and one \lstinline!BarrierData! trade component sub-node plus elements
specific to the FX Digital Barrier Option. 

The structure of an example \lstinline!FxDigitalBarrierOptionData! node for a FX Digital Barrier Option is shown in Listing
\ref{lst:fxdigitalbarrieroption_data}.

\begin{listing}[H]
%\hrule\medskip
\begin{minted}[fontsize=\footnotesize]{xml}
        <FxDigitalBarrierOptionData>
            <OptionData>
                <LongShort>Long</LongShort>
                <OptionType>Call</OptionType>
                <Style>European</Style>              
                <ExerciseDates>
                 <ExerciseDate>2021-12-14</ExerciseDate>
                </ExerciseDates> 
                ...
            </OptionData>
            <BarrierData>
                <Type>DownAndIn</Type>
                <Levels>
                    <Level>1.18</Level>
                </Levels>
            </BarrierData>
            <StartDate>2019-01-25</StartDate>
            <Calendar>TARGET</Calendar>
            <FXIndex>FX-ECB-EUR-USD</FXIndex>
            <Strike>1.1</Strike>
            <PayoffAmount>100000</PayoffAmount>
            <PayoffCurrency>USD</PayoffCurrency>            
            <ForeignCurrency>EUR</ForeignCurrency>
            <DomesticCurrency>USD</DomesticCurrency>
        </FxDigitalBarrierOptionData>
\end{minted}
\caption{FX Digital Barrier Option data}
\label{lst:fxdigitalbarrieroption_data}
\end{listing}

The meanings and allowable values of the elements in the \lstinline!FxDigitalBarrierOptionData!  node follow below.

\begin{itemize}

\item OptionData: This is a trade component sub-node outlined in section \ref{ss:option_data}. 
The relevant fields in the \lstinline!OptionData! node for an FxDigitalBarrierOption are:

\begin{itemize}
\item \lstinline!LongShort! The allowable values are \emph{Long} or \emph{Short}.

\item \lstinline!OptionType! The allowable values are \emph{Call} or \emph{Put}. Given knock-in or no knock-out,  \emph{Call} means that the digital payout will occur if the fx rate at the expiry date is above the given strike, and \emph{Put} means that the digital payout will occur if the fx rate at the expiry date is below the given strike.

\item  \lstinline!Style! The FX Digital Barrier Option type allows for \emph{European} option exercise style only.

%\item  \lstinline!Settlement! The allowable values are \emph{Cash} or \emph{Physical}.

\item An \lstinline!ExerciseDates! node where exactly one ExerciseDate date element must be given.

\item \lstinline!Premiums! [Optional]: Option premium amounts paid by the option buyer to the option seller.

Allowable values:  See section \ref{ss:premiums}

\end{itemize}

\item BarrierData: This is a trade component sub-node outlined in section \ref{ss:barrier_data}. 

Note that the \emph{FxDigitalBarrierOption} is a single barrier instrument, and can have only one BarrierData node with one barrier level.

Level specified in BarrierData should be quoted as the amount in DomesticCurrency per one unit of ForeignCurrency, with both currencies as defined in FxDigitalBarrierOptionData node.

Type specified in BarrierData can be one of: \emph{UpAndIn, DownAndIn, UpAndOut, DownAndOut}

\item StartDate[Optional]: The start date for checking if a barrier has been breached prior to today's date.  If omitted or left blank no check is made and it is assumed no barrier has been breached in the past. Has no impact if set to today's date or a date in the future. If`StartDate' is provided then the fixings for dates between this date and the asof date are checked to see if the option was triggered. If no fixing is available then we skip that date. This is to allow for backwards compatibility.

Allowable values:  See \lstinline!Date! in Table \ref{tab:allow_stand_data}.

\item Calendar[Optional]: The calendar associated with the FX Index. Required if StartDate is set to a date prior to today's date, otherwise optional.

Allowable values: See Table \ref{tab:calendar} Calendar.

\item FXIndex[Optional]: A reference to an FX Index source to check if the barrier has been breached. Required if StartDate is set to a date prior to today's date, otherwise optional, and can be omitted but not left blank.

Allowable values:  The format of the FX Index is``FX-SOURCE-CCY1-CCY2'' as described in table \ref{tab:fxindex_data}. 
\item \lstinline!FXIndexDailyLows! [Optional]: Refers to an FX Index that tracks the daily low quotes. This is used to check if the barrier was breached at any point during the day. If not provided, ORE will automatically derive the index name by appending the suffix \emph{\_LOW} to the FXIndex source (e.g. \emph{FX-SOURCE\_LOW-CCY1-CCY2}). If no fixings are available, the system will fall back to using the fixings from the FXIndex.

Allowable values:  The format of the FX Index is``FX-SOURCE-CCY1-CCY2'' as described in table \ref{tab:fxindex_data}.  

\item \lstinline!FXIndexDailyHighs! [Optional]: Refers to an FX Index that tracks the daily high quotes. This is used to check if the barrier was breached at any point during the day. If not provided, ORE will automatically derive the index name by appending the suffix \emph{\_HIGH} to the FXIndex source (e.g. \emph{FX-SOURCE\_HIGH-CCY1-CCY2}). If no fixings are available, the system will fall back to using the fixings from the FXIndex.

Allowable values:  The format of the FX Index is``FX-SOURCE-CCY1-CCY2'' as described in table \ref{tab:fxindex_data}. 
\item Strike: The FX strike price, expressed as the amount in DomesticCurrency per one unit of ForeignCurrency. 

Allowable values:  Any positive real number.

\item PayoffAmount: The fixed payoff amount expressed in the PayoffCurrency. It is cash-or-nothing payoff that depends on the option being in or out of the money, and whether the barrier has been breached.

Allowable values:  Any positive real number.

\item PayoffCurrency[Optional]: The payoff currency of the FX digital option is the currency of the payoff amount. Must be either the Domestic or Foreign currency for this trade, If omitted this defaults to DomesticCurrency as defined in FxDigitalBarrierOptionData node.

Allowable values:  See Table \ref{tab:currency} \lstinline!Currency!.

\item ForeignCurrency: The foreign currency of the FX digital barrier option is equivalent to the bought currency.  

Allowable values:  See Table \ref{tab:currency} \lstinline!Currency!.

\item DomesticCurrency: The domestic currency of the FX digital barrier option is equivalent to the sold currency. 

Allowable values:  See Table \ref{tab:currency} \lstinline!Currency!.

\end{itemize}

Note that FX Digital Barrier Options also cover Precious Metals, i.e. with
currencies XAU, XAG, XPT, XPD, and Cryptocurrencies,  see supported Cryptocurrencies in Table \ref{tab:currency}.
