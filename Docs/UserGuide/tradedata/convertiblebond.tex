\subsubsection{Convertible Bond}
\label{ss:convertible_bond}

\ifdefined\IncludePayoff{{\bf Payoff}

A convertible bond is a bond, that can be converted to a prespecified number of shares. The shares are usually from the
bond issuer, but it is also possible that the shares are from a different issuer (exchangeables). In addition, the share
currency can be different from the bond currency in both cases (cross-currency convertibles).

The bond might be callable by the issuer (typically in American style) and / or puttable by the investor (typcially in
Bermudan style). The issuer calls can be ``hard calls'', which are call rights in the traditional sense, as opposed to
``soft calls'' which can only the exercised if the equity price observed on (and possibly during a period before) the
exercise date is above a prespecified threshold. If a soft call is exercised, the investor has the right to convert the
bond into shares instead of accepting the payment from the issuer call (``forced conversion'').

For a detachable or stripped convertible bond the optionality can be traded separately from the bond. We set the NPV for
a detachable convertible bond to the difference of the convertible bond npv and the bond floor npv, where the bond floor
denotes the underyling vanilla bond stripped of any optionality.

Additional features of convertible bonds include dividend protection, contingent conversion, mandatory conversion,
conversion ratio resets, make-whole calls, dividend-forfeit clauses, copay clauses.

Refer to \cite{Spiegeleer_convertible_handbook} and \cite{Bloomberg_OVCV} for a deeper dive into the convertible bond
universe.

{\bf Input}}\fi

A convertible bond is set up in ORE using a {\tt ConvertibleBondData} block as shown in listing
\ref{lst:convertiblebonddata1}. The bond details are read from reference data in this case. 

A convertible bond is a bond, that can be converted into a prespecified number of shares, given by:
$$
NumberOfShares = \frac{BondNotional}{ConversionRatio}
$$

Where the Conversion Ratio is specified in the underlying bond reference data.

The shares are usually from the bond issuer, but it is also possible that the shares are
from a different issuer (exchangeables). In addition, the share currency can be different
from the bond currency in both cases (cross-currency convertibles).


The bond might be callable by the issuer (typically in American style) and / or puttable
by the investor (typically in Bermudan style). The issuer calls can be “hard calls”,
which are call rights in the traditional sense, as opposed to “soft calls” which can only
the exercised if the equity price observed on the exercise date is above a prespecified
threshold given by TriggerRatios. If a soft call is exercised, the investor has the right to convert the bond into
shares instead of accepting the payment from the issuer call (“forced conversion”).


The meanings and allowable
values of the elements in the {\tt ConvertibleBondData} block are as follows:

\begin{itemize}
  \item SecurityId: The underlying security identifier\\
      Allowable values:  Typically the ISIN of the underlying bond, with the ISIN: prefix.
  \item BondNotional: The notional of the underlying bond expressed in the currency of the bond.\\
      Allowable values:  Any positive real number.
    \item CreditRisk [Optional] Boolean flag indicating whether to show Credit Risk on the Bond product. \\
      Allowable Values: \emph{true} or \emph{false} Defaults to \emph{true} if left blank or omitted.          
\end{itemize}

\begin{listing}[H]
\begin{minted}[fontsize=\footnotesize]{xml}
  <Trade id="ConvertibleBond">
    <TradeType>ConvertibleBond</TradeType>
    <Envelope>...</Envelope>
    <ConvertibleBondData>
      <BondData>
        <SecurityId>ISIN:XS0451905367</SecurityId>
        <BondNotional>1000000.00</BondNotional>
      </BondData>
    </ConvertibleBondData>
  </Trade>
\end{minted}
\caption{Convertible bond set up using reference data}
\label{lst:convertiblebonddata1}
\end{listing}

Alternatively the bond can be set up with further explicit details using the blocks as shown in listing
\ref{lst:convertiblebonddata2}. All fields that are not given in the trade XML are filled up with the information from
the reference data if available in the reference data. In other words, if reference data is given, the trade xml can
still be used to overwrite the information partially, if this seems appropriate. The meanings and allowable values of
the elements in the block are as follows:

\begin{itemize}
  \item BondData: The vanilla part of the bond, see \ref{ss:bond}.
  \item CallData: The call terms of the bond, as described below. Optional, if not given, no calls are present.
  \item PutData: The put terms of the bond, as described below. Optional, if not given, no puts are present.
  \item ConversionData: The conversion terms of the bond, as described below. This node must always be given, even if no
    conversion rights are present (in which case an empty conversion date list can be used).
  \item DividendProtectionData: The dividend protection terms of the bond, as described below. Optional, if not given,
    no dividend prtection is present.
  \item Detachable: If true, the trade represents the embedded optionality, i.e. the difference between the full
    convertible bond and the bond floor. Optional, defaults to false. \\
    Allowable values: true, false
\end{itemize}

The convertible bond trade type supports perpetual schedules, i.e. perpetual convertible bonds can be represented by
omitting the EndDate in the following schedules to indicate perpetual schedules. Only rule based schedules can be used
to indicate perpetual schedules.

\begin{itemize}
\item BondData / LegData: Omitting the EndDate in this schedule indicates that the underlying bond runs perpetually.
\item CallData: Omitting the EndDate in this schedule indicates perpetual call dates. For American call dates, where
  only two dates have to be specified (start and end date of the american call window), a rule based schedule with Tenor
  = 0D, Rule = Zero and without EndDate can be used to indicate an end date infinitely far away in the future.
\item PutData: Same as CallData.
\item ConversionData: Omitting the EndDate in this schedule indicates perpetual conversion rights. For American rights,
  the same comment as under CallData applies.
\item ConversionData / ConversionResets: Omitting the EndDate in this schedule indicates perpetual conversion resets.
\item DividendProtectionData: Omitting the EndDate in this schedule indicates a perpetual dividend protection schedule.
\end{itemize}

\begin{listing}[H]
\begin{minted}[fontsize=\footnotesize]{xml}
  <Trade id="ConvertibleBond">
    <TradeType>ConvertibleBond</TradeType>
    <Envelope>...</Envelope>
    <ConvertibleBondData>
      <BondData> ... </BondData>
      <CallData> ... </CallData>
      <PutData> ... </PutData>
      <ConversionData> ... </ConversionData>
      <DividendProtectionData> ... </DividendProtectionData>
      <Detachable>false</Detachable>
    </ConvertibleBondData>
  </Trade>
\end{minted}
\caption{Convertible bond set up using the detail blocks}
\label{lst:convertiblebonddata2}
\end{listing}

\underline{Specification of CallData / PutData:}

All lists specified in subnodes (except the date list itself of course) can be specified as either an explicit list of
values corresponding to the schedule dates list or using the attribute \verb+startDate+. An explicit value list can be
shorter than the list of dates, in which case the last value from the list is associated to the remaining dates.

See listings
\ref{lst:convertiblebonddata_callputdata_1},\ref{lst:convertiblebonddata_callputdata_2},\ref{lst:convertiblebonddata_callputdata_3},\ref{lst:convertiblebonddata_callputdata_4},\ref{lst:convertiblebonddata_callputdata_5},\ref{lst:convertiblebonddata_callputdata_6},\ref{lst:convertiblebonddata_callputdata_7}
for examples of exercise schedules.

\begin{itemize}

\item Styles: A list of the exercise styles. Notice that Bermudan is used to define European exercises as well, namely
as a Bermudan exercise with a single exercise date. The attribute \verb+startDate+ can be used to specify the list. \\
Allowable values: American, Bermudan

\item ScheduleData: A schedule of exercise dates (for Bermudan exercises) or start / end dates (for American exercises) \\
  Allowable values: see \ref{ss:schedule_data}.

\item Prices: A list of exercise prices in relative terms, i.e. if the price is $1.02$ then the amount paid on the
  exercise is this price times the current notional of the bond (plus accrued interest, if the price type is clean, see
  below). The attribute \verb+startDate+ can be used to specify the list.\\
  Allowable values: Any positive real number.

\item PriceType: A list of the flavour in which the exercise prices are given. The attribute \verb+startDate+ can be
  used to specify the list.\\
  Allowable values: Clean, Dirty.

\item IncludeAccrual: A list of flags specifying whether accruals have to be paid on exercise. This is independent of
  the quoting style of the exercise prices (PriceType).\\
  Allowable values: true, false

\item Soft: A list of flags specifying whether the call is soft (true) or hard (false). The attribute \verb+startDate+
  can be used to specify the list. Optional, defaults to false. Only applicable to Calls, not to Puts. Optional, if not
  given, false is assumed, i.e. hard calls. If soft calls are specified, at least one conversion exercise date with
  corresponding conversion rate must be defined under ConversionData. \\
  Allowable values: true, false

\item TriggerRatios: A list of trigger ratios $T$ for soft calls. A soft call can be executed only if the equity price
  on the exercise date is above the Conversion Price (defined below) times the trigger ratio, i.e. $S_t > C^P_tT$. Only applicable to
  Calls, not to Puts. Required for soft calls, can be omitted otherwise.\\
  
$$
Conversion Price, C^P_t = \frac{1}{ConversionRatio}
$$
  
For cross-currency trades the conversion price is usually quoted in equity ccy, i.e.  
  
$$
Conversion Price, C^P_t = \frac{1}{ConversionRatio \cdot X_t}
$$  
  
where $X_t$ converts one equity ccy unit to bond ccy  
  
  Allowable values: Any positive real number.

\item NOfMTriggers: A list of n-of-m trigger specifications for calls, i.e. the soft-call trigger defined by
  TriggerRatios must be observed on n of the m calendar days in the period before (and including) a call date. Only applicable
  to Calls, not to Puts. Optional, defaults to ``1-of-1'' \\
  Allowable values: x-of-y with x, y non-negative integers, ``1-of-1'' corresponds to a vanilla call specification

\item MakeWhole: A list of make whole conditions. Optional. Possible subnodes are:
  \begin{itemize}
    \item ConversionRatioIncrease: In case of a call exercise, the conversion ratio (applicable in case of a forced
      conversion) is adjusted upwards. The adjustment is additive, i.e. if the current conversion ratio is $CR$ the
      conversion ratio applicable in case of a forced conversion will be $CR+d$ where $d$ is interpolated from a matrix
      of effective dates (rows) and stock prices (columns). The conversion rate adjustment might be capped by a
      prespecified rate. If the exercise date / stock price lies outside the matrix, $d$ is zero, i.e. no adjustment is
      made. Notice that a soft call trigger is checked w.r.t. $CR$, i.e. the unadjusted conversion ratio.
      \begin{itemize}
      \item Cap: An upper bound for the adjusted conversion ratio. Optional, if not given, no cap will be applied.\\
        Allowable values: Any non-negative real number.
      \item StockPrices: A comma separated list of stock prices defining the interpolation grid's x values. At least two
        stock prices must be given.\\
        Allowable values: A list of non-negative real numbers.
      \item CrIncreases: A node that contains at least two subnodes CrIncrease. Each subnode must have an attribute
        startDate defining the effective date of the adjustment and a list of conversion ratio adjustments $d$. The
        number of adjustments must match the number of prices given in the StockPrices node. \\
        Allowable values: A list of non-negative real numbers.
      \end{itemize}
  \end{itemize}

\end{itemize}

\begin{listing}[H]
\begin{minted}[fontsize=\footnotesize]{xml}
  <!-- Bermudan issuer call on three dates at a clean price of 100 (hard calls),
       accruals are paid on exercise -->
  <CallData>
    <Styles>
      <Style>Bermudan</Style>
    </Styles>
    <ScheduleData>
      <Dates>
        <Dates>
          <Date>2016-08-03</Date>
          <Date>2017-08-03</Date>
          <Date>2018-08-03</Date>
        </Dates>
      </Dates>
    </ScheduleData>
    <Prices>
      <Price>1.00</Price>
    </Prices>
    <PriceTypes>
      <PriceType>Clean</PriceType>
    </PriceTypes>
    <IncludeAccruals>
      <IncludeAccrual>true</IncludeAccrual>
    </IncludeAccruals>
    <Soft>
      <Soft>false</Soft>
    </Soft>
    <TriggerRatios/>
    <NOfMTriggers>
      <NOfMTrigger>20-of-30</NOfMTrigger>
    </NOfMTriggers>
  </CallData>
\end{minted}
\caption{Convertible bond call data example 1}
\label{lst:convertiblebonddata_callputdata_1}
\end{listing}

\begin{listing}[H]
\begin{minted}[fontsize=\footnotesize]{xml}
  <!-- Bermudan issuer call on three dates at a clean price of 101, 102 and 103,
       soft calls with trigger ratio of 0.8, 0.85, 0.9,
       accrual are _not_ paid on exercise -->
  <CallData>
    <Styles>
      <Style>Bermudan</Style>
    </Styles>
    <ScheduleData>
      <Dates>
        <Dates>
          <Date>2016-08-03</Date>
          <Date>2017-08-03</Date>
          <Date>2018-08-03</Date>
        </Dates>
      </Dates>
    </ScheduleData>
    <Prices>
      <Price>1.01</Price>
      <Price>1.02</Price>
      <Price>1.03</Price>
    </Prices>
    <PriceTypes>
      <PriceType>Clean</PriceType>
    </PriceTypes>
    <IncludeAccruals>
      <IncludeAccrual>false</IncludeAccrual>
    </IncludeAccruals>
    <Soft>
      <Soft>true</Soft>
    </Soft>
    <TriggerRatios>
      <TriggerRatio>0.8</TriggerRatio>
      <TriggerRatio>0.85</TriggerRatio>
      <TriggerRatio>0.9</TriggerRatio>
    </TriggerRatios>
  </CallData>
\end{minted}
\caption{Convertible bond call data example 2}
\label{lst:convertiblebonddata_callputdata_2}
\end{listing}

\begin{listing}[H]
\begin{minted}[fontsize=\footnotesize]{xml}
  <!-- American issuer call between 2016-08-03 and 2018-08-03
       at a clean price of 100 (hard calls) -->
  <CallData>
    <Styles>
      <Style>American</Style>
    </Styles>
    <ScheduleData>
      <Dates>
        <Dates>
          <Date>2016-08-03</Date>
          <Date>2018-08-03</Date>
        </Dates>
      </Dates>
    </ScheduleData>
    <Prices>
      <Price>1.00</Price>
    </Prices>
    <PriceTypes>
      <PriceType>Clean</PriceType>
    </PriceTypes>
    <IncludeAccruals>
      <IncludeAccrual>true</IncludeAccrual>
    </IncludeAccruals>
    <Soft>
      <Soft>false</Soft>
    </Soft>
    <TriggerRatios/>
  </CallData>
\end{minted}
\caption{Convertible bond call data example 3}
\label{lst:convertiblebonddata_callputdata_3}
\end{listing}

\begin{listing}[H]
\begin{minted}[fontsize=\footnotesize]{xml}
  <!-- American issuer call between 2016-08-03 and 2020-08-03 (excl),
       hard calls at 100 between 2016-08-03 and 2018-08-03 (excl),
       soft calls at 102 between 2018-08-03 and 2019-08-03 (excl),
       soft calls at 103 between 2019-08-03 and 2020-08-03 -->
  <CallData>
    <Styles>
      <Style>American</Style>
    </Styles>
    <ScheduleData>
      <Dates>
        <Dates>
          <Date>2016-08-03</Date>
          <Date>2018-08-03</Date>
          <Date>2019-08-03</Date>
          <Date>2020-08-03</Date>
        </Dates>
      </Dates>
    </ScheduleData>
    <Prices>
      <Price>1.00</Price>
      <Price startDate="2018-08-03">1.02</Price>
      <Price startDate="2019-08-03">1.03</Price>
    </Prices>
    <PriceTypes>
      <PriceType>Clean</PriceType>
    </PriceTypes>
    <IncludeAccruals>
      <IncludeAccrual>true</IncludeAccrual>
    </IncludeAccruals>
    <Soft>
      <Soft>false</Soft>
      <Soft startDate="2018-03-03">true</Soft>
    </Soft>
    <TriggerRatios>
      <TriggerRatio>0.8</TriggerRatio>
      <TriggerRatio startDate="2019-08-03">0.9</TriggerRatio>
    </TriggerRatios>
  </CallData>
\end{minted}
\caption{Convertible bond call data example 4}
\label{lst:convertiblebonddata_callputdata_4}
\end{listing}

\begin{listing}[H]
\begin{minted}[fontsize=\footnotesize]{xml}
  <!-- Bermudan (hard) calls at 100 at 3 dates from 2016 to 2018,
       followed by American (soft) calls at 102 between 2018 and 2020 -->
  <CallData>
    <Styles>
      <Style>Bermudan</Style>
      <Style startDate="2018-08-03">American</Style>
    </Styles>
    <ScheduleData>
      <Dates>
        <Dates>
          <Date>2016-08-03</Date>
          <Date>2017-08-03</Date>
          <Date>2018-08-03</Date>
          <Date>2020-08-03</Date>
        </Dates>
      </Dates>
    </ScheduleData>
    <Prices>
      <Price>1.00</Price>
      <Price startDate="2018-08-03">1.02</Price>
    </Prices>
    <PriceTypes>
      <PriceType>Clean</PriceType>
    </PriceTypes>
    <IncludeAccruals>
      <IncludeAccrual>true</IncludeAccrual>
    </IncludeAccruals>
    <Soft>
      <Soft>false</Soft>
      <Soft startDate="2018-08-03">true</Soft>
    </Soft>
    <TriggerRatios>
      <TriggerRatio>0.8</TriggerRatio>
    </TriggerRatios>
  </CallData>
\end{minted}
\caption{Convertible bond call data example 5}
\label{lst:convertiblebonddata_callputdata_5}
\end{listing}

\begin{listing}[H]
\begin{minted}[fontsize=\footnotesize]{xml}
  <!-- Bermudan puts calls at 100, 101, 102 at 3 dates from 2016 to 2018 -->
  <PutData>
    <Styles>
      <Style>Bermudan</Style>
    </Styles>
    <ScheduleData>
      <Dates>
        <Dates>
          <Date>2016-08-03</Date>
          <Date>2017-08-03</Date>
          <Date>2018-08-03</Date>
        </Dates>
      </Dates>
    </ScheduleData>
    <Prices>
      <Price>1.00</Price>
      <Price>1.01</Price>
      <Price>1.02</Price>
    </Prices>
    <PriceTypes>
      <PriceType>Clean</PriceType>
    </PriceTypes>
    <IncludeAccruals>
      <IncludeAccrual>true</IncludeAccrual>
    </IncludeAccruals>
  </PutData>
\end{minted}
\caption{Convertible bond put data example 6}
\label{lst:convertiblebonddata_callputdata_6}
\end{listing}

\begin{listing}[H]
\begin{minted}[fontsize=\footnotesize]{xml}
<CallData>
...
   <MakeWhole>
     <ConversionRatioIncrease>
       <Cap>0.0740740</Cap>
       <StockPrices>13.50,15.00,16.20,18.00</StockPrices>
       <CrIncreases>
         <CrIncrease startDate="2020-06-25">0.0123456,0.0107487,0.0097173,0.0084567</CrIncrease>
         <CrIncrease startDate="2021-07-01">0.0123456,0.0096880,0.0086963,0.0075294</CrIncrease>
         <CrIncrease startDate="2022-07-01">0.0123456,0.0083927,0.0074222,0.0063383</CrIncrease>
         <CrIncrease startDate="2023-07-01">0.0123456,0.0069360,0.0058790,0.0048322</CrIncrease>
         <CrIncrease startDate="2024-07-01">0.0123456,0.0054453,0.0040025,0.0028833</CrIncrease>
         <CrIncrease startDate="2025-07-01">0.0123456,0.0049380,0.0000000,0.0000000</CrIncrease>
       </CrIncreases>
     </ConversionRatioIncrease>
   </MakeWhole>
</CallData>
\end{minted}
\caption{Convertible bond make whole data (conversion ratio increase)}
\label{lst:convertiblebonddata_callputdata_7}
\end{listing}

\underline{Specification of ConversionData:}

As in the case of the CallData, all lists can be specified as either an explicit list of values corresponding to the
schedule dates list or using the attribute \verb+startDate+. The ConversionRatios element is an expcetion, the given
start dates are interpreted independently of these schedule dates.

See listings \ref{lst:convertiblebonddata_conversion_1},
\ref{lst:convertiblebonddata_conversion_2},\ref{lst:convertiblebonddata_conversion_3},\ref{lst:convertiblebonddata_conversion_4},
\ref{lst:convertiblebonddata_conversion_5},\ref{lst:convertiblebonddata_conversion_6}
for examples of conversion schedules.

\begin{itemize}

\item Styles: The styles of the conversion rights. Notice that Bermudan is used to define European conversion rights as
  well, namely as a Bermudan conversion right with a single date. The attribute \verb+startDate+ can be used to
  specify the list. Can be omitted, if no conversion dates are given.\\
  Allwoable values: American, Bermudan

\item ScheduleData: The dates defining when the bond is convertible. For Bermudan exercises, the conversion can be
  executed on the single dates given in the list. For American exercises, the conversion can be executed between a given
  start and end date. Can be omitted, if no conversion rights are present.\\
  Allowable values: see \ref{ss:schedule_data}.

\item ConversionRatios: A list of conversion ratios $C^R$. The attribute \verb+startDate+ can be used to specify a date
  from which the ratio is valid. Notice that this date is always interpreted ``as is'', i.e. it is not mapped onto the
  next date in the defined schedule. If no startDate is given for a ratio, this ratio is interpreted as the initial
  ratio. \\
  Allowable values: Any non-negative real number.

\item FixedConversionAmounts: If this node is given, the conversion is specified to be conversion to fixed cash amounts
  instead of equity. If the cash amount currency is different from the bond currency, the FXIndex node must be
  given. See \ref{lst:convertiblebonddata_conversion_6} for an example. As for ConversionRatios the attribute
  \verb+startDate+ can be used to specify a date from which the amount is valid and this date is interpreted ``as is'',
  i.e. not mapped onto the next date in the defined schedule. The nodes
\begin{itemize}
  \item ConversionRatios
  \item ContingentConversion
  \item MandatoryConversion
  \item ConversionResets
  \item Underlying
  \item Exchangeable
\end{itemize}
must {\em not} be given, if this node is present. Furthermore, the following nodes from other sections are not
applicable if the conversion is specified to be fixed cash amounts, and must therefore not be given:
\begin{itemize}
\item CallData/Soft
\item CallData/TriggerRatios
\item CallData/NoMTriggers
\item CallData/MakeWhole
\item DividendProtectionData (including all subnodes)
\end{itemize}

\item ContingentConversion: This adds a condition $C^R_t S_t > B$ on the convertibility for the periods defined by the
  conversion dates. Optional.
  \begin{itemize}
  \item Observations: A list of observation modes. \\
    Allowable values: Spot (trigger is checked on the conversion date), StartOfPeriod (trigger is checked on the start
    of the conversion period defined by the dates list, for American style conversion only)
  \item Barriers: A list of barriers $B$ associated to the conversion dates. \\
    Allowable values: Positive real number or zero (conversion is not made contingent for this date).
  \end{itemize}

\item MandatoryConversion: This adds a mandatory conversion obligation at a date greater than all other conversion dates
  (if any). Optional.
  \begin{itemize}
  \item Date: The mandatory conversion date.\\
    Allowable values: Any date not earlier than the last otherwise specified conversion date.
  \item Type: The type of the mandatory conversion.\\
    Allowable values: PEPS
  \item PepsData: Details of mandatory conversion type PEPS.
    \begin{itemize}
      \item UpperBarrier: upper barrier for PEPS payoff.\\
        Allowable values: A real number.
      \item LowerBarrier: lower barrier for PEPS payoff.\\
        Allowable values:  A real number.
      \item UpperConversionRatio: conversion ratio for upper barrier in PEPS payoff.\\
        Allowable values: A real number.
      \item LowerConversionRatio: conversion ratio for lower barrier in PEPS payoff.\\
        Allowable values:  A real number.
    \end{itemize}
  \end{itemize}

\item ConversionResets: This adds a reset schedule for the conversion rate. If a reset feature is defined, only an
  initial ConversionRatio can be defined, the future conversion ratios are determined by the resets. The startDate
  attribute can be used to define references, thresholds, gearings, floors, global floors. Optional.
  \begin{itemize}
  \item ScheduleData: The conversion reset dates. \\
      Allowable values: see \ref{ss:schedule_data}.
  \item References: Whether the initial conversion price $C^P_0$ or the current conversion price $C^P_t$ is the reference for the reset.\\
    Allowable values: InitialConversionPrice, CurrentConversionPrice
  \item Thresholds: The threshold $T$ that triggers a reset ($S_t < TC^P_0$ or $S_t < TC^P_t$, depending on Reference)\\
    Allowable values: positive number or zero (disables the reset on this date effectively)
  \item Gearings: The gearings $g$ for the conversion rate adjustment. Option, defaults to $0$ (= no gearing applicable)\\
    Allowable values: positive number or zero (no gearing applicable on this date).
  \item Floors: The floors $f$ for the conversion rate adjustment. Optional, defaults to $0$ (= no floor applicable)\\
    Allowable values: positive number or zero (no floor applicable on this date)
  \item GlobalFloors: The global floors for the conversion rate adjustment. Option, defaults to $0$ (= no global floor applicable)\\
    Allowable values: positive number or zero (no global floor applicable on this date)
  \end{itemize}

\item Underlying: The equity underlying. \\
  Allwoable values: See \ref{ss:underlying}, the underlying type must be equity.

\item FXIndex: If equity ccy is different from bond ccy, an fx index for the two involved ccy is required. \\
  Allowable values:  The format of the FX Index is``FX-SOURCE-CCY1-CCY2'' as described in table \ref{tab:fxindex_data}.

\item Exchangeable: Node with data for exchangeables. Option, if omitted, the structure is considered non-exchangeable. Subnodes are:\\
  \begin{itemize}
  \item IsExchangeable: indicates whether the convertible bond is exchangeable\\
    Allowable values: true, false
  \item EquityCreditCurve: the credit curve modeling the equity issuer default, required if IsExchangeable is
    true. \\
    Allowable values: A valid credit curve identifier, e.g the ISIN of a reference bond with the ISIN: prefix:
    \verb+ISIN:XXNNNNNNNNNN+
  \item Secured: Indicates whether the convertible is secured with pledged shares or not. Optional, defaults to false.\\
    Allowable values: true, false.
  \end{itemize}
\end{itemize}

\begin{listing}[H]
\begin{minted}[fontsize=\footnotesize]{xml}
  <!-- Three conversion dates (Bermudan), conversion ratio is 0.5 -->
    <ConversionData>
      <Styles>
        <Style>Bermudan</Style>
      </Styles>
      <ScheduleData>
        <Dates>
          <Dates>
            <Date>2016-08-03</Date>
            <Date>2017-08-03</Date>
            <Date>2018-08-03</Date>
          </Dates>
        </Dates>
      </ScheduleData>
      <ConversionRatios>
        <ConversionRatio>0.05</ConversionRatio>
      </ConversionRatios>
      <Underlying>
        <Type>Equity</Type>
        <Name>RIC:.ABCD</Name>
      </Underlying>
      <FXIndex>FX-ECB-EUR-USD</FXIndex>
      <Exchangeable>
        <IsExchangeable>true</IsExchangeable>
        <EquityCreditCurve>ISIN:XS0982710740</EquityCreditCurve>
        <Secured>true</Secured>
      </Exchangeable>
    </ConversionData>
\end{minted}
\caption{Convertible bond conversion example 1}
\label{lst:convertiblebonddata_conversion_1}
\end{listing}

\begin{listing}[H]
\begin{minted}[fontsize=\footnotesize]{xml}
  <!-- American conversion between 2016-08-03 and 2020-08-03, with
       conversion ratio 0.5 for 2016-08-03 through 2018-08-03 (excl) and
       conversion ratio 0.6 for 2018-08-03 through 2020-08-03 -->
    <ConversionData>
      <Styles>
        <Style>American</Style>
      </Styles>
      <ScheduleData>
        <Dates>
          <Dates>
            <Date>2016-08-03</Date>
            <Date>2018-08-03</Date>
            <Date>2020-08-03</Date>
          </Dates>
        </Dates>
      </ScheduleData>
      <ConversionRatios>
        <ConversionRatio>0.05</ConversionRatio>
        <ConversionRatio startDate="2018-08-03">0.06</ConversionRatio>
      </ConversionRatios>
      <Underlying>
        <Type>Equity</Type>
        <Name>RIC:.ABCD</Name>
      </Underlying>
    </ConversionData>
\end{minted}
\caption{Convertible bond conversion example 2}
\label{lst:convertiblebonddata_conversion_2}
\end{listing}

\begin{listing}[H]
\begin{minted}[fontsize=\footnotesize]{xml}
  <!-- American conversion between 2016-08-03 and 2018-08-03, with conversion
       ratio 0.5, the conversion is contingent on the parity being above 1.3
       on 2016-08-03 for the conversion between 2016-08-03 and 2017-08-03 (excl)
       on 2017-08-03 for the conversion between 2017-08-03 and 2018-08-03 -->
    <ConversionData>
      <Styles>
        <Style>American</Style>
      </Styles>
      <ScheduleData>
        <Dates>
          <Dates>
            <Date>2016-08-03</Date>
            <Date>2017-08-03</Date>
            <Date>2018-08-03</Date>
          </Dates>
        </Dates>
      </ScheduleData>
      <ConversionRatios>
        <ConversionRatio>0.05</ConversionRatio>
      </ConversionRatios>
      <ContingentConversion>
        <Observations>
          <Observation>StartOfPeriod</Observation>
        </Observations>
        <Barriers>
          <Barrier>1.3</Barrier>
        </Barriers>
      </ContingentConversion>
      <Underlying>
        <Type>Equity</Type>
        <Name>RIC:.ABCD</Name>
      </Underlying>
    </ConversionData>
\end{minted}
\caption{Convertible bond conversion example 3}
\label{lst:convertiblebonddata_conversion_3}
\end{listing}

\begin{listing}[H]
\begin{minted}[fontsize=\footnotesize]{xml}
  <!-- American converion between 2016-08-03 and 2018-08-03 with CR 0.5.
       Mandatory conversion on 2020-08-03:
       LowerConversionRatio applies if stock price < LowerBarrier,
       UpperConversionRatio applies if stock price > UpperBarrier -->
    <ConversionData>
      <Styles>
        <Style>American</Style>
      </Styles>
      <ScheduleData>
        <Dates>
          <Dates>
            <Date>2016-08-03</Date>
            <Date>2018-08-03</Date>
          </Dates>
        </Dates>
      </ScheduleData>
      <ConversionRatios>
        <ConversionRatio>0.05</ConversionRatio>
      </ConversionRatios>
      <MandatoryConversion>
        <Date>2020-08-03</Date>
        <Type>PEPS</Type>
        <PepsData>
          <UpperBarrier>32.5</UpperBarrier>
          <LowerBarrier>20.5</LowerBarrier>
          <UpperConversionRatio>0.08</UpperConversionRatio>
          <LowerConversionRatio>0.03</LowerConversionRatio>
        </PepsData>
      </MandatoryConversion>
      <Underlying>
        <Type>Equity</Type>
        <Name>RIC:.ABCD</Name>
      </Underlying>
    </ConversionData>
\end{minted}
\caption{Convertible bond conversion example 4}
\label{lst:convertiblebonddata_conversion_4}
\end{listing}

\begin{listing}[H]
\begin{minted}[fontsize=\footnotesize]{xml}
  <!-- American conversion between 2016-08-03 and 2018-08-03 with CR 0.5.
       The conversion ratio is reset on 2016-11-03, 2017-02-03, 2018-05-03
       using T = 0.9, g = 0.8, f = 0.6, F = 0.6. -->
    <ConversionData>
      <Styles>
        <Style>American</Style>
      </Styles>
      <ScheduleData>
        <Dates>
          <Dates>
            <Date>2016-08-03</Date>
            <Date>2018-08-03</Date>
          </Dates>
        </Dates>
      </ScheduleData>
      <ConversionRatios>
        <ConversionRatio>0.05</ConversionRatio>
      </ConversionRatios>
      <ConversionResets>
        <ScheduleData>
          <Dates>
            <Dates>
              <Date>2016-11-03</Date>
              <Date>2017-02-03</Date>
              <Date>2018-05-03</Date>
            </Dates>
          </Dates>
        </ScheduleData>
        <References>
          <Reference>InitialConversionPrice</Reference>
        </References>
        <Thresholds>
          <Threshold>0.9</Threshold>
        </Thresholds>
        <Gearings>
          <Gearing>0.8</Gearing>
        </Gearings>
        <Floors>
          <Floor>0.7</Floor>
        </Floors>
        <GlobalFloors>
          <GlobalFloor>15</GlobalFloor>
        </GlobalFloors>
      </ConversionResets>
      <Underlying>
        <Type>Equity</Type>
        <Name>RIC:.ABCD</Name>
      </Underlying>
    </ConversionData>
\end{minted}
\caption{Convertible bond conversion example 5}
\label{lst:convertiblebonddata_conversion_5}
\end{listing}

\begin{listing}[H]
\begin{minted}[fontsize=\footnotesize]{xml}
  <!-- American conversion between 2024-08-24 and 2027-05-13, with
       conversion to 0.87 GBP cash for 2024-08-24 through 2024-11-23 (excl) and
       conversion to 0.75 GBP cash for 2024-11-23 through 2027-05-13 -->
    <ConversionData>
      <Styles>
        <Style>American</Style>
      </Styles>
      <ScheduleData>
        <Dates>
          <Dates>
            <Date>2024-08-24</Date>
            <Date>2024-11-23</Date>
            <Date>2027-05-13</Date>
          </Dates>
        </Dates>
      </ScheduleData>
      <FixedAmountConversion>
        <Currency>GBP</Currency>
        <Amounts>
          <Amount>0.87</Amount>
          <Amount startDate="2024-11-24">0.75</Amount>
        </Amounts>
      </FixedAmountConversion>
    </ConversionData>
\end{minted}
\caption{Convertible bond conversion example 6}
\label{lst:convertiblebonddata_conversion_6}
\end{listing}

\underline{Specification of DividendProtectionData:}

As for the CallData, all lists can be specified as either an explicit list of values corresponding to the schedule dates
list or using the attribute \verb+startDate+.

See listings \ref{lst:convertiblebonddata_divprot_1}, \ref{lst:convertiblebonddata_divprot_2}
for examples of dividend protection schedules.

\begin{itemize}
\item ScheduleData: The dates of the dividend protection schedule. The first date marks the date when the dividend
  protection becomes effective, i.e. dividend payments from this date on are taken into account in conversion ratio
  adjustments or passthroughs. The second date is then the first date on which the accumulated dividends between the
  first and second date trigger a conversion ratio reset or passthrough, and similar for all subsequent dates. The last
  given date is the last date with a conversion ratio reset or passthrough. \\ Allowable values: see
  \ref{ss:schedule_data}.
\item AdjustmentStyles: Whether the dividend excessing the threshold is passed through or the conversion ratio is
  adjusted. In both cases, the adjustment can be upwards only or up and down.\\
  Allwoable values: CrUpOnly, CrUpDown, CrUpOnly2, CrUpDown2, PassThroughUpOnly, PassThroughUpDown
\item DividendTypes: Whether the conversion ratio adjustment is calculated in terms of absolute or relative
  dividends. Does not have an effect for pass through dividends (should be set to Aboslute in this case).\\
  Allwoable values: Absolute, Relative
\item Thresholds: The threshold $H$. Notice that the threshold applies to each single period of the dividend protection
  schedule. If the threshold is e.g. provided on an annual basis in the terms of the convertible bond, but the dividend
  protection schedule is quarterly, then the threshold in the trade xml should be the annual threshold divided by
  $4$.\\
  Allwoable values: Any non-negativee number.
\end{itemize}

\begin{listing}[H]
\begin{minted}[fontsize=\footnotesize]{xml}
  <!-- Divdend protection based on aboslute dividend amounts via adjustment
       of the conversion rate, up-only adjustment. -->
    <DividendProtectionData>
      <ScheduleData>
        <Dates>
          <Dates>
            <Date>2016-08-03</Date>
            <Date>2017-08-03</Date>
            <Date>2018-08-03</Date>
            <Date>2019-08-03</Date>
          </Dates>
        </Dates>
      </ScheduleData>
      <AdjustmentStyles>
        <AdjustmentStyle>CrUpOnly</AdjustmentStyle>
      </AdjustmentStyles>
      <DividendTypes>
        <DividendType>Absolute</DividendType>
      </DividendTypes>
      <Thresholds>
        <Threshold>1.2</Threshold>
      </Thresholds>
    </DividendProtectionData>
\end{minted}
\caption{Convertible bond dividend protection example 1}
\label{lst:convertiblebonddata_divprot_1}
\end{listing}

\begin{listing}[H]
\begin{minted}[fontsize=\footnotesize]{xml}
  <!-- Dividend protection based on relative dividend amounts via adjustment
       of the conversion rate, up-only adjustment. -->
    <DividendProtectionData>
      <ScheduleData>
        <Dates>
          <Dates>
            <Date>2016-08-03</Date>
            <Date>2017-08-03</Date>
            <Date>2018-08-03</Date>
            <Date>2019-08-03</Date>
          </Dates>
        </Dates>
      </ScheduleData>
      <AdjustmentStyles>
        <AdjustmentStyle>CrUpOnly</AdjustmentStyle>
      </AdjustmentStyles>
      <DividendTypes>
        <DividendType>Relative</DividendType>
      </DividendTypes>
      <Thresholds>
        <Threshold>0.01</Threshold>
      </Thresholds>
    </DividendProtectionData>
\end{minted}
\caption{Convertible bond dividend protection example 2}
\label{lst:convertiblebonddata_divprot_2}
\end{listing}
