%========================================================
\section{Getting and Building ORE}\label{sec:installation}
%========================================================

You can get ORE in two ways, either by downloading a release bundle as described in section \ref{sec:release} (easiest if you just want to use ORE) or by
checking out the source code from the github repository as described in section \ref{sec:build_ore} (easiest if you want to build and develop ORE).

\subsection{ORE Releases}\label{sec:release}

ORE releases are regularly provided in the form of source code archives, Windows exe\-cutables {\tt ore.exe}, example
cases and documentation. Release archives will be provided at \url{https://github.com/opensourcerisk/engine/releases}.

The release contains the QuantLib source version that ORE depends on. This is the latest QuantLib release that precedes the ORE release including a small number of patches.

\medskip
The release consists of a single archive in zip format
\begin{itemize}
\item {\tt ORE-<VERSION>.zip}
\end{itemize}

When unpacked, it creates a directory {\tt ORE-<VERSION>} with the following files respectively subdirectories
\begin{enumerate}
%\item {\tt bin/win32/ore.exe}
%\item {\tt bin/x64/ore.exe}
\item {\tt App/}
\item {\tt Docs/}
\item {\tt Examples/}
\item {\tt FrontEnd/}
\item {\tt OREAnalytics/}
\item {\tt OREData/}
\item {\tt ORETest/}
\item {\tt QuantExt/}
\item {\tt QuantLib/}
\item {\tt ThirdPartyLibs/}
\item {\tt tools/}
\item {\tt xsd/}
\item {\tt userguide.pdf}
\end{enumerate} 

The first three items and {\tt userguide.pdf} are sufficient to run the compiled ORE application
on the list of examples described in the user guide (this works on Windows only). The Windows executables are located in {\tt App/bin/Win32/Release/} respectively {\tt App/bin/x64/Release/}. To continue with the compiled
executables:
\begin{itemize}
\item Ensure that the scripting language Python is installed on your computer, see also section \ref{sec:python}
  below;
\item Move on to the examples in section \ref{sec:examples}.
\end{itemize}

\medskip
The release bundle contains the ORE source code, which is sufficient to build ORE from sources as follows (if you build ORE for development purposes, we recommend using git though, see section \ref{sec:build_ore}):
\begin{itemize}
\item Set up Boost as described in section \ref{sec:boost}, unless already installed
\item Build QuantLib, QuantExt, OREData, OREAnalytics, App (in this order) as described in section \ref{sec:build}
\item Note that ThirdPartyLibs does not need to be built, it contains RapidXml, header only code for reading and
  writing XML files
\item Move on to section \ref{sec:python} and the examples in section \ref{sec:examples}.
\end{itemize}

Open {\tt Docs/html/index.html} to see the API documentation for QuantExt, OREData and OREAnalytics, generated by
doxygen.

\subsection{Building ORE}\label{sec:build_ore}

ORE's source code is hosted at \url{https://github.com/opensourcerisk/engine}.

\subsubsection{Git}

To access the current code base on GitHub, one needs to get {\tt git} installed first.
   
\begin{enumerate}
\item Install and setup Git on your machine following instructions at \cite{git-download}

\item Fetch ORE from github by running the following: 

{\tt\% git clone https://github.com/opensourcerisk/engine.git ore}      

This will create a folder 'ore' in your current directory that contains the codebase.

\item Initially, the QuantLib subdirectory under {\tt ore} is empty as it is a submodule pointing to the official
  QuantLib repository. To pull down locally, use the following commands:

{\tt
\% cd ore \\
\% git submodule init \\
\% git submodule update
}

\end{enumerate}

Note that one can also run 

{\footnotesize \tt\% git clone --recurse-submodules https://github.com/opensourcerisk/engine.git ore}

in step 2, which also performs the steps in 3.

\subsubsection{Boost}\label{sec:boost}

QuantLib and ORE depend on the boost C++ libraries. Hence these need to be installed before building QuantLib and
ORE. On all platforms the minimum required boost version is 1\_78.
%Other versions may work on some platforms and system configurations, but were not tested.

\subsubsection*{Windows}

\begin{enumerate}
\item Download the pre-compiled binaries for your MSVC version (e.g. MSVC-14.3 for MSVC2022) from \cite{boost-binaries}
%, any recent version should work
\begin{itemize}
\item 32-bit: \cite{boost-binaries}{\bs}VERSION{\bs}boost\_VERSION-msvc-14.3-32.exe{\bs}download 
\item 64-bit: \cite{boost-binaries}{\bs}VERSION{\bs}boost\_VERSION-msvc-14.3-64.exe{\bs}download
\end{itemize}
\item Start the installation file and choose an installation folder (the ``boost root directory''). Take a note of that folder as it will be needed later on.   
\item Finish the installation by clicking Next a couple of times.
\end{enumerate}
    
Alternatively, compile all Boost libraries directly from the source code:

\begin{enumerate}
\item Open a Visual Studio Tools Command Prompt
\begin{itemize}
\item 32-bit: VS2022 x86 Native Tools Command Prompt
\item 64-bit: VS2022 x64 Native Tools Command Prompt
\end{itemize}
\item Navigate to the boost root directory
\item Run bootstrap.bat
\item Build the libraries from the source code
\begin{itemize}
\item 32-bit: \\
  {\footnotesize\tt .{\bs}b2 --stagedir=.{\bs}lib{\bs}Win32{\bs}lib --build-type=complete toolset=msvc-14.3 \bs \\
    address-model=32 --with-test --with-system --with-filesystem  \bs \\
    --with-serialization --with-regex --with-date\_time stage}
\item 64-bit: \\
  {\footnotesize\tt .{\bs}b2 --stagedir=.{\bs}lib{\bs}x64{\bs}lib --build-type=complete toolset=msvc-14.3 \bs \\
    address-model=64 --with-test --with-system --with-filesystem \bs \\
    --with-serialization --with-regex --with-date\_time stage}
\end{itemize}
\end{enumerate}

\subsubsection*{Unix}

\begin{enumerate}
\item Download Boost from \cite{boost} and build following the instructions on the site
\item Define the environment variable BOOST that points to the boost directory
(so includes should be in BOOST and libs should be in BOOST/stage/lib)
\end{enumerate}

\subsubsection{ORE Libraries and Application}\label{sec:build}

\subsubsection*{Windows}

\begin{enumerate}

\item Download and install Visual Studio Community Edition (Version 2019 or later, 2022 is recommended). 
During the installation, make sure you install the Visual
C++ support under the Programming Languages features (disabled by default).

\item Configure boost paths: \\

Set environment variables, e.g.:
\begin{itemize}
  	\item  {\tt \%BOOST\%} pointing to your directory, e.g, {\tt C:{\bs}boost\_1\_72\_0} 
  	\item {\tt \%BOOST\_LIB32\%} pointing to your Win32 lib directory, e.g, {\tt C:{\bs}boost\_1\_72\_0{\bs}lib32\-msvc\-14.3} 
	\item  {\tt \%BOOST\_LIB64\%} pointing to your x64 lib directory, e.g, {\tt C:{\bs}boost\_1\_72\_0{\bs}lib64\-msvc\-14.3} 
 \end{itemize}
 
\item Download and install CMake for Windows (https://cmake.org/download/). Visual Studio Community Edition 2019 or later supports CMake and you can install the feature 'C++ CMake Tools for Windows' instead of installing CMake as standalone program.

\end{enumerate}

%\subsubsection*{Visual Studio with CMake}

Visual Studio 2019 and later supports CMake Projects.

\begin{enumerate}
\item Start Visual Studio 2019 or later.
\item Select "Open a local folder" from the start page or menu.
\item In the dialog window, select the ORE root directory.
\item Visual Studio will read the cmake presets from CMakePresets.json and the project file CMakeList.txt and configure the project.
\item Once the configuration is finished and one can build the project.
\item The executables are built in the subfolder {\tt /build/TARGET/CONFIGURATION/EXECUTABLE}, e.g. {\tt /build/App/Release/ore.exe}.
\end{enumerate}

ORE is shipped with configuration and build presets using Visual Studio 2022 and the Ninja build system. Those presets are configured in the CMakePreset.json which is read by Visual Studio by default when opening the CMake project. If you want to use Visual Studio 2019 instead, you would have to change the Generator in the CMakePreset.json from "Visual Studio 17 2022" to "Visual Studio 16 2019".

You can switch in the solution explorer from the file view to the projects view, where the CMake Targets View can be selected. In this view, the various target projects can be seen below "ORE Project" and be used in a similar manner as the usual VS projects.

%\subsubsection*{Generate Visual Studio Projects with CMake}
\medskip

Alternatively, Visual Studio project files can be auto-generated from the CMake project files or ORE can be built with the CMake command line tool, similar to UNIX / Mac systems.

\begin{enumerate}

\item Generate MSVC project files from CMake files:
\begin{itemize}
\item Open a Visual Studio Tools Command Prompt
\begin{itemize}
\item 32-bit: VS2022/x86 Native Tools Command Prompt for VS 2022
\item 64-bit: VS2022/x64 Native Tools Command Prompt for VS 2022
\end{itemize}
\item Navigate to the ORE root directory
\item Run CMake command:
\begin{itemize}
\item 64-bit: \\
{\tt cmake -G "Visual Studio 17 2022" -A x64 -DBOOST\_INCLUDEDIR=\%BOOST\% -DBOOST\_LIBRARYDIR=\%BOOST\_LIB64\% -DQL\_ENABLE\_SESSIONS=ON -DMSVC\_LINK\_DYNAMIC\_RUNTIME=true -B build}
\item 32-bit: \\
{\tt cmake -G "Visual Studio 17 2022" -A x32 -DBOOST\_INCLUDEDIR=\%BOOST\% -DBOOST\_LIBRARYDIR=\%BOOST\_LIB32\% -DQL\_ENABLE\_SESSIONS=ON -DMSVC\_LINK\_DYNAMIC\_RUNTIME=true -B build}
\end{itemize}
Replace the generator "Visual Studio 17 2022" with the actual installed version.
The solution and project files will be generated in the {\tt $\langle$ORE\_ROOT$\rangle${\bs}build} subdirectory.
\end{itemize}

\item build the cmake project with the command {\tt cmake --build build -v --config Release}, 

\item or open the MSVC solution file {\tt build{\bs}ORE.sln} and build the entire solution with Visual Studio (again, make sure to select the correct platform in the configuration manager first).
\end{enumerate}

\subsubsection*{Optional: Install optional dependencies with VCPKG}

VCPKG is an open source c++ library manager. ORE can be built optionally with ZLIB and Eigen library support. 

For both features the libraries needed to be installed on the system. On Windows one can use the VCPKG package manager to install those dependencies:

\begin{itemize}
\item Install vcpkg: https://vcpkg.io/en/getting-started.html
\item Install dependencies with invoking the command \\
\medskip
{\tt vcpkg install --triplet x64-windows zlib} \\
{\tt vcpkg install --triplet x64-windows eigen3} \\
\medskip
\end{itemize}

To make VCPKG visible to CMAKE, create an environment variable {\tt VCPKG\_ROOT} pointing to the root of the vcpkg directory and configure ORE with the flag {\tt -DCMAKE\_TOOLCHAIN\_FILE=\%VCPKG\_ROOT\%/scripts/buildsystems/vcpkg.cmake}. 

To use VCPKG with Visual Studio add the toolChainFile to the configurePresets in the CMakePresets.json:

{\tt "toolchainFile": "\$env\{VCPKG\_ROOT\}/scripts/buildsystems/vcpkg.cmake",}

\subsubsection*{Unix}

With the 5th release we have discontinued automake support so that ORE can only be built with CMake on Unix systems, as follows.

\begin{enumerate}
\item set environment variable to locate the boost include and boost library directories\\
\medskip
  {\tt export BOOST\_LIB=path/to/boost/lib}\\
  {\tt export BOOST\_INC=path/to/boost/include}
\medskip
\item Change to the ORE project directory that contains the {\tt QuantLib}, {\tt QuantExt}, etc, folders; create subdirectory {\tt build} and change to subdirectory {\tt build}
\item Configure CMake by invoking \\
\medskip
{\tt cmake -DBOOST\_ROOT=\${BOOST\_INC} -DBOOST\_LIBRARYDIR=\${BOOST\_LIB} -DQL\_ENABLE\_SESSIONS=ON ..} \\
\medskip
where the {\tt QL\_ENABLE\_SESSIONS} variable is set to ON in order to enable some multi-threading applications in ORE.

Alternatively, set environment variables {\tt BOOST\_ROOT} and {\tt BOOST\_LIBRARYDIR} directly and run \\
\medskip
{\tt cmake ..} \\
\medskip
\item Build all ORE libraries, QuantLib, as well as the doxygen API documentation for QuantExt, OREData and OREAnalytics, by invoking \\
\medskip
{\tt make -j4} \\
\medskip
using four threads in this example.
\medskip
\item Run all test suites by invoking \\
\medskip
{\tt ctest -j4}
\item Run Examples (see section \ref{sec:examples})
\end{enumerate}

Note: 
\begin{itemize}
\item If the boost libraries are not installed in a standard path they might not be found during runtime because of a missing rpath
tag in their path. Run the script {\tt rename\_libs.sh} to set the rpath tag in all libraries located in {\tt
  \${BOOST}/stage/lib}.
\item Unset {\tt LD\_LIBRARY\_PATH} respectively {\tt DYLD\_LIBRARY\_PATH} before running the ORE executable or the test suites, in order not to override the rpath information embedded into the libaries built with CMake
\item On Linux systems, the 'locale' settings can negatively affect the ORE process and output. To avoid this, we
recommend setting the environment variable {\tt LC\_NUMERIC} to {\tt C}, e.g. in a bash shell, do

{\tt\footnotesize
\% export LC\_NUMERIC=C
}

before running ORE or any of the examples below. This will suppress thousand separators in numbers when converted to
strings.

\item Generate {\tt CMakeLists.txt}:

The .cpp and .hpp files included in the build process need to be explicitly specified in the various {\tt CMakeLists.txt} 
files in the project directory. The python script (in {\tt Tools/update\_cmake\_files.py}) can be used to update all CMakeLists.txt files 
automatically. 

\end{itemize}

\subsubsection{Supported Compiler and Boost Versions}
\label{sec:compiler_boost_versions}

The following table \ref{tab:compiler_boost_versions} reflects the compiler / boost version combinations that
the users/developers at Acadia/LSEG can confirm as working combinations with the latest ORE v12.

\begin{table}[hbt]
  \begin{tabular}{|l|c|c|}
    \hline
    Compiler & Boost & ORE \\
    \hline
    AppleClang version 13.0.0 & 1.82.0 & 12\\
    AppleClang version 13.1.6 & 1.84.0 & 12\\
    AppleClang version 14.0.0 & 1.82.0 & 12\\
    AppleClang version 15.0.0 & 1.83.0 & 12\\
    gcc 10.2.1 & 1.74.0 & 12 \\
    gcc 11.4.0 & 1.74.0 & 12 \\
    VS2019 & 1.72.0 & 12 \\
    VS2022 & 1.78.0 & 12 \\
    VS2022 & 1.83.0 & 12 \\
    \hline
  \end{tabular}
  \caption{Supported compiler and boost versions for ORE v12.}
  \label{tab:compiler_boost_versions}
\end{table}

\subsubsection*{ZLIB support}

To enable zlib support configure CMake with the flag {\tt -DORE\_USE\_ZLIB=ON}. 

If zlib is not installed on the system, it can be installed on Windows with the package manager VCPKG.

\subsection{Python and Jupyter}\label{sec:python}

Python (version 3.5 or higher) is required to use the ORE Python language bindings in section \ref{sec:oreswig}, 
or to run the examples in section \ref{sec:examples} and plot exposure
evolutions. Moreover, we use Jupyter \cite{jupyter} in section \ref{sec:visualisation} to visualise simulation
results. Both are part of the 'Anaconda Open Data Science Analytics Platform' \cite{Anaconda}. Anaconda installation
instructions for Windows, OS X and Linux are available on the Anaconda site, with graphical installers for
Windows\footnote{With Windows, after a fresh installation of Python the user may have to run the {\tt python} command
  once in a command shell so that the Python executable will be found subsequently when running the example scripts in
  section \ref{sec:examples}.}, Linux and OS X.

With Linux and OS X, the following environment variable settings are required
\begin{itemize}
\item set {\tt LANG} and {\tt LC\_ALL } to {\tt en\_US.UTF-8} or {\tt en\_GB.UTF-8}
\item set {\tt LC\_NUMERIC} to {\tt C}. 
\end{itemize}
The former is required for both running the Python scripts in the examples section, as well as successful installation
of the following packages. \\

The full functionality of the Jupyter notebook introduced in section \ref{sec:jupyter} requires furthermore installing
\begin{itemize}
\item jupyter\_dashboards: \url{https://github.com/jupyter-incubator/dashboards}
\item ipywidgets: \url{https://github.com/ipython/ipywidgets}
\item pythreejs: \url{https://github.com/jovyan/pythreejs}
\item bqplot: \url{https://github.com/bloomberg/bqplot}
\end{itemize}
With Python and Anaconda already installed, this can be done by running these commands
\begin{itemize}
\item {\tt conda install -c conda-forge ipywidgets}
\item {\tt pip install jupyter\_dashboards}
\item {\tt jupyter dashboards quick-setup --sys-prefix}
\item {\tt conda install -c conda-forge bqplot}
\item {\tt conda install -c conda-forge pythreejs}
\end{itemize}
Note that the bqplot installation requires the environment settings mentioned above.

\subsection{Building ORE-SWIG and Python Wheels}\label{sec:oreswig}

Since release 4, ORE comes with Python and Java language bindings following the QuantLib-SWIG example.
The ORE bindings extend the QuantLib SWIG wrappers and allow calling ORE functionality in the 
QuantExt/OREData/OREAnalytics libraries alongside with functionality in QuantLib.  

\medskip
The ORE-SWIG source code is hosted in a separate git repository at \url{https://github.com/opensourcerisk/ore-swig}.
The {\tt README.md} in the top level directory of this git repository contains build instructions and refers to tutorials 
for installing and building Python wrappers and wheels.

\medskip
Typical usage of the Python wrapper is shown in ORE's {\tt Example\_42} and in ORE SWIG's {\tt OREAnalytics/Python/Examples} directory.
