\section{CVA Capital}\label{sec:cva}

General provisions
\begin{itemize}
\item Regulatory CVA may differ from accounting CVA, e.g. excludes the effect of the bank's own default (DVA), there are several best practices constraints
\item CVA risk is defined as the risk of losses arising from changing CVA values in response to changes in counterparty credit spreads and market risk factors
\item Transactions with qualified central counterparties are excluded from CVA Capital calculations
\item CVA Capital is calculated for the full portfolio (across all netting sets), including CVA hedges
\item There are two approaches, basic (BA-CVA) and standardised (SA-CVA), the latter requires regulatory approval. SA-CVA banks can carve out netting sets and apply BA-CVA to these.
\item There is a materiality threshold of 100 billion EUR aggregate notional of non-centrally cleared derivatives. When below, a bank can choose to set its CVA Capital to 100\% of the CCR Capital. The regulator can remove this option.
  %{\color{red}For the purpose of the GFMA impact study we assume that all participants exceed the materiality thresholds.}
\item When calculating CCR Capital, the maturity adjustment factor may be capped at 1 for all netting sets contributing to CVA Capital.
  %{\color{red} This distinguishes the SA-CCR calculations for netting sets with CCPs (no CVA Capital charge) from SA-CCR for netting sets with non-CCP (CVA charge applies).} 
\end{itemize}
 
\subsection{Basic Approach, BA-CVA}\label{sec:ba-cva}

There are two flavours of the basic approach
\begin{itemize}
\item a reduced version that does not recognise hedges
\item a full version that does
\end{itemize}

Note: The implementation in ORE covers the reduced version so far.

\subsubsection{Reduced Version}\label{sssec_bacva_reduced}

The total BA-CVA Capital charge according to the targeted revised framework \cite{d507} is 
$$
D_{\BACVA} \times K_{reduced}
$$
with discount scalar $D_{BA-CVA} = 0.65$ and
$$
K_{reduced} = \sqrt{\underbrace{\left(\rho\sum_c \SCVA_c\right)^2}_{\mbox{systematic component}} + \underbrace{(1-\rho^2)\sum_c \SCVA_c^2}_{\mbox{idiosyncratic component}}}
$$
where $\rho=0.5$ and $\SCVA_c$ is the stand-alone BA-CVA charge for counterparty $c$.

\medskip\noindent
Stand-alone BA-CVA Capital charge, sum over netting sets $\NS$:
$$
\SCVA_c=\frac{1}{\alpha} \cdot \RW_c \cdot \sum_{\NS} M_{\NS}\cdot \EAD_{\NS}\cdot \DF_{\NS}
$$
where 
\begin{itemize}
\item $\alpha=1.4$
\item $\RW_c$ is the counterparty risk weight, see \cite{d424} page 111
\item $M_{\NS}$ is the netting set's effective maturity, see paragraphs 38 and 39 of Annex 4 of the Basel II framework \cite{bcbs118}, page 216-217 
\item $\EAD_{\NS}$ is the netting set's exposure at default, calculated in the same way as the bank calculates it for minimum capital requirements for CCR
  %, i.e. see the SA-CCR section for the purpose of the GFMA impact study
\item $\DF_{\NS}$ is a supervisory discount factor, equal to 1 for banks that use IMM to calculate $\EAD$, otherwise equal to $(1-\exp(-0.05\cdot M_{\NS}))/(0.05\cdot M_{\NS})$
  %; for the GFMA impact study we use SA-CCR, hence the latter definition of $\DF_{\NS}$ is applied
\end{itemize}


\medskip\noindent
Note: The implementation in ORE
\begin{itemize}
\item uses the SA-CCR EAD amounts
\item ignores the idiosyncratic component so that 
\begin{align*}
K_{reduced} &= \rho\sum_c \SCVA_c \\
\BACVA &= D_{\BACVA} \times \rho\times \sum_c \SCVA_c 
\end{align*}
\end{itemize}

\subsubsection{Full Version}\label{sssec_bacva_full}

Eligible hedges are single-name or index CDS, referencing the counterparty directly or a counterparty in the same sector and region.
$$
K_{full} = \beta\cdot K_{reduced} + (1-\beta)\cdot K_{hedged}
$$
where $\beta=0.25$ to floor the effect of hedging, and
$$
K_{hedged} = \sqrt{\underbrace{\left(\rho\cdot\sum_c(\SCVA_c -\SNH_c)- IH\right)^2}_{\mbox{systematic component}} + \underbrace{(1-\rho^2)\cdot\sum_c (\SCVA_c -\SNH_c)^2}_{\mbox{idiosyncratic component}} + \underbrace{\sum_c \HMA_c}_{\mbox{indirect hedges}} }
$$
where
\begin{itemize}
\item $\SNH_c$ is a sum across all single-name hedges that are taken out to hedge the CVA risk of counterparty $c$:
$$
\SNH_c = \sum_{h\in c} r_{rc}\cdot \RW_h \cdot M^{SN}_h\cdot B^{SN}_h\cdot \DF^{SN}_h
$$
see \cite{d424} page 113 and \cite{d507}.
\item $\IH$ is a sum over index hedges the are taken out to hedge CVA risk:
$$
\IH = \sum_i \RW_i\cdot M^{ind}_i\cdot B^{ind}_i\cdot \DF^{ind}_i
$$
see \cite{d424} page 113-114 and \cite{d507}.
\item $\HMA_c$ is a ``hedging misalignment parameter'' to avoid that single-name hedges can take the capital charge to zero:
$$
\HMA_c = \sum_{h\in c} (1- r_{hc}^2)\cdot \RW_h \cdot M^{SN}_h\cdot B^{SN}_h\cdot \DF^{SN}_h
$$
with same parameters as in the calculation of $\SNH_c$.
\end{itemize}

Note: The full version of BA-CVA is not implemented in ORE yet.

\ifdefined\MethodologyOnly\else{
\subsubsection{Implementation Details}

Since the BA-CVA calculation is a relatively simple calculation that utilises the EAD amounts
in a SA-CCR calculation, the same configuration for SA-CCR applies for BA-CVA. See Section \ref{sec:saccr_implementation_details}.
}\fi % \ifdefined\MethodologyOnly

\subsection{Standard Approach, SA-CVA}\label{sec:sa-cva}

SA-CVA uses as inputs the sensitivities of regulatory CVA (see below) to counterparty credit spreads and market risk factors driving covered transactions' values.

\medskip\noindent
The SA-CVA calculation generally takes delta and vega risk into account for five risk types: interest rates (IR), foreign exchange (FX), reference credit spreads, equity and commodity. Note that vega risk includes sensitivity of option instruments and sensitivity of the CVA model calibration to input volatilities.
%%%({\color{red} hence the relevance of IR vega risk for the GFMA study}).

\medskip\noindent
We denote $s_k^{\CVA}$ the sensitivity of the aggregate CVA to risk factor $k$ and $s_k^{\Hdg}$ the sensitivity of all eligible CVA hedges to risk factor $k$. Eligible are hedges of both credit spreads and exposure components. Shift sizes are specified in section C.6 of \cite{d424}.
%%%{\color{red} For the GFMA impact study, the relevant risk factors and shifts are given in table \ref{tab:cva_sensi}:} 

Given the CVA sensitivities and regulatory risk weights and correlations, the calculation of SA-CVA is straightforward.

\noindent
Bucket level capital charge:
$$
K_b = \sqrt{ \left[\sum_{k\in b} \WS_k^2 + \sum_{k\ne l \in b} \rho_{kl} \cdot \WS_k \cdot \WS_l\right] + R\cdot \sum_{k\in b} \left(\WS_k^{\Hdg}\right)^2 }
$$
where $R=0.01$ and
$$
\WS_k = \WS_k^{\CVA} + \WS_k^{\Hdg}, \qquad \WS_k^{\CVA} = \RW_k \cdot s_k^{\CVA}, \qquad \WS_k^{\Hdg} = \RW_k \cdot s_k^{\Hdg}
$$
with risk weights $\RW_k$ and correlations $\rho_{kl}$ as specified in Section C.6 of \cite{d424} and in \cite{d507}.

\medskip\noindent
Bucket-level capital charges must then be aggregated across buckets within each risk type :
$$
K = m_{\CVA}\cdot\sqrt{ \sum_b K_b^2 + \sum_b \sum_{c\ne b} \gamma_{bc} \cdot K_b \cdot K_c}
$$
with multiplier $m_{\CVA}=1.25$ and correlation parameters $\gamma_{bc}$ as specified in Section C.6 of \cite{d424}.

\subsubsection{Regulatory CVA Calculation and CVA Sensitivity}

Regulatory CVA is the basis for the calculation of the CVA risk capital requirement. 
Calculations of regulatory CVA must be performed for each counterparty with which a bank has at least one covered position.

\medskip\noindent
Regulatory CVA at a counterparty level must be calculated according to the following principles \cite{d424} pages 115-117:
\begin{itemize}
\item Based on Monte Carlo simulation of exposure evolution, consistent with front office/accounting CVA
\item Risk neutral probability measure 
\item Model calibration to market data where possible
\item Use of PDs implied from credit spreads observed in the market, use of market-consensus LGDs
\item Netting recognition applies as in accounting CVA
\item Collateral (VM, IM): Exposure simulation must capture the effects of margining collateral that is recognised as a risk mitigant along each exposure path. All the relevant contractual features such as the nature of the margin agreement (unilateral vs bilateral), the frequency of margin calls, the type of collateral, thresholds, independent amounts, initial margins and minimum transfer amounts must be appropriately captured by the exposure model; the Margin Period of Risk has to be taken into account
  %({\color{red}For the GFMA study we assume independence because the portfolios do not contain credit derivatives, CVA hedges are taken out})
\end{itemize}

The regulatory CVA calculation is based on ORE's XVA analytics. It takes CSA details for simulating VM balances into account (thresholds, minimum transfer amounts), as well as the Margin Period of Risk. Initial Margin is modelled as a stochastic Dynamic Delta VaR along all Monte Carlo paths.

The product scope for CVA sensitivity and SA-CVA is so far:
\begin{itemize}
\item FX Forwards
\item FX Options
\item Cross Currency Swaps
\end{itemize}

For this scope we assume independence of credit and other market factors. With this simplification, the calculation of CVA sensitivities w.r.t. credit factors does not require the recalculation of exposure profiles. IR/FX delta and vega calculation, however, does require the recalculation of exposures under each shift scenario. The number of scenarios is minimized and tailored to the portfolio. Moreover we make use of multithreading and further parallelization techniques to reduce calculation times. The shifts/recalculations required to cover the product scope above are listed below in table \ref{tab:cva_sensi}.

\ifdefined\MethodologyOnly\else{
\subsubsection{Implementation Details}\label{sec:sacva_implementation_details}

This section describes some of the requirements and implementation of the SA-CVA capital analytic
in ORE.

The SA-CVA capital analytic uses the following input files:
\begin{enumerate}
  \item Netting Set Definitions (see User Guide)
  \item Counterparty Information (see User Guide)
  \item [Optional] Collateral Balances (see User Guide)
  \item [Optional] Scenario Generator Data (see the \lstinline!Parameters! block in the user guide's simulation parameterisation section)
  \item [Optional] Cross Asset Model (see the \lstinline!CrossAssetModel! block in the documentation above
  \item [Optional] Instantaneous Correlations (see the \lstinline!InstantaneousCorrelations! block in the documentation above
  \item [Optional] Scenario Sim Market Parameters (see \lstinline!Market! block in the documentation above
\end{enumerate}
}\fi % \ifdefined\MethodologyOnly

\begin{center}
\begin{table}[hbt]
\begin{tabular}{|p{4cm}|p{3cm}|p{8cm}|}
\hline
Risk Type & Risk Factor & Shift \\
\hline
IR Delta for currencies USD, EUR, GBP, AUD, CAD, SEK, JPY & 
Currency and tenor (1y, 2y, 5y, 10y, 30y) & Shift of each tenor point for all curves with the given currency curve, absolute shift size 1BP \\
\hline
IR Delta for any other currency  & 
Currency & 
Parallel shift of all yield curves for given given currency by 1BP \\
\hline
FX Delta & 
Foreign currency (vs. a fixed domestic currency) & 
FX shift for any foreign1/foreign2 currency pair is obtained by triangulation from the ``fundamental'' foreign/domestic FX rates; relative shift by 1\% for the base currency FX rate \\
\hline
FX Vega & 
Foreign currency (vs. a fixed domestic currency) & 
Volatility shift for any foreign1/foreign2 currency pair is implied from the ``fundamental'' foreign1/domestic and foreign2/domestic volatility and a fixed implied correlation; simultaneous 1\% relative shift for all volatilities in the fundamental volatility surface \\
\hline
Counterparty Credit Delta & 
Entity and tenor point (0.5y, 1y, 3y, 5y, 10y)& 
Absolute shift of the relevant credit spread by 1BP, aggregation of sensitivities across entities within sector buckets 1-7 \\
\hline
Reference Credit Delta & 
Reference credit sector (1-15) & 
Simultaneous absolute shift of the relevant credit spreads by 1BP, for all reference names in the bucket, across all tenor points\\
\hline
\end{tabular}
\caption{Risk factors and shifts.}
\label{tab:cva_sensi}
\end{table}
\end{center}

