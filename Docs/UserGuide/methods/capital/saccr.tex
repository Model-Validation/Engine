\section{Counterparty Credit Risk Capital}

Financial institutions either apply a {\em standardized} or an {\em advanced}
approach for determining the \index{capital!regulatory}regulatory capital amounts to be assigned
to their derivative activity. The advanced approach is
accessible to institutions with an internal model method (IMM) for
credit risk capital which is approved by the regulator. This method
involves sophisticated analysis of future exposures by Monte Carlo
simulation methods using real-world measure risk factor evolutions \cite{Lichters}.
Institutions without approved IMM have to apply a standardized approach instead, which is
simplified in that it does not require Monte Carlo exposure simulation
but resorts to formulas suggested by the Basel Commitee for Banking
Supervision and enforced by the respective regulator. These formulas
attempt to conservatively approximate the credit exposures which would
have been obtained by more sophisticated IMM approaches.

The former standardized approach (Current Exposure Method) as published by the Basel Commitee for
Banking Supervision (BCBS) in 2006 \cite{bcbs128} is summarized in section \ref{sec_cem}. 
A revised standardized approach for counterparty credit risk\index{CCR} from
derivative acitvity (SA-CCR) as published in 2014 \cite{bcbs279} is in
effect from beginning of 2017.

%-------------------------------------------------------------------------
\subsection{Current Exposure Method (CEM)}\label{sec_cem}
%-------------------------------------------------------------------------

The key quantity in the current standardized approach (or current
exposure method, CEM) is the exposure at default (EaD) or {\em
  credit equivalent amount} which consists of two additive terms,
current replacement cost and potential future exposure add-on, 
$$ 
\text{EaD} = \text{RC} + \text{Notional} \times \text{NettingFactor} \times \text{AddOn}. 
$$

The replacement cost RC is simply given by the current exposure
which is the current positive value of the netting set after subtracting
collateral (C) 
$$
\text{RC} = \max(0, \text{PV} - \text{C}).
$$
Replacement cost is aggregated over all derivative contracts and
netting sets.

The second term is supposed to approximately reflect the potential future exposure
over the remaining life of the contract. It depends on the (fairly rough) product type
classification and on time to maturity in three bands as shown in table
\ref{tab_addon} which is in use in this form since 1988.
\begin{table}[hbt]
\begin{center}
\begin{tabular}{|l || c | c | c | c | c |}
\hline
Residual & Interest & FX   & Equity & Precious & Other \\ 
Maturity &  Rates  & Gold &    & Metals   & Commodities\\
\hline
\hline
$\leq$ 1 Year & 0.0\% & 1\% & 6\% & 7\% & 10\% \\
1-5 Years     & 0.5\% & 5\% & 8\% & 7\%   & 12\% \\
$>$ 5 Years     & 1.5\% & 7.5\% & 10\% & 8\% & 15\% \\
\hline
\end{tabular}
\end{center}
\caption{Add-on factor by product and time to maturity. Single
  currency interest rate swaps are assigned a zero add-on, i.e.\ judged
  on replacement cost basis only, if their maturity is less than one year. Forwards,
  swaps, purchased options and derivative contracts not covered in
  the columns above shall be treated as ``Other Commodities''. Credit
  derivatives (total return swaps and credit default swaps) are
  treated separately with 5\% and 10\% add-ons depending on whether
  the reference obigation is regarded as ``qualifying'' (public sector
  entities (!), rated investment grade or approved by the regulator). 
N-th to default basket transactions are assigned an add-on 
based on the  credit quality of n-th lowest credit quality in the basket. }
\label{tab_addon}
\end{table}

The netting factor acknowledges netting also in the potential future
exposure estimate of the netting set. If there is no netting as of today, i.e.\  
net NPV equals gross NPV, the netting factor is equal to 1, otherwise
it can be as low as 0.4. Additional netting benefit results from
using a central clearing counterparty (CCP): 
$$
\text{NettingFactor} = \left\{\begin{array}{ll}
\displaystyle 0.4 + 0.6 \times \frac{\max\left(\sum_i PV_i, 0\right)}{\sum_i
  \max(PV_i,0)} & \text{bilateral netting}\\ \\
\displaystyle 0.15 + 0.85 \times \frac{\max\left(\sum_i PV_i, 0\right)}{\sum_i
  \max(PV_i,0)} & \text{central clearing}
\end{array}
\right.
$$

Finally, the notional amounts that enter into the potential future
exposure term are understood as effective notional amounts which
e.g.\ take into account leverage which may be expressed through
factors in structured product payoff formulas. 

\medskip
\noindent 
Note that CEM was valid until end of 2016.

\subsection{Standardized Approach for Counterparty Credit Risk (SA-CCR)}

With its 2014 publication \cite{bcbs279}, the \index{BCBS}Basel Committee for
Banking Supervision has revised  the standardized approach. The new
method takes collateralization into account in a more detailed way
than before which has the potential to reduce the credit equivalent
amounts. On the other hand, the new method attempts to mimic a more
conservative potential exposure, the {\em effective expected positive
exposure} \index{effective expected positive
exposure|see{EEPE}}(EEPE) which tends to increase the resulting credit
equivalent amounts. In summary, only a detailed impact analysis for
specific portfolios will be able to tell whether the overall impact of
the new method results in an increase or in a decrease of derivative capital charges.
In the following we summarize the ingredients of the new methodology.

SA-CCR in ORE currently supports the following trade types per asset class (unsupported
types are ignored in calculations with a structured warning raised): \begin{itemize}
  \item Foreign Exchange: \emph{Swap} (cross currency), \emph{CrossCurrencySwap}, \emph{FxBarrierOption}, \emph{FxForward}, \emph{FxOption}, \emph{FxTouchOption}
  \item Commodity: \emph{CommodityForward}, \emph{CommoditySwap}
  \item Interest Rate: \emph{Swap} (Vanilla IR, basis and CPI swaps not yet supported), \emph{Swaption} (European)
  \item Equity: \emph{EquityOption}
\end{itemize}

Note for swap-type products that two legs are required.

\subsubsection{Exposure at Default (EAD)}
The EAD is still composed of a replacement cost and a potential future exposure add-on term
but scaled up by a factor of $1.4$ which is motivated by the committee's attempt
to mimic a different (higher) exposure measure:
$$
\text{EAD} = 1.4 \times (\text{RC} + \text{PFE})
$$
On the other hand, the replacement cost per netting set takes into account
more details of the collateral agreement: 
$$
\text{RC} = \max(\text{PV} - \text{C};\; \text{TH} + \text{MTA} - \text{NICA};\; 0)
$$
where 
\begin{itemize}
\item PV is the netting set mark-to-market value,
\item TH is the CSA's threshold amount, 
\item MTA the CSA's minimum transfer amount, 
\item NICA is the independent collateral \index{independent amount} amount (i.e.\ any received independent 
amount plus initial margin amount),
\item C is the current collateral (i.e.\ variation margin plus NICA).
\end{itemize}
So even if the posted collateral C matches the PV so that the first
term on the right-hand side vanishes, the various CSA slippage
terms can cause a positive replacement cost contribution here. This
supposedly imitates CVA behaviour.

For unmargined netting sets,
\begin{equation*}
\text{RC} = \max(\text{PV} - \text{C};\; 0).
\end{equation*}

\subsubsection{Potential Future Exposure (PFE)}
The PFE term is primarily driven by the aggregate add-on factor
which is significantly more complex than in the current standardized approach's definition:
$$
\text{PFE} = \text{Multiplier} \times \text{AddOn}.
$$
The PFE is obtained by scaling down the AddOn using a multiplier which recognises and rewards excess collateral: 
\begin{align*}
\text{Multiplier} &= \min\left(1; 0.05 + 0.95 \times \exp\left(\frac{\text{PV}-\text{C}}{1.9
  \times \text{AddOn}} \right)\right)  \\
&=1 \quad \text{if $PV \geq C$}, \\
&<1 \quad \text{if $PV < C$ (excess collateral).} 
\end{align*}
The aggregate/netting set add-on is a composite

\begin{equation*}
\text{AddOn} = \sum_a \text{AddOn}^{(a)}
\end{equation*}
where the sum is taken over the various asset classes in the netting
set, and AddOn$^{(a)}$ denotes the add-on factor for asset class $a$.
The add-on factor within each asset class $a$

\begin{equation}\label{sa-ccr-addon}
\text{AddOn}^{(a)} = \sum_i \text{AddOn}^{(a)}_i
\end{equation}
is a sum of the hedging set add-ons (where AddOn$^{(a)}_i$ is the add-on for hedging set $i$ in asset class $a$).
The list of possible hedging sets for each asset class is given in Table \ref{tab:hedgingset}. The asset class
assignment is based on a trade's ``primary risk driver'', but split assignment may be required for complex trades.

{
\begin{table}[hbt]
\begin{center}
\scriptsize
\begin{tabular}{|l|p{5cm}|p{5cm}|}
\hline
Asset Class & Hedging Set & Hedging Subset \\
\hline
\hline
Interest Rate & Currency & Maturity buckets (\emph{1Y}, \emph{1Y-5Y}, \emph{5Y})\\
\hline
Foreign Exchange & Currency pair & - \\
\hline
Equity & - & Qualifier \\
\hline
Credit & - & Qualifier \\
\hline
Commodity & \emph{Energy}, \emph{Metal}, \emph{Agriculture}, \emph{Other} & Qualifier/Group \\
\hline
\end{tabular}
\caption{Hedging set/subset construction by asset class -- See \cite{bcbs279} paragraph 161.}
\label{tab:hedgingset}
\end{center}
\end{table}
}

Note: 
\begin{itemize}
  \item For Interest Rate, a trade is assigned to a hedging subset
  based on the end date of the period referenced by the underlying $E_i$.\footnote{See \cite{bcbs279} paragraph 166.}
  \item For Commodity, similar underlyings can be grouped
  under the same hedging subset.\footnote{See \cite{bcbs279} Example 3}
  Currently, similar underlyings will be grouped together under the following categories:
  \emph{Coal}, \emph{Crude oil}, \emph{Light Ends}, \emph{Middle Distillates}, \emph{Heavy Distillates},
  \emph{Natural Gas}, \emph{Power}.
  \item For Equity and Credit, a single hedging set is used for the entire asset class.
  The hedging subset is then given by the underlying entity, where partial offsetting is applied across different entities, and full offset is applied within each entity.
  \item Within each asset class, a separate hedging set is reserved for basis trades and volatility/variance trades.
  Basis hedging sets are given in the format \lstinline!QUALIFIER1/QUALIFIER2!.
  Volatility/variance trades are not yet supported.
\end{itemize}

As mentioned above, the potential future exposure term aims to mimic a
particularly conservative exposure measure. This choice is built into
the definition of the supervisory factors, quoting \cite{bcbs279}: 
``A factor or factors specific to each asset class is used to 
{\bf convert the effective notional amount into Effective EPE} 
based on the measured volatility of the asset class. Each factor has 
been calibrated to reflect the Effective EPE of a single \index{at the
  money}at-the-money 
linear trade of unit notional and one-year maturity. This includes the 
estimate of realised \index{volatility!realized}volatilities assumed by supervisors for each 
underlying asset class.''
The supervisory factors are displayed in Table \ref{tab_saccr_vols}.

\subsubsection{Trade-Specific Parameters}

Before defining the hedging set add-on, AddOn$^{(a)}_i$ (\ref{sa-ccr-addon}),
for each asset class we define the trade-specific parameters that will be used.

The trade specific parameters $\delta_i$, $d_j^{(a)}$ and $MF_i^{(\text{type})}$ are defined
(for trade $j$ and asset class $a$) as follows:
\begin{enumerate}
\item {\bf $d_j^{(a)}$ (trade-level adjusted notional)}
\begin{itemize}
\item {\bf Foreign Exchange}:\begin{itemize}
  \item For trades where one of the legs is in the base currency: The adjusted notional is the foreign leg notional
  converted to the base currency.
  \item For trades where both legs are denominated in a currency other than the base currency: Both leg notionals are
  converted to the base currency, and the larger of the 2 notionals is used as the adjusted notional.
\end{itemize}
\item {\bf Interest Rate, Credit}: $\text{Notional} \times \text{SD}_j$, where SD$_j$ is the supervisory duration,
with \begin{equation*}
 \text{SD}_j = \frac{\exp{(-0.05 \cdot S_j)} - \exp{(-0.05 \cdot E_j)}}{0.05},
\end{equation*}
where \begin{itemize}
    \item $S_j$ is the start date (in years) of the period referenced by the underlying,
    \item $E_j$ is the end date (in years) of the period referenced by the underlying.
  \end{itemize}
\item {\bf Equity, Commodity}: $\text{Price per unit} \times \text{No.\ of units}$
\item Note: We index the adjusted notional by the asset class $a$ is because complex trades can be assigned
  to multiple asset classes and hence have an adjusted notional in more than one asset class.
\end{itemize}
\item {\bf $\MF_j^{(\text{type})}$ (maturity factor)}
\begin{itemize}
\item For uncollateralized positions, this is computed from the time to
 maturity $M_j$ (in years) of the trade: $MF_j^{(\text{unmargined})} = \sqrt{\min(\max(M_j, 2/52),1)}$. Note the
 floor of 10 business days on $M_j$.
\item For collateralized positions, this is computed from the margin period of
  risk MPR (in years) used: $MF_j^{(\text{margined})} = 1.5\cdot\sqrt{\MPR}$ 
\end{itemize}
\item {\bf $\delta_j$ (delta adjustment for direction and non-linearity)}
\begin{itemize}
\item Options: In this case $\delta_j$ is an option delta (derived from
  the Black76 formula),
$$\delta_j = \omega \cdot \Phi\!\left(\phi \cdot \frac{\ln(P_j/K_j) +
    0.5\,\sigma_j^2\,T_j}{\sigma_j\,\sqrt{T_j}} \right)\!,$$
where \begin{itemize}
    \item $\Phi(\cdot)$ is the cumulative normal distribution function,
    \item $P_j$ is the price of the underlying (typically the forward price),
    \item $K_j$ is the strike price of the option,
    \item $T_j$ is the latest option exercise date,
    \item $\omega$ is $+1$ for long calls and short puts, $-1$ for short calls and long puts,
    \item $\phi$ is $+1$ for calls, $-1$ for puts,
    \item $\sigma_j$ is the supervisory option volatility as defined in Table \ref{tab_saccr_vols}.
  \end{itemize}  
 \item CDO tranches: $\pm \frac{15}{(1+14\,A)(1+14\,D)}$ for purchased
  (sold) protection, where A and D denote the attachment and
  detachment point of the tranche, respectively
\item Others: $\pm 1$ depending on whether long or short in the primary risk factor
\end{itemize}
\end{enumerate}

\subsubsection{Hedging Set/Subset Add-On}

We continue with sketching the hedging set level add-on from (\ref{sa-ccr-addon}).

\subsubsection*{Interest Rate}

Within currency hedging set $i$, each trade is assigned to 1 of 3 maturity buckets based on the end date of the period referenced by the trade's underlying:
\begin{itemize}
\item $D_1$: $<$ 1 year
\item $D_2$: 1-5 years
\item $D_3$: $>$ 5 years
\end{itemize}

The effective notional $D_{i, k}$ for maturity bucket $k$ of currency hedging set $i$
\begin{equation*}
  D_{i, k} = \sum_{j=1}^n \delta_j \times d_j^{(\text{IR})} \times \text{MF}_j^{(\text{type})}.
\end{equation*}
is calculated as the sum of all trades $j$ in each maturity bucket.

Partial offsetting is applied when aggregating the contribution across the 3 maturity buckets,
giving the effective notional for hedging set $i$:
\begin{equation*}
  \text{EffectiveNotional}_i^{(\text{IR})} = \sqrt{D_{i,1}^2 + D_{i,2}^2 + D_{i,3}^2 + 1.4 \cdot (D_{i,1}\cdot D_{i,2} + D_{i,2}\cdot D_{i,3}) 
+ 0.6 \cdot D_{i,1} \cdot D_{i,3}}
\end{equation*}
Each contribution is a sum over all trades in each hedging set $i$ and maturity bucket: 
One may choose not to apply any offset across maturity buckets, in which case
\begin{equation*}
  \text{EffectiveNotional}_i^{(\text{IR})} = |D_{i,1}| + |D_{i,2}| + |D_{i,3}|.
\end{equation*}
Finally, we multiply the effective notional by the supervisory factor (see Table \ref{tab_saccr_vols}) to
obtain the add-on:
\begin{equation*}
  \text{AddOn}_i^{(\text{IR})} = SF_i^{(\text{IR})} \times \text{EffectiveNotional}_i^{(\text{IR})}
\end{equation*}

\subsubsection*{Foreign Exchange}

Unlike for IR instruments, FX does not use hedging subsets (i.e.\ the notional amounts are maturity independent)
so trades within the same currency pair hedging set are allowed to fully offset each other. The effective notional
calculation is similar to that of IR:
\begin{equation*}
  \text{EffectiveNotional}_i^{(\text{FX})} = \sum_{j=1}^n \delta_j \times d^{(\text{FX})}_j \times MF_j^{(\text{type})}
\end{equation*}
and
\begin{equation*}
  \text{AddOn}_i^{(\text{FX})} = SF_i^{(\text{FX})} \times \left|\text{EffectiveNotional}_i^{(\text{FX})}\right|
\end{equation*}

\subsubsection*{Credit}

All credit instruments are assigned to a single hedging set, and hedging subsets are defined by each underlying
entity/name. Trades that reference the same entity are fully offset, giving the entity-level effective notional
amount (for hedging subset $k$):
\begin{equation*}
  \text{EffectiveNotional}_k^{(\text{CR})} = \sum_{j=1}^n \delta_j \times d_j^{(\text{CR})} \times MF_j^{(\text{type})}
\end{equation*}
Multiplying by the supervisory factor, we get the entity-level add-on (for hedging subset $k$):
\begin{equation*}
  \text{AddOn}_k^{(\text{CR})} = SF_k^{(\text{CR})} \times \text{EffectiveNotional}_k^{(\text{CR})}
\end{equation*}
For single-name entities, the supervisory factor is determined by the credit rating, while for index entities this is
determined based on whether the index is investment grade or speculative grade (see Table \ref{tab_saccr_vols}).

Finally, the asset class add-on is calculated by applying a partial offset across the entity-level add-ons:
\begin{equation*}
  {\text{AddOn}^{(\text{CR})} = \sqrt{ \underbrace{\left(\sum_k \rho_k^{(\text{CR})} \cdot \text{AddOn}_k^{(\text{CR})} \right)^2}_{\mbox{systematic component}} + \underbrace{\sum_k \left( 1 - \left( \rho_k^{(\text{CR})} \right)^2 \right) \cdot \left(\text{AddOn}_k^{(\text{CR})}\right)^2}_{\mbox{idiosyncratic component}} }}
\end{equation*}

\subsubsection*{Equity}

The hedging set/subset construction for equities is similar to that for credit, so the same calculation applies for effective notional where a hedging subset is formed for each equity underlying:
\begin{equation*}
  \text{EffectiveNotional}_k^{(\text{EQ})} = \sum_{j=1}^n \delta_j \times d_j^{(\text{EQ})} \times MF_j^{(\text{type})}
\end{equation*}
Likewise, for the entity-level add-on:
\begin{equation*}
  \text{AddOn}_k^{(\text{EQ})} = SF_k^{(\text{EQ})} \times \text{EffectiveNotional}_k^{(\text{EQ})}
\end{equation*}
There are only 2 supervisory factors for equities, based on whether the underlying is a single name or an index
(see Table \ref{tab_saccr_vols}).

Finally, we apply partial offset once again across the entity-level add-ons to get the asset class add-on:
\begin{equation*}
  {\text{AddOn}^{(\text{EQ})} = \sqrt{ \underbrace{\left(\sum_k \rho_k^{(\text{EQ})} \cdot \text{AddOn}_k^{(\text{EQ})} \right)^2}_{\mbox{systematic component}} + \underbrace{\sum_k \left( 1 - \left( \rho_k^{(\text{EQ})} \right)^2 \right) \cdot \left(\text{AddOn}_k^{(\text{EQ})}\right)^2}_{\mbox{idiosyncratic component}} }}
\end{equation*}

\subsubsection*{Commodity}

The Commodity asset class uses 4 hedging sets (and no offsetting is allowed between hedging sets in any asset class),
the Commodity add-on is more specifically defined as:
\begin{equation*}
  \text{AddOn}^{(\text{COM})} = \text{AddOn}_\text{Energy}^{(\text{COM})} + \text{AddOn}_\text{Metal}^{(\text{COM})} + \text{AddOn}_\text{Agriculture}^{(\text{COM})} + \text{AddOn}_\text{Other}^{(\text{COM})}
\end{equation*}
For Commodity, the calculation of hedging set level add-ons is the same as the calculation of asset class
add-ons for Equity and Credit. As before, we start with calculating the effective notional at the entity level
(i.e.\ hedging subset $k$) under hedging set $i$, applying full offset across trade contributions:
\begin{equation*}
  \text{EffectiveNotional}_{i,k}^{(\text{COM})} = \sum_{j=1}^n \delta_j \times d_j^{(\text{COM})} \times MF_j^{(\text{type})}
\end{equation*}
Then we calculate the add-on for hedging subset $k$:
\begin{equation*}
  \text{AddOn}_{i,k}^{(\text{COM})} = SF_i^{(\text{COM})} \times \text{EffectiveNotional}_{i,k}^{(\text{COM})}
\end{equation*}
Finally, we get the add-on for hedging set $i$ (note the correlation terms are outside the sums as they apply to
the hedging set, not to the hedging subset):
\begin{equation*}
  {\text{AddOn}_i^{(\text{COM})} = \sqrt{ \underbrace{\left( \rho_i^{(\text{COM})} \cdot \sum_k \text{AddOn}_{i,k}^{(\text{COM})} \right)^2}_{\mbox{systematic component}} + \underbrace{ \left( 1 - \left( \rho_i^{(\text{COM})} \right)^2 \right) \cdot \sum_k \left(\text{AddOn}_{i,k}^{(\text{COM})}\right)^2}_{\mbox{idiosyncratic component}} }}
\end{equation*}

{
\begin{table}[hbt]
\begin{center}
\scriptsize
\begin{tabular}{|l|c|p{1.5cm}|c|p{1.5cm}|}
\hline
Asset Class& Subclass & Supervisory Factor & Correlation &
Supervisory Option Volatility\\
\hline
\hline
Interest rate && 0.5 \% & N/A & 50\%\\
\hline
Foreign exchange && 4.0 \% & N/A & 15\%\\
\hline
Credit, Single Name & AAA & 0.38\% & 50\% & 100\%\\
\cline{2-5}
& AA & 0.38\% & 50\% & 100\%\\
\cline{2-5}
& A & 0.42\% & 50\% & 100\%\\
\cline{2-5}
& BBB & 0.54\% & 50\% & 100\%\\
\cline{2-5}
& BB & 1.06\% & 50\% & 100\%\\
\cline{2-5}
& B & 1.6\% & 50\% & 100\%\\
\cline{2-5}
& CCC & 6.0\% & 50\% & 100\%\\
\hline
Credit, Index & IG & 0.38\% & 80\% & 80\%\\
\cline{2-5}
& SG & 1.06\% & 80\% & 80\%\\
\hline
Equity, Single Name && 32\% & 50\% & 120\%\\
\hline
Equity, Index && 20\% & 80\% & 75\% \\
\hline
Commodity & Electricity & 40\% & 40\% & 150\%\\
\cline{2-5}
& Oil/Gas & 18\% & 40\% & 70\%\\
\cline{2-5}
& Metals & 18\% & 40\% & 70\%\\
\cline{2-5}
& Agricultural & 18\% & 40\% & 70\%\\
\cline{2-5}
& Other & 18\% & 40\% & 70\%\\
\hline
\end{tabular}\caption{Supervisory factors and option volatilities from \cite{bcbs279} Table 2.}\label{tab_saccr_vols}
\end{center}
\end{table}
}

\subsubsection{Capital Charge}\label{sssec_saccr_capital_charge}

The CCR capital charge is then computed as 
$$
K = \EAD \times \underbrace{\PD \times \LGD}_{\text{Risk Weight}}
$$

In the {\em Standardized Approach}, the risk weight is given by a fixed percentage depending on the
counterparty type and its external rating \cite{bcbs128, gregory2020}, see table \ref{tab:standard_weights}.
\begin{table}[hbt]
\scriptsize
\begin{tabular}{|c|c|c|c|c|c|c|}
\hline
 & AAA to AA- & A+ to A- & BBB+ to BBB- & BB+ to BB- & Below BB- & Unrated \\
\hline
\hline
Sovereign & 0\% & 20\% & 50\% & 100\% & 150\% & 100\% \\
Financials & 20\% & 30\% & 50\% & 100\% & 150\% & 100\% \\
Corporate & 20\% & 50\% & 75\% & 100\% & 150\% & 100\% \\
\hline
\end{tabular}
\caption{Example risk weights under the standardized approach for credit risk.}
\label{tab:standard_weights}
\end{table}
Under the {\em Internal Ratings-based Approach (IRB)}, banks can use their internal estimates of PD
({\em Foundation IRB}), or of PD and LGD ({\em Advanced IRB}).


\subsubsection{Risk Weighted Assets (RWA)}

RWA is calculated as a simple multiple of the capital charge 
$$
\RWA = 12.5 \times K
$$

\ifdefined\MethodologyOnly\else{
\subsubsection{Implementation Details}\label{sec:saccr_implementation_details}

This section describes some of the requirements and implementation of the SA-CCR capital analytic
in ORE.

\subsubsection*{Configuration Defaults}

The SA-CCR capital analytic requires the following input files (with default values specific to the
SA-CCR as outlined below; defaults that are not specific to any analytic are outlined in the
corresponding subsection for that input file):
\begin{enumerate}
  \item \textbf{Netting Set Definitions} (see the User Guide for a full description of the file format)
    \begin{itemize}
      \item \lstinline!ActiveCSAFlag!: Defaults to \emph{False}
      \item \lstinline!MarginPeriodOfRisk!: Defaults to \emph{2W}
      \item \lstinline!IndependentAmountHeld!: Defaults to zero
      \item \lstinline!ThresholdReceive!: Defaults to zero
      \item \lstinline!MinimumTransferAmountReceive!: Defaults to zero
      \item \lstinline!CalculateIMAmount!: Defaults to zero
      \item \lstinline!CalculateVMAmount!: Defaults to zero
    \end{itemize}
  \item \textbf{Counterparty Information} (see the User Guide for a full description of the file format)
    \begin{itemize}
      \item \lstinline!ClearingCounterparty!: Defaults to \emph{False}
      \item \lstinline!SaCcrRiskWeight!: Defaults to \emph{1.0}
    \end{itemize}
  \item \textbf{Collateral Balances} (see the User Guide for a full description of the file format)
    \begin{itemize}
      \item \lstinline!Currency!: Defaults to \emph{USD}
      \item \lstinline!IndependentAmountHeld!: Defaults to zero
      \item \lstinline!InitialMargin!: Defaults to zero
      \item \lstinline!VariationMargin!: Defaults to zero
    \end{itemize}
\end{enumerate}

For each trade in the portfolio XML its netting set ID and counterparty ID (specified in the
\lstinline!Envelope! node) should have a corresponding entry in the netting set definitions
and counterparty information files, respectively. If, for example, the corresponding netting
set definition for the trade is missing, then an entry will be created internally with all
the necessary inputs taking on the default values outlined above, with an appropriate warning
provided in the log file.

Similarly for the collateral balances, an entry is required for all collateralised netting sets.
If no entry is provided for the netting set of a given trade, a default collateral balance entry
will be assumed, with an appropriate warning message.

If a configuration file is not provided at all (and hence all counterparties and netting sets are
missing configurations), no warning messages will be given, and all entries in the configuration
will assume the defaults.

\subsubsection*{Additional Requirements/Details}

Since the SA-CCR aggregation step is done at the netting set level, there can only be one
counterparty associated to each netting set at most, i.e.\ there can only be a one-to-one or
many-to-one mapping from netting set ID to counterparty ID. This association is determined based
on the portfolio XML. For example, for a two-trade portfolio, we would have the following mappings
(from netting set to counterparty):
\begin{itemize}
  \item One-to-one
    \begin{enumerate}
      \item \lstinline!NettingSetId! = ``NS\_A'', \lstinline!CounterpartyId! = ``CPTY\_A''
      \item \lstinline!NettingSetId! = ``NS\_B'', \lstinline!CounterpartyId! = ``CPTY\_B''
    \end{enumerate}
  \item Many-to-one
    \begin{enumerate}
      \item \lstinline!NettingSetId! = ``NS\_A'', \lstinline!CounterpartyId! = ``CPTY\_A''
      \item \lstinline!NettingSetId! = ``NS\_B'', \lstinline!CounterpartyId! = ``CPTY\_A''
    \end{enumerate}
  \item One-to-many (not allowed)
    \begin{enumerate}
      \item \lstinline!NettingSetId! = ``NS\_A'', \lstinline!CounterpartyId! = ``CPTY\_A''
      \item \lstinline!NettingSetId! = ``NS\_A'', \lstinline!CounterpartyId! = ``CPTY\_B''
    \end{enumerate}
\end{itemize}

\vspace{1em}

For a given netting set, if the initial margin amount is set to be calculated internally (see
\lstinline!CalculateIMAmount! in the User Guide's netting set definitions, an IM
amount does not need to be provided in the collateral balances, and a SIMM amount will be
calculated and used. If, however,
an IM amount is still provided, then this provided balance will override the SIMM-generated amount.

This same procedure applies to the variation margin amount. In this case, the calculated VM will be
equivalent to a perfect collateralisation of the netting set (i.e.\ VM = netting set NPV).

\vspace{1em}

For clearing counterparties (see \lstinline!ClearingCounterparty! in the counterparty information (see User Guide),
the initial margin amount will be set to zero in the collateral balance, if provided.
}\fi % \ifdefined\MethodologyOnly
