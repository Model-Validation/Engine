%==========================
\chapter{Value Adjustments}
%==========================

\section{CVA and DVA}\label{sec:app_cvadva}

Using the expected exposures in \ref{sec:app_exposure} unilateral discretised CVA and DVA are given by \cite{Lichters}
\begin{align}
\CVA &= \sum_{i} \PD(t_{i-1},t_i)\times\LGD\times \EPE(t_i) \label{CVA}\\
\DVA &= \sum_{i} \PD_{Bank}(t_{i-1},t_i)\times\LGD_{Bank}\times \ENE(t_i) \label{DVA}
\end{align}
where
\begin{align*}
\EPE(t) & \mbox{ expected exposure (\ref{EE})}\\
\ENE(t) & \mbox{ expected negative exposure (\ref{ENE})}\\
PD(t_i,t_j) & \mbox{ counterparty probability of default in } [t_i;t_j]\\
PD_{Bank}(t_i,t_j) & \mbox{ our probability of default in } [t_i;t_j]\\
LGD & \mbox{ counterparty loss given default}\\
LGD_{Bank} & \mbox{ our loss given default}\\
\end{align*}

Note that the choice $t_i$ in the arguments of $\EPE(t_i)$ and $\ENE(t_i)$ means we are choosing the {\em advanced}
rather than the {\em postponed} discretization of the CVA/DVA integral \cite{BrigoMercurio}. This choice can be easily
changed in the ORE source code or made configurable. \\

Moreover, formulas (\ref{CVA}, \ref{DVA}) assume independence of credit and other market risk factors, so that $\PD$ and
$\LGD$ factors are outside the expectations. With the extension of ORE to credit asset classes and in particular for
wrong-way-risk analysis, CVA/DVA formulas is generaised and is applicable to calculations with dynamic credit

\begin{align}
\CVA^{dyn} &= \sum_{i} \E^N\left[\frac{\PD^{dyn}(t_{i-1},t_i)\times \PE(t_i)}{N(t)} \right]\times\LGD \label{CVA_dynamic} \\
\DVA^{dyn} &= \sum_{i} \E^N\left[\frac{\PD^{dyn}_{Bank}(t_{i-1},t_i)\times \NE(t_i)}{N(t)} \right]\times\LGD_{Bank} \label{DVA_dynamic}
\end{align}
where
\begin{align*}
\PE(t) & \mbox{ random variables representing positive exposure at } t: (NPV(t)-C(t))^+\\
\NE(t) & \mbox{ random variables representing negative exposure at } t: (-NPV(t)+C(t))^+\\
PD^{dyn}(t_i,t_j) & \mbox{ random variables representing counterparty probability of default in } [t_i;t_j]\\
PD^{dyn}_{Bank}(t_i,t_j) & \mbox{ random variables representing our probability of default in } [t_i;t_j]\\
LGD & \mbox{ counterparty loss given default}\\
LGD_{Bank} & \mbox{ our loss given default}\\
\end{align*}

\section{FVA}\label{sec:fva}

%Any exposure (uncollateralised or residual after taking collateral into account) gives rise to funding cost or benefits
%depending on the sign of the residual position. This can be expressed as a Funding Value Adjustment (FVA). A simple
%definition of FVA can be given in a very similar fashion as the sum of unilateral CVA and DVA which we defined by
%(\ref{CVA},\ref{DVA}), namely as an expectation of exposures times funding spreads:
%\begin{align}
%  \FVA &= \underbrace{\sum_{i=1}^n f_b(t_{i-1},t_i)\,\delta_i \, \E^N\left[S_C(t_{i-1})\, S_B(t_{i-1})\, (\NPV(t_i))^+\,
%         D(t_i)\right]}_{\mbox{Funding Benefit Adjustment (FBA)}}\nonumber\\
%       & {} - \underbrace{\sum_{i=1}^n f_l(t_{i-1},t_i)\,\delta_i \, \E^N\left[S_C(t_{i-1})\, S_B(t_{i-1})\, (-\NPV(t_i))^+\, D(t_i)\right]}_{\mbox{Funding Cost Adjustment (FCA)}}\label{eq_simple_fva}
%\end{align}
%where
%\begin{align*}
%D(t_i) & \mbox{ stochastic discount factor, $1/N(t_i)$ in LGM}\\
%\NPV(t_i) & \mbox{ portfolio value after potential collateralization}\\
%S_C(t_j) & \mbox{ survival probability of the counterparty}\\
%S_B(t_j) & \mbox{ survival probability of the bank}\\
%f_b(t_j) & \mbox{ borrowing spread for the bank relative to the collateral compounding rate}\\
%f_l(t_j) & \mbox{ lending spread for the bank relative to the collateral compounding rate}
%\end{align*}
%For details see e.g. Chapter 14 in Gregory \cite{Gregory12} and the discussion in \cite{Lichters}.

Any exposure (uncollateralised or residual after taking collateral into account) gives rise to funding cost or benefits
depending on the sign of the residual position. This can be expressed as a Funding Value Adjustment (FVA). A simple
definition of FVA can be given in a very similar fashion as the sum of unilateral CVA and DVA which we defined by
(\ref{CVA},\ref{DVA}), namely as an expectation of exposures times funding spreads:
\begin{align}
  \FVA &= \underbrace{\sum_{i=1}^n f_l(t_{i-1},t_i)\,\delta_i \, \E^N\left\{S_C(t_{i-1})\, S_B(t_{i-1})\, [-\NPV(t_i)+C(t_i)]^+\,
         D(t_i)\right\}}_{\mbox{Funding Benefit Adjustment (FBA)}}\nonumber\\
       & {} - \underbrace{\sum_{i=1}^n f_b(t_{i-1},t_i)\,\delta_i \, \E^N\left\{S_C(t_{i-1})\, S_B(t_{i-1})\, [\NPV(t_i)-C(t_i)]^+\, D(t_i)\right\}}_{\mbox{Funding Cost Adjustment (FCA)}}\label{eq_simple_fva}
%  \FVA &= - \underbrace{\sum_{i=1}^n f_b(t_{i-1},t_i)\,\delta_i \, \E^N\left[S_C(t_{i-1})\, S_B(t_{i-1})\, (\NPV(t_i))^+\,
 %        D(t_i)\right]}_{\mbox{Funding Cost Adjustment (FCA)}}\nonumber\\
 %      & {} \underbrace{\sum_{i=1}^n f_l(t_{i-1},t_i)\,\delta_i \, \E^N\left[S_C(t_{i-1})\, S_B(t_{i-1})\, (-\NPV(t_i))^+\, D(t_i)\right]}_{\mbox{Funding Benefit Adjustment (FBA)}}\label{eq_simple_fva}
\end{align}
where
\begin{align*}
D(t_i) & \mbox{ stochastic discount factor, $1/N(t_i)$ in LGM}\\
\NPV(t_i) & \mbox{ portfolio value at time } t_i\\
C(t_i) & \mbox{Collateral account balance at time } t_i \\ 
S_C(t_j) & \mbox{ survival probability of the counterparty}\\
S_B(t_j) & \mbox{ survival probability of the bank}\\
f_b(t_j) & \mbox{ borrowing spread for the bank relative to OIS flat}\\
f_l(t_j) & \mbox{ lending spread for the bank relative to OIS flat}
\end{align*}
For details see e.g. Chapter 14 in Gregory \cite{Gregory12} and the discussion in \cite{Lichters}.

\medskip
The reasoning leading to the expression above is as follows. Consider, for example, a single partially collateralised derivative (no collateral at all or CSA with a significant threshold) between us (the Bank) and counterparty 1 (trade 1). 

We assume that we enter into an offsetting trade with (hypothetical) counterparty 2 which is perfectly collateralised (trade 2). We label the NPV of trade 1 and 2 $\NPV_{1,2}$ respectively (from our perspective, excluding CVA). Then $\NPV_2=-\NPV_1$. The respective collateral amounts due to trade 1 and 2 are $C_1$ and $C_2$ from our perspective. Because of the perfect collateralisation of trade 2 we assume $C_2=\NPV_2$. The imperfect collateralisation of trade 1 means $C_1 \ne \NPV_1$. The net collateral balance from our perspective is then $C=C_1+C_2$ which can be written $C=C_1+C_2 = C_1 + \NPV_2 = -\NPV_1 + C_1$.

\begin{itemize}
\item If $C>0$ we receive net collateral and pay the overnight rate on this notional amount. On the other hand we can invest the received collateral and earn our lending rate, so that we have a benefit proportional to the lending spread $f_l$ (lending rate minus overnight rate). It is a benefit assuming $f_l >0$. $C>0$ means $-\NPV_1 + C_1 > 0$ so that we can cover this case with ``lending notional'' $[-\NPV_1 + C_1]^+$.
\item If $C<0$ we post collateral amount $-C$ and receive the overnight rate on this amount. Amount $-C$ needs to be funded in the market, and we pay our borrowing rate on it. This leads to a funding cost proportional to the borrowing spread $f_b$ (borrowing rate minus overnight). $C<0$ means $\NPV_1 - C_1 > 0$, so that we can cover this case with ``borrowing notional'' $[\NPV_1 - C_1]^+$. If the borrowing spread is positive, this term proportional to $f_b \times [\NPV_1 - C_1]^+$ is indeed a cost and therefore needs to be subtracted from the benefit above.
\end{itemize}
   
Formula \eqref{eq_simple_fva} evaluates these funding cost components on the basis of the original trade's or portfolio's $\NPV$. Perfectly collateralised portfolios hence do not contribute to FVA because under the hedging fiction, they are hedged with a perfectly collateralised opposite portfolio, so any collateral payments on portfolio 1 are canceled out by those of the opposite sign on portfolio 2.

\section{COLVA}

When the CSA defines a collateral compounding rate that deviates from the overnight rate, this gives rise to another
value adjustment labeled COLVA \cite{Lichters}. In the simplest case the deviation is just given by a constant spread
$\Delta$:
\begin{align}
\COLVA &= \E^N\left[ \sum_i -C(t_i)\cdot \Delta \cdot \delta_i \cdot D(t_{i+1}) \right]
\label{COLVA}
\end{align}
where $C(t)$ is the collateral balance\footnote{see \ref{sec:app_exposure}, $C(t)>0$ means that we have {\em received}
  collateral from the counterparty} at time $t$ and $D(t)$ is the stochastic discount factor $1/N(t)$ in LGM. Both
$C(t)$ and
$N(t)$ are computed in ORE's Monte Carlo framework, and the expectation yields the desired adjustment. \\
 
Replacing the constant spread by a time-dependent deterministic function in ORE is straight forward. 
  
\section{Collateral Floor Value}

A less trivial extension of the simple COLVA calculation above, also covered in ORE, is the case where the deviation
between overnight rate and collateral rate is stochastic itself. A popular example is a CSA under which the collateral
rate is the overnight rate {\em floored at zero}. To work out the value of this CSA feature one can take the difference
of discounted margin cash flows with and without the floor feature. It is shown in \cite{Lichters} that the following
formula is a good approximation to the collateral floor value
\begin{align}
\Pi_{Floor} &= \E^N\left[ \sum_i -C(t_i)\cdot (-r(t_i))^+\cdot\delta_i \cdot D(t_{i+1}) \right]
\label{CSA_floor_value_approx}
\end{align}
where $r$ is the stochastic overnight rate and $(-r)^+ = r^+ - r$ is the difference between floored and 'un-floored' compounding rate. \\

Taking both collateral spread and floor into account, the value adjustment is 
\begin{align}
\Pi_{Floor,\Delta} &= \E^N\left[ \sum_i -C(t_i)\cdot ((r(t_i)-\Delta)^+-r(t_i))\cdot\delta_i \cdot D(t_{i+1}) \right] 
\label{CSA_floor_value_approx_2}
\end{align}

\section{Dynamic Initial Margin and MVA}\label{sec:app_dim}

The introduction of Initial Margin posting in non-cleared OTC derivatives business reduces residual credit exposures and
the associated value adjustments, {\bf CVA} and {\bf DVA}.

On the other hand, it gives rise to additional funding cost. The value of the latter is referred to as Margin Value Adjustment ({\bf MVA}).\\

To quantify these two effects one needs to model Initial Margin under future market scenarios, i.e. Dynamic Initial Margin ({\bf DIM}). Potential approaches comprise 
\begin{itemize}
\item Monte Carlo VaR embedded into the Monte Carlo simulation
\item Regression-based methods
\item Delta VaR under scenarios
\item ISDA's Standard Initial Margin (SIMM) under scenarios
\end{itemize} 

We skip the first option as too computationally expensive for ORE.

\subsection{Regression Approach}

In ORE releases up to version 12 we have focussed on a relatively
simple regression approach as in \cite{Anfuso2016,LichtersEtAl}. Consider the netting set values $\NPV(t)$ and $\NPV(t+\Delta)$ that
are spaced one margin period of risk $\Delta$ apart. Moreover, let $F(t,t+\Delta)$ denote cumulative netting set cash
flows between time $t$ and $t+\Delta$, converted into the NPV currency. Let $X(t)$ then denote the netting set value
change during the margin period of risk excluding cash flows in that period:
$$
X(t) = \NPV(t+\Delta) + F(t, t+\Delta) - \NPV(t) 
$$  
ignoring discounting/compounding over the margin period of risk. We actually want to determine the distribution of
$X(t)$ conditional on the `state of the world' at time $t$, and pick a high (99\%) quantile to determine the Initial
Margin amount for each time $t$. Instead of working out the distribution, we content ourselves with estimating the
conditional variance $\V(t)$ or standard deviation $S(t)$ of $X(t)$, assuming a normal distribution and scaling $S(t)$
to the desired 99\% quantile by multiplying with the usual factor $\alpha=2.33$ to get an estimate of the Dynamic
Initial Margin $\DIM$:
$$
\V(t) = \E_t[X^2] - \E_t^2[X], \qquad S(t)=\sqrt{\V(t)}, \qquad \DIM(t) = \alpha \,S(t)
$$ 
We further assume that $\E_t[X]$ is small enough to set it to the expected value of $X(t)$ across all Monte Carlo
samples $X$ at time $t$ (rather than estimating a scenario dependent mean). The remaining task is then to estimate the
conditional expectation $\E_t[X^2]$. We do this in the spirit of the Longstaff Schwartz method using regression of
$X^2(t)$ across all Monte Carlo samples at a given time. As a regressor (in the one-dimensional case) we could use
$\NPV(t)$ itself. However, we rather choose to use an adequate market point (interest rate, FX spot rate) as regression
variable $x$, because this is generalised more easily to the multi-dimensional case. As regression basis functions we
use polynomials, i.e. regression functions of the form $c_0 + c_1\,x + c_2\,x^2 + ...+ c_n\,x^n$ where the order $n$ of
the polynomial can be selected by the user. Choosing the lowest order $n=0$, we obtain the simplest possible estimate,
the variance of $X$ across all samples at time $t$, so that we apply a single $\DIM(t)$ irrespective of the 'state of
the world' at time $t$ in that case.  The extension to multi-dimensional regression is also implemented in ORE. The user
can choose several regressors simultaneously (e.g. a EUR rate, a USD rate, USD/EUR spot FX rate, etc.) in order order to
cover complex multi-currency portfolios.

\medskip
Given the DIM estimate along all paths, we can next work out the Margin Value Adjustment \cite{Lichters} in discrete form
%{\color{red}
\begin{align}
\MVA &= \sum_{i=1}^n (f_b - s_I)\, \delta_i\: S_C(t_i)\: S_B(t_i) \times \E^N\left[
\DIM(t_i)\,D(t_i)\right]. \label{MVA} 
\end{align}
%}
with borrowing spread $f_b$ as in the FVA section \ref{sec:fva} and spread $s_I$ received on initial margin, both
spreads relative to the cash collateral rate.

\subsection{VaR under Scenarios: Dynamic Parametric VaR}

Because of the limitations of the regression approach, it needs benchmarking/validation.
In \cite{LichtersEtAl} we have applied a dynamic parametric VaR method for that purpose covering
\begin{itemize}
\item Delta VaR
\item Delta Gamma Normal VaR
\item Delta Gamma VaR (Cornish-Fisher)
\end{itemize}
This has been added to ORE with release 13 and is implemented for the small range of products that are discussed
in \cite{LichtersEtAl}, i.e.
\begin{itemize}
\item Swaps
\item Cross Currency Swaps
\item European Swaptions
\item FX Forwards
\item FX Options
\end{itemize}
where relevant sensitivities can be computed analytically under scenarios which feed into the parametric VaR
calculation. The covariance structure of the VaR model is implied from the calibrated cross asset model (rather than
externally provided), because the primary motivation of the method was benchmarking of the regression approach, in particular
to check the performance of regression methods in option portfolios.

The usage of Dynamic Parametric VaR as Initial Margin proxy is demonstrated in Example 13 (Dynamic Initial Margin and MVA),
compared to Regression IM. 

\section{KVA}\label{sec:app_kva}

\subsection{CCR}

The KVA is calculated for the Counterparty Credit Risk Capital charge
(CCR) following the IRB method concisely  described in
\cite{Gregory15}, Appendix 8A.
It is following the Basel rules by computing risk capital as the
product of alpha weighted  exposure at default, worst case probability
of default at 99.9  and a maturity adjustment factor also described in
the Basel annex 4.
The risk capital charges are discounted with a capital discount factor
and summed up to  give the total CCR KVA after being multiplied with
the risk  weight and a capital charge (following the RWA method).

\medskip Basel II internal rating based (IRB) estimate of worst case
probability of  default: large homogeneous pool (LHP) approximation of
Vasicek (1997), KVA regulatory probability of default is the worst
case probability of default floored at 0.03 (the latter is valid for 
corporates and banks, no such floor applies to sovereign counterparties):
$$
\PD_{99.9\%} = \max\left(floor, N \left(\frac{N^{-1}(\PD) + \sqrt{\rho}
  N^{-1}(0.999)}{\sqrt{1 - \rho}}\right) - \PD\right)
$$
$N$ is the cumulative standard normal distribution,

$$
\rho = 0.12 \frac{1 - e^{-50 \PD}}{1 - e^{-50}} + 0.24 \left(1 - \frac{1 -
  e^{-50 \PD}}{1 - e^{-50}}\right)
$$

\medskip Maturity adjustment factor for RWA method capped at 5, floored at 1:
$$
\MA(\PD, M) = \min\left(5, \max\left(1, \frac{1 + (M - 2.5) B(\PD)}{1 - 1.5 B(\PD)}\right)\right)
$$
\medskip where $B(\PD) = (0.11852 - 0.05478 \ln(\PD))^2$ and M is the
effective  maturity of the portfolio (capped at 5):

$$M = \min\left(5, 1 + \frac{\sum\limits_{t_k > 1yr} \EE_B(t_k)\Delta t_k
  B(0,t_k)}{\sum\limits_{t_k \leq 1yr} \EEE_B(t_k)\Delta t_k B(0,t_k)}\right)
$$

\medskip where $B(0,t_k)$ is the risk-free discount factor from the
simulation  date $t_k$ to today, $\Delta t_k$ is the difference
between time points, 
$\EE_B(t_k)$ is the expected (Basel) exposure at time $t_k$ and $\EEE_B(t_k)$ is the
associated effective expected exposure.

\medskip 
Expected risk capital at $t_i$:
$$
\RC(t_i) = EAD(t_i) \times LGD \times \PD_{99.9\%} \times \MA(\PD, M)
$$
where
\begin{itemize}
\item $\EAD(t_i) = \alpha \times \EEPE(t_i)$
\item $\EEPE(t_i)$ is estimated as the time average of the running maximum of $\EPE(t)$ over the time interval $t_i\leq t\leq t_i+1$
\item $\alpha$ is the multiplier resulting from the IRB calculations (Basel II defines a supervisory alpha of 1.4, but gives banks the option to estimate their own $\alpha$,subject to a floor of 1.2).
\item the maturity adjustment MA is derived from the EPE profile for times $t\geq t_i$
\end{itemize}

\medskip 
$\KVA_{CCR}$ is the sum of the expected risk capital amount discounted at {\em capital discount rate} $r_{cd}$ and compounded at rate given by the product of {\em capital hurdle} $h$ and {\em regulatory adjustment} $a$:
$$
\KVA_{CCR} = \sum_i \RC(t_i) \times \frac{1}{ (1 + r_{cd})^{\delta(t_{i-1}, t_i)}} \times \delta(t_{i-1}, t_i) \times h \times a
$$
assuming Actual/Actual day count to compute the year factions $delta$.

In ORE we compute KVA CCR from both perspectives - ``our'' KVA driven by EPE and the counterparty default risk, and similarly ``their'' KVA driven by ENE and our default risk.

\subsection{BA-CVA}\label{sec:app_kva_cva}

This section briefly summarizes the calculation of a capital value adjustment associated with the CVA capital charge (in the basic approach, BA-CVA) as introduced in Basel III \cite{bcbs189, d325, d424}. ORE implements the {\em stand-alone} capital charge $\SCVA$ for a netting set and computes a KVA for it\footnote{In the reduced version of BA-CVA, where hedges are not recognized, the total BA-CVA capital charge across all counterparties $c$ is given by
$$
K = \sqrt{\left(\rho \sum_c \SCVA_c\right)^2 +(1-\rho^2)\sum_c \SCVA_c^2}
$$  
with supervisory correlation $\rho=0.5$ to reflect that the credit spread risk factors across counterparties are not perfectly correlated. Each counterparty $\SCVA_c$ is given by a sum over all netting sets with this counterparty.}. In the basic approach, the stand-alone capital charge for a netting set is given by
$$
\SCVA = \RW_c\cdot M\cdot \EEPE \cdot\DF
$$
with 
\begin{itemize}
\item supervisory risk weight $\RW_c$ for the counterparty;
\item effective netting set maturity $M$ as in section \ref{sec:app_kva} (for a bank using IMM to calculate EAD), but without applying a cap of 5;
\item supervisory discount $\DF$ for the netting set which is equal to one for banks using IMM to calculate $\EEPE$ and $\DF=\left(1-\exp\left(-0.05\,M\right)\right)/(0.05\,M)$ for banks not using IMM to calculate $\EEPE$. 
\end{itemize}

The associated capital value adjustment is then computed for each netting set's stand-alone CVA charge as above
$$
\KVA_{BA-\CVA} = \sum_i \SCVA(t_i) \times \frac{1}{ (1 + r_{cd})^{\delta(t_{i-1}, t_i)}} \times \delta(t_{i-1}, t_i) \times h \times a
$$
with 
$$
\SCVA(t_i) = \RW_c \cdot M(t_i)\cdot \EEPE(t_i)\cdot\DF
$$
where we derive both $M$ and EEPE from the EPE profile for times $t\geq t_i$.

In ORE we compute KVA BA-CVA from both perspectives - ``our'' KVA driven by EPE and the counterparty risk weight, and similarly ``their'' KVA driven by ENE and our risk weight. \\

Note: Banks that use the BA-CVA for calculating CVA capital requirements are allowed to cap the maturity adjustment factor $\MA(\PD,M)$ in section \ref{sec:app_kva} at 1 for netting sets that contribute to CVA capital, if using the IRB approach for CCR capital.

\section{Collateral (Variation Margin) Model}\label{sec:app_collateral}

The collateral model implemented in ORE is based on the evolution of collateral account balances along each Monte Carlo
path taking into account thresholds, minimum transfer amounts and independent amounts defined in the CSA, as well as
margin periods of risk.

ORE computes the collateral requirement (aka \emph{Credit Support Amount}) through time along each Monte Carlo path
\begin{align}\label{eq:CSA}
CSA(t_m) &= 
\begin{cases}
\max(0, \NPV(t_m) + \IA - \Th_{rec}),& \NPV(t_m) + \IA \ge 0 \\
\min(0, \NPV(t_m) + \IA + \Th_{pay}),& \NPV(t_m) + \IA < 0
\end{cases}
\end{align}
where
\begin{itemize}
\item $\NPV(t_m)$ is the value of the netting set as of
  time $t_m$ from our persepctive,
  \item $\Th_{rec}$ is the threshold exposure below which we do not 
  require collateral, likewise $\TH_{pay}$ is the threshold that applies to collateral posted to the counterparty,
%\item $MTA$ is the minimum transfer amount for collateral margin
%  flow requests (possibly asymmetric)
\item $\IA$ is the sum of all collateral independent amounts attached to
  the underlying portfolio of trades (positive amounts imply that we
  have received a net inflow of independent amounts from the
  counterparty), assumed here to be cash.
\end{itemize}

As the collateral account already has a value of $C(t_m)$ at time $t_m$, the collateral shortfall is simply the
difference between $C(t_m)$ and $\CSA(t_m)$. However, we also need to account for the possibility that margin calls
issued in the past have not yet been settled (for instance, because of disputes). If $M(t_m)$ denotes the net value of
all outstanding margin calls at $t_m$, and $\Delta(t)$ is the difference 
$$
\Delta(t) = \CSA(t_m) - C(t_m) - M(t_m)
$$
between the {\em Credit Support Amount} and the current and outstanding collateral, then the actual margin
\emph{Delivery Amount} $D(t_m)$ is calculated as follows:
\begin{align}\label{eq:DA}
D(t_m) &= 
\begin{cases}
\Delta(t),& \left| \Delta(t) \right| \ge MTA \\
0,& \left| \Delta(t) \right| < MTA
\end{cases}
\end{align}
where $MTA$ is the minimum transfer amount. 

Consider the upper case of \eqref{eq:CSA}: If the initial value of the netting set is zero ($\NPV(t_0)=0$) and 
if $\Th_{rec}=0$, but the combined $\IA>0$, then the Credit Support Amount equals the Independent Amount, $\CSA(t_0)=\IA$.
If moreover the initial collateral balance is zero (because the Independent Amount has not been received yet),
then $\Delta(t_0)=\CSA(t_0)=\IA$, and the delivery amount $D(t_0)$ also matches the $\IA$ (assuming this exceeds the MTA),
so that the next call leads to the transfer of the Independent Amount to us. For a positive $\Th_{rec}>0$, the transfer to us is reduced accordingly.
In that case we can view the Independent Amount as an offset to the threshold.

Consider the lower case of \eqref{eq:CSA}: If the netting set value is negative from our perspective and in absolute terms larger than the $\IA$, 
then the Credit Support Amount is just the negative difference $\CSA=-|\NPV| + \IA + \Th_{pay}$ so that we need to post collateral, but only the amount 
beyond the combined threshold $\IA + \Th_{pay}$.

\subsubsection*{Margin Period of Risk} \label{sec:mpor}
After a counterparty defaults, it takes time to close out the portfolio. During this time period the portfolio value will change upon market conditions, therefore the portfolio's close-out value is subject to market risk, which is referred also as the close-out risk and the corresponding close-out period is called as the {\em Margin Period of Risk} (MPoR).  

Therefore, when a loss on the defaulted counterparty is realised at time $t_d$, the last time the collateral could be received is $t_d-\tau$, where $\tau$ denotes the MPoR. That is, the collateral at time $t_d$ is determined by the collateral value at $t_d-\tau$, namely $CSA(t_d-\tau)$, see equation \ref{eq:CSA}.

In ORE, we have two approaches to incorporate MPoR in the exposure simulations:
\begin{itemize}
 \item {\em Close-out Approach}: Simulating on an auxiliary close-out grid additional to the default time grid.
 \item {\em Lagged Approach}: Simulating only on a default time grid and delaying the margin calls on the grid.
\end{itemize}

\medskip In the {\em Close-out Approach}, we use an auxiliary ``close-out'' grid in addition to the main simulation grid (see the user guide's simulation parameterisation section). The main simulation grid is used to compute “default values” which feed into the collateral balance $C(t$) filtered by MTA and Threshold etc. The auxiliary “close-out” grid, offset from the main grid by the MPoR, is used to compute the delayed close-out values $V(t)$ associated with default time $t$\footnote{We note that in ORE when the exposure of an uncollateralised netting-set or a single trade without considering the netting-set is calculated, then the default value is calculated at the main simulation grid, not on the close-out grid.}. The difference between $V(t)$ and $C(t)$ causes a residual exposure $[V (t)-C(t)]^+$ even if minimum transfer amounts and thresholds are zero, see for example \cite{Pykhtin2010}. This approach allows a detailed modelling of what happens in the close-out period by calculating the close-out values in different ways. ORE currently supports two options: 
%
\begin{itemize}
\item the close-out value can be computed as of default date, by just evolving the market from default date to close-out date (“sticky date”), or
\item the close-out value can be computed as of close-out date, by evolving both valuation date and market over the close-out period (“actual date”), i.e., the portfolio ages and cash flows might occur in the close-out period causing spikes in the evolution of exposures.
\end{itemize}

The option ``sticky date'' is more aggressive in that it avoids any exposure evolution spikes due to contractual cashflows that occur in the close-out period after default, the only exposure effect is due to market evolution over the period. The ``actual date'' option is more conservative in that it includes the effect of all contractual cash flows in the close-out period, in particular outgoing cashflows at any time in the period which cause an exposure jump upwards. A more detailed framework for collateralised exposure modelling is introduced in the article \cite{Andersen2016}, indicating a potential route for extending ORE.

\medskip On the other hand, in the {\em Lagged Approach} the simulation is conducted only on a default time grid. The collateral values are calculated, by delaying the delivery amounts between default times, specified by the {\em Margin Period of Risk} (MPoR) which leads to residual exposure. 

In table \ref{table:lagged}, we present a toy example to illustrate how the delayed margin calls lead to residual exposures. In this example, we assume that the default time grid is equally-spaced with time steps that match the MPoR (which is 1M). Further, we assume zero threshold and MTA. At the initial time, the delivery amount is $2.00$, which is the difference between the initial value of the portfolio and the default value at 1M. If this amount were settled immediately, then the collateral value would have been $10$ and hence the residual exposure would habe been zero at 1M. The delay of the delivery amount by MPoR implies a collateral value of $8.00$ until 1M and hence a residual exposure of $2$. 
%
\begin{table}[!ht]
    \centering
    \begin{tabular}{|p{1cm}|p{1.4cm}|p{1.5cm}|p{1.5cm}|p{1.6cm}|p{1.cm}|}
    \hline
        Time Grid & Default Value & Delivery Amount & Delivery Amount Delayed  & Collateral Value   & NPV  \\ \hline
         0 & 8.00 & 2.00 &   True &~ &  ~   \\ \hline
        1M & 10.00 & 5.00& True & 8.00&  10.00   \\ \hline
        2M & 15.00 & -3.00 & True & 10.00 & 15.00 \\ \hline
        3M & 12.00 & -3.00 & True & 15.00 & 12.00  \\ \hline
        4M & 9.00 & 5.00 & True & 12.00 & 9.00   \\ \hline
        5M & 14.00 & 6.00 & True & 9.00 & 14.00  \\ \hline
        6M & 20.00 &  ~  & ~ & 14.00 & 20.00  \\ \hline
    \end{tabular}
    \caption{Toy example for delayed margin calls.}\label{table:lagged}
\end{table}
%

Some remarks and observations:
\begin{itemize}
 \item {\em Lagged Approach} has the disadvantage that we need to use equally-spaced time grids with time steps that match the MPoR. In the above example, let us assume that the MPoR is 2W. Then, delaying the first delivery amount by 2W would still imply a collateral value of $10.00$ at 1M and hence a zero residual exposure.
  \item In {\em Lagged Approach} approach, we support three calculation (settlement) types where the delay of the {\em Delivery Amount } depends on its sign. The above example corresponds to a ``symmetric'' calculation type where both positive and negative delivery amounts are settled with delay, see the user guide's Parameterisation section for other calculation types.
   \item In ORE, the {\em Close-out Approach} is the preferred method -and the {\em Lagged Approach} is the legacy method- to incorporate MPoR in the collateral model. 
\end{itemize}

\section{Exposure Allocation}\label{sec:app_allocation}

XVAs and exposures are typically computed at netting set level. For accounting purposes it is typically required to {\em
  allocate} XVAs from netting set to individual trade level such that the allocated XVAs add up to the netting set
XVA. This distribution is not trivial, since due to netting and imperfect correlation single trade (stand-alone) XVAs
hardly ever add up to the netting set XVA: XVA is sub-additive similar to VaR. ORE provides an allocation method
(labeled {\em marginal allocation } in the following) which slightly generalises the one proposed in
\cite{PykhtinRosen}. Allocation is done pathwise which first leads to allocated expected exposures and then to allocated
CVA/DVA by inserting these exposures into equations (\ref{CVA},\ref{DVA}). The allocation algorithm in ORE is as
follows:
\begin{itemize}
\item Consider the netting set's discounted $\NPV$ after taking collateral into account, on a given path at time $t$:
$$
E(t)=D(0,t)\,(\NPV(t)-C(t))
$$ 
\item On each path, compute contributions $A_i$ of the latter to trade $i$ as
$$
A_{i} (t) = \left\{ \begin{array}{ll} 
E(t) \times \NPV_{i}(t) / \NPV(t), & |\NPV(t)| > \epsilon \\
E(t) / n, & |\NPV(t)| \le \epsilon
\end{array}
\right. 
$$
with number of trades $n$ in the netting set and trade $i$'s value $\NPV_i(t)$.
\item The $\EPE$ fraction allocated to trade $i$ at time $t$ by averaging over paths:
$$
\EPE_i(t) = \E\left[ A_i^+(t) \right]
$$
\end{itemize}
By construction, $\sum_i A_i(t) = E(t)$ and hence $\sum_i \EPE_i(t) = \EPE(t)$.\\

We introduced the {\em cutoff } parameter $\epsilon>0$ above in order to handle the case where the netting set value
$\NPV(t)$ (almost) vanishes due to netting, while the netting set 'exposure' $E(t)$ does not. This is possible in a
model with nonzero MTA and MPoR. Since a single scenario with vanishing $\NPV(t)$ suffices to invalidate the expected
exposure at this time $t$, the cutoff is essential. Despite introducing this cutoff, it is obvious that the marginal
allocation method can lead to spikes in the allocated exposures. And generally, the marginal allocation leads to both
positive and negative $\EPE$ allocations.

\medskip As a an example for a simple alternative to the marginal allocation of $\EPE$ we provide allocation based on
today's single-trade CVAs
$$
w_i = \CVA_i / \sum_i \CVA_i.
$$
This yields allocated exposures proportional to the netting set exposure, avoids spikes and negative $\EPE$, but does
not distinguish the 'direction' of each trade's contribution to $\EPE$ and $\CVA$.
