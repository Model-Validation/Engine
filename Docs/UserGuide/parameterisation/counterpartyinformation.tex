%========================================================
\section{Counterparty Information}\label{sec:counterpartyinformation}
%========================================================

The counterparty information file - {\tt counterparty.xml} - 
contains a list of counterparty-level details.
The file is written in XML format, with a top-level
\lstinline!CounterpartyInformation! node consisting of two
children nodes: \lstinline!Counterparties! and
\lstinline!Correlations!.

\vspace{1em}

\subsection{Counterparties}
The \lstinline!Counterparties! node is used to define inputs
for each counterparty in the calculations. Each counterparty is
then defined with its own \lstinline!Counterparty! node, with the following
XML template:

\begin{listing}[H]
%\hrule\medskip
\begin{minted}[fontsize=\footnotesize]{xml}
    <Counterparties>
        <Counterparty>
            <CounterpartyId> </CounterpartyId>
            <CreditQuality>IG</CreditQuality>
            <BaCvaRiskWeight> </BaCvaRiskWeight>
            <SaCcrRiskWeight> </SaCcrRiskWeight>
            <SaCvaRiskBucket> </SaCvaRiskBucket>
        </Counterparty>
        <Counterparty>
            .......
        </Counterparty>
    </Counterparties>
\end{minted}
\caption{Counterparty definition}
\label{lst:counterparties}
\end{listing}

The meanings of the various elements of the \lstinline!Counterparty!
node are as follows (default input values for certain analytics are
specified in their own respective sections, otherwise the
defaults below are applicable):
\begin{itemize}
\item \lstinline!CounterpartyId!: The unique identifier for the counterparty.\\
  Allowable values: Any string.
\item \lstinline!ClearingCounterparty! [Optional]: Whether the counterparty is a
clearing counterparty.\\
  Allowable values: Boolean node, allowing \emph{Y}, \emph{N}, \emph{1}, \emph{0},
  \emph{true}, \emph{false}, etc. The full set of allowable values is given in Table
  \ref{tab:boolean_allowable}. If left blank or omitted, defaults to \emph{False}.
\item \lstinline!CreditQuality! [Optional]: Credit quality/rating. \\
  Allowable values: \emph{HY} (high yield), \emph{IG} (investment grade), 
  \emph{NR} (not rated). Defaults to \emph{NR} if left blank or omitted.
\item \lstinline!BaCvaRiskWeight! [Optional]: BA-CVA supervisory risk weight based
on sector and credit quality\ifdefined\RiskCatalogue{, as defined in Section
\ref{sssec_bacva_reduced}}\fi. This field is only used when calculating BA-CVA or SA-CVA.\\
  Allowable values: Any number. If left blank or omitted, defaults to zero.
\item \lstinline!SaCcrRiskWeight! [Optional]: SA-CCR supervisory risk weight based
on sector and credit quality\ifdefined\RiskCatalogue{, as defined in Section
\ref{sssec_saccr_capital_charge}}\fi. This field is only used when calculating
SA-CCR or BA-CVA (which itself calculates SA-CCR).\\
  Allowable values: Any number. If left blank or omitted, defaults to \emph{1.0}.
\item \lstinline!SaCvaRiskBucket! [Optional]: SA-CVA delta risk buckets for
counterparty credit spread.\\
  Allowable values: Any integer from \emph{1} to \emph{8} (inclusive).
\end{itemize}

\subsection{Correlations}

The \lstinline!Correlations! node is used to define counterparty correlations which are
used for calculating correlations between counterparty credit risk factor, with the following
XML template:

\begin{listing}[H]
%\hrule\medskip
\begin{minted}[fontsize=\footnotesize]{xml}
    <Correlations>
        <Correlation cpty1="CPTY_A" cpty2="CPTY_B">0.5</Correlation>
        <Correlation cpty1="CPTY_B" cpty2="CPTY_C">0.5</Correlation>
        ....
    </Correlations>
\end{minted}
\caption{Counterparty correlations definition}
\label{lst:counterparty_correlations}
\end{listing}
