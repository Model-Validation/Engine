%--------------------------------------------------------
\subsection{Historical Return Configuration} \label{sec:historicalreturnconfiguration}
%--------------------------------------------------------

The \lstinline|ReturnConfiguration| allows the user to specify, for each risk factor type, how historical returns are computed in historical scenario generation and backtesting. This configuration controls the return type (e.g., log, absolute, relative) and an optional displacement for each risk factor. Additionally, it is possible to override the default return configuration for specific names (e.g., for a particular equity or commodity).

The root element is \lstinline|ReturnConfiguration|, which contains one or more 
\lstinline|<Return>| blocks. Each \lstinline|Return| must have a
 \lstinline|key| attribute, either the risk factor key type for the default configuration (e.g., \texttt{DiscountCurve}, \texttt{FXSpot}, \texttt{EquitySpot}, etc.). or the risk factor key type plus the underlying name for a specific override (e.g \texttt{EquitySpot/EQUITY1}.

\begin{listing}
\begin{minted}[fontsize=\footnotesize]{xml}
<ReturnConfiguration>
    <Return key="DiscountCurve">
        <Type>Log</Type>
        <Displacement>0.0</Displacement>
    </Return>
    <Return key="EquitySpot">
        <Type>Relative</Type>
        <Displacement>0.0</Displacement>
    </Return>
    <Return key="EquitySpot/EQUITY_1">
        <Type>Absolute</Type>
        <Displacement>0.0</Displacement>
    </Return>
    <!-- ... more configurations ... -->    
</ReturnConfiguration>
\end{minted}
\caption{Historical return configuration}
\label{lst:historical_return_configuration}
\end{listing}

\subsubsection*{Elements and Attributes}

\begin{itemize}
    \item \lstinline|Return key="..."|: Specifies the configuration for a risk factor type. The \lstinline|key| attribute must match a valid risk factor key type or the specific key type and name.
    \begin{itemize}
        \item \lstinline|Type|: The return type. Allowed values are \texttt{Log}, \texttt{Absolute}, and \texttt{Relative}.
        \item \lstinline|Displacement|: The displacement value (floating point). Used to avoid division by zero or negative values in relative/log returns.
    \end{itemize}
\end{itemize}

\subsubsection*{Notes}

\begin{itemize}
    \item If no override is specified for a particular name, the default return configuration for the risk factor type is used.
    \item The displacement should be set to a small positive value if the risk factor can approach zero, to avoid numerical issues.
\end{itemize}