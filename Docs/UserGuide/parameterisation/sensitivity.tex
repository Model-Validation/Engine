%--------------------------------------------------------
\subsection{Sensitivity Analysis: {\tt sensitivity.xml}}\label{sec:sensitivity}
%--------------------------------------------------------

ORE currently supports sensitivity analysis with respect to
\begin{itemize}
\item Discount curves  (in the zero rate domain)
\item Index curves (in the zero rate domain)
\item Yield curves including e.g. equity forecast yield curves (in the zero rate domain)
\item FX Spots
\item FX volatilities
\item Swaption volatilities, ATM matrix or cube 
\item Cap/Floor volatility matrices (in the caplet/floorlet domain)
\item Default probability curves (in the ``zero rate'' domain, expressing survival probabilities $S(t)$ in term of zero rates $z(t)$ via $S(t)=\exp(-z(t)\times t)$ with Actual/365 day counter)
\item Equity spot prices
\item Equity volatilities, ATM or including strike dimension 
\item Zero inflation curves
\item Year-on-Year inflation curves
\item CDS volatilities
\item Bond credit spreads
\item Base correlation curves
\item Correlation termstructures
\end{itemize}

The {\tt sensitivity.xml} file specifies how sensitivities are computed for each market component. 
The general structure is shown in listing \ref{lst:sensitivity_config}, for a more comprehensive case see {\tt Examples/Example\_15}. A subset of the following parameters is used in each market component to specify the sensitivity analysis:

\begin{itemize}
\item {\tt ShiftType:} Both absolute or relative shifts can be used to compute a sensitivity, specified by the key words
  {\tt Absolute} resp. {\tt Relative}.
\item {\tt ShiftSize:} The size of the shift to apply.
\item {\tt ShiftScheme:} The finite difference scheme to use ({\tt Forward}, {\tt Backward}, {\tt Central}), if not given, this parameter defaults to {\tt Forward}
\item {\tt ShiftTenors:} For curves, the tenor buckets to apply shifts to, given as a comma separated list of periods.
\item {\tt ShiftExpiries:} For volatility surfaces, the option expiry buckets to apply shifts to, given as a comma
  separated list of periods.
\item {\tt ShiftStrikes:} For cap/floor, FX option and equity option volatility surfaces, the strikes to apply shifts to, given as a comma separated
  list of absolute strikes
\item {\tt ShiftTerms:} For swaption volatility surfaces, the underlying terms to apply shifts to, given as a comma
  separated list of periods.
\item {\tt Index:} For cap / floor volatility surfaces, the index which together with the currency defines the surface.
  list of absolute strikes
\item {\tt CurveType:} In the context of Yield Curves used to identify an equity ``risk free'' rate forecasting curve; set to {\tt EquityForecast} in this case
\end{itemize}

The ShiftType, ShiftSize, ShiftScheme nodes take an optional attribute key that allows to configure different values for
different sensitivity templates. The sensitivity templates are defined in the pricing engine configuration. This is best
explained by an example: In Example 15 the product type BermudanSwaption has a sensitivity template \verb+IR_FD+
attached, see \ref{lst:sensi_template}. This can be used to specify different shifts for trades that were built against
this engine configuration, see \ref{lst:sensi_config_template}: For Bermudan swaptions a larger shift size of 10bp and a
central difference scheme is used to compute discount curve sensitivities in EUR. Since no separate shift type is
specified, the default shift type {\tt Absolute} is used. Note regarding the reports:

\begin{itemize}
\item the sensi scenario report contains scenario NPVs related to the possibly product specific configured shift sizes
\item the sensi report contains renormalized sensitivities, i.e. sensitivities are always expressed w.r.t. the default shift sizes
\item the sensi config report only contains the default configuration
\end{itemize}

\begin{longlisting}
\begin{minted}[fontsize=\scriptsize]{xml}
  <Product type="BermudanSwaption">
    <Model>LGM</Model>
    <ModelParameters>
      ...
    </ModelParameters>
    <Engine>Grid</Engine>
    <EngineParameters>
      ...
      <Parameter name="SensitivityTemplate">IR_FD</Parameter>
    </EngineParameters>
  </Product>
\end{minted}
\caption{Sensitivity template definition}
\label{lst:sensi_template}
\end{longlisting}
\begin{longlisting}
\begin{minted}[fontsize=\scriptsize]{xml}
    <DiscountCurve ccy="EUR">
      <ShiftType>Absolute</ShiftType>
      <ShiftSize>0.0001</ShiftSize>
      <ShiftScheme>Forward</ShiftScheme>
      <ShiftSize key="IR_FD">0.001</ShiftSize>
      <ShiftScheme key="IR_FD">Central</ShiftScheme>
      <ShiftTenors>6M,1Y,2Y,3Y,5Y,7Y,10Y,15Y,20Y</ShiftTenors>
    </DiscountCurve>
\end{minted}
\caption{Sensitivity template definition}
\label{lst:sensi_config_template}
\end{longlisting}

The cross gamma filter section contains a list of pairs of sensitivity keys. For each possible pair of sensitivity keys
matching the given strings, a cross gamma sensitivity is computed. The given pair of keys can be (and usually are)
shorter than the actual sensitivity keys. In this case only the prefix of the actual key is matched. For example, the
pair {\tt DiscountCurve/EUR,DiscountCurve/EUR} matches all actual sensitivity pairs belonging to a cross sensitivity by
one pillar of the EUR discount curve and another (different) pillar of the same curve. We list the possible keys by
giving an example in each category:

\begin{itemize}
\item {\tt DiscountCurve/EUR/5/7Y}: 7y pillar of discounting curve in EUR, the pillar is at position 5 in the list of
  all pillars (counting starts with zero)
\item {\tt YieldCurve/BENCHMARK\_EUR/0/6M}: 6M pillar of yield curve ``BENCHMARK\_EUR'', the index of the 6M pillar is
  zero (i.e. it is the first pillar)
\item {\tt IndexCurve/EUR-EURIBOR-6M/2/2Y}: 2Y pillar of index forwarding curve for the Ibor index ``EUR-EURIBOR-6M'',
  the pillar index is 2 in this case
\item {\tt OptionletVolatility/EUR/18/5Y/0.04}: EUR caplet volatility surface, at 5Y option expiry and $4\%$ strike, the
  running index for this expiry - strike pair is 18; the index counts the points in the surface in lexical order
  w.r.t. the dimensions option expiry, strike
\item {\tt FXSpot/USDEUR/0/spot}: FX spot USD vs EUR (with EUR as base ccy), the index is always zero for FX spots, the
  pillar is labelled as ``spot'' always
\item {\tt SwaptionVolatility/EUR/11/10Y/10Y/ATM}: EUR Swaption volatility surface at 10Y option expiry and 10Y
  underlying term, ATM level, the running index for this expiry, term, strike triple has running index 11; the index
  counts the points in the surface in lexical order w.r.t. the dimensions option expiry, underlying term and strike
\end{itemize}

Additional flags:

\begin{itemize}
\item ComputeGamma: If set to false, second order sensitivity computation is suppressed
\item UseSpreadedTermStructures: If set to true, spreaded termstructures over t0 will be used for sensitivity
  calculation (where supported), to improve the alignment of the scenario sim market and t0 curves
\end{itemize}

\begin{longlisting}
%\hrule\medskip
  \begin{minted}[fontsize=\scriptsize]{xml}
<SensitivityAnalysis>
  <DiscountCurves>
    <DiscountCurve ccy="EUR">
      <ShiftType>Absolute</ShiftType>
      <ShiftSize>0.0001</ShiftSize>
      <ShiftTenors>6M,1Y,2Y,3Y,5Y,7Y,10Y,15Y,20Y</ShiftTenors>
    </DiscountCurve>
    ...
  </DiscountCurves>
  ...
  <IndexCurves>
    <IndexCurve index="EUR-EURIBOR-6M">
      <ShiftType>Absolute</ShiftType>
      <ShiftSize>0.0001</ShiftSize>
      <ShiftTenors>6M,1Y,2Y,3Y,5Y,7Y,10Y,15Y,20Y</ShiftTenors>
    </IndexCurve>
  </IndexCurves>
  ...
  <YieldCurves>
    <YieldCurve name="BENCHMARK_EUR">
      <ShiftType>Absolute</ShiftType>
      <ShiftSize>0.0001</ShiftSize>
      <ShiftTenors>6M,1Y,2Y,3Y,5Y,7Y,10Y,15Y,20Y</ShiftTenors>
    </YieldCurve>
  </YieldCurves>
  ...
  <FxSpots>
    <FxSpot ccypair="USDEUR">
      <ShiftType>Relative</ShiftType>
      <ShiftSize>0.01</ShiftSize>
    </FxSpot>
  </FxSpots>
  ...
  <FxVolatilities>
    <FxVolatility ccypair="USDEUR">
      <ShiftType>Relative</ShiftType>
      <ShiftSize>0.01</ShiftSize>
      <ShiftExpiries>1Y,2Y,3Y,5Y</ShiftExpiries>
      <ShiftStrikes/>
    </FxVolatility>
  </FxVolatilities>
  ...
  <SwaptionVolatilities>
    <SwaptionVolatility ccy="EUR">
      <ShiftType>Relative</ShiftType>
      <ShiftSize>0.01</ShiftSize>
      <ShiftExpiries>1Y,5Y,7Y,10Y</ShiftExpiries>
      <ShiftStrikes/>
      <ShiftTerms>1Y,5Y,10Y</ShiftTerms>
    </SwaptionVolatility>
  </SwaptionVolatilities>
  ...
  <CapFloorVolatilities>
    <CapFloorVolatility ccy="EUR">
      <ShiftType>Absolute</ShiftType>
      <ShiftSize>0.0001</ShiftSize>
      <ShiftExpiries>1Y,2Y,3Y,5Y,7Y,10Y</ShiftExpiries>
      <ShiftStrikes>0.01,0.02,0.03,0.04,0.05</ShiftStrikes>
      <Index>EUR-EURIBOR-6M</Index>
    </CapFloorVolatility>
  </CapFloorVolatilities>
  ...
  <SecuritySpreads>
    <SecuritySpread security="SECURITY_1">
      <ShiftType>Absolute</ShiftType>
      <ShiftSize>0.0001</ShiftSize>
    </SecuritySpread>
  </SecuritySpreads>
  ...
  <Correlations>
    <Correlation index1="EUR-CMS-10Y" index2="EUR-CMS-2Y">
      <ShiftType>Absolute</ShiftType>
      <ShiftSize>0.01</ShiftSize>
      <ShiftExpiries>1Y,2Y</ShiftExpiries>
      <ShiftStrikes>0</ShiftStrikes>
    </Correlation>
  </Correlations>
  ...
  <CrossGammaFilter>
    <Pair>DiscountCurve/EUR,DiscountCurve/EUR</Pair>
    <Pair>IndexCurve/EUR,IndexCurve/EUR</Pair>
    <Pair>DiscountCurve/EUR,IndexCurve/EUR</Pair>
  </CrossGammaFilter>
  ...
  <ComputeGamma>true</ComputeGamma>
  <UseSpreadedTermStructures>false</UseSpreadedTermStructures>
</SensitivityAnalysis>
\end{minted}
\caption{Sensitivity configuration}
\label{lst:sensitivity_config}
\end{longlisting}

\subsubsection*{Par Sensitivity Analysis}

To perform a par sensitivity analysis, additional sensitivity configuration is required that describes the assumed par instruments and related conventions.
This additional data is required for:
\begin{itemize}
\item DiscountCurves
\item IndexCurves
\item CapFloorVolatilities
\item CreditCurves
\item ZeroInflationIndexCurves
\item YYInflationIndexCurves
\item YYCapFloorVolatilities
\end{itemize}

By default, par conversion is applied to all risk factor types listed above. However, it is possible to exclude
selected risk factor types by adding an optional block to the sensitivity configuration as shown in listing \ref{lst:par_conversion_excludes}. Uncomment the risk factor type(s) here that should be excluded from par conversion.

\begin{longlisting}
  \begin{minted}[fontsize=\footnotesize]{xml}
<SensitivityAnalysis>

  <ParConversionExcludes>
    <!--<Type>DiscountCurve</Type>-->
    <!--<Type>YieldCurve</Type>-->
    <!--<Type>IndexCurve</Type>-->
    <!--<Type>OptionletVolatility</Type>-->
    <!--<Type>SurvivalProbability</Type>-->
    <!--<Type>ZeroInflationCurve</Type>-->
    <!--<Type>YearOnYearInflationCurve</Type>-->
    <!--<Type>YoYInflationCapFloorVolatility</Type>-->
  </ParConversionExcludes>

  ...
  
</SensitivityAnalysis>
\end{minted}
\caption{Par conversion excludes}
\label{lst:par_conversion_excludes}
\end{longlisting}

By default, par sensitivity excludes index historical fixings on valuation date when calculating the sensitivities of par instruments to zero rates. However, it is possible to exclude
selected indexes by adding an optional block to the sensitivity configuration as shown in listing \ref{lst:par_sensi_excludes_fixings}. It handles a regex.

\begin{longlisting}
  \begin{minted}[fontsize=\footnotesize]{xml}
<SensitivityAnalysis>

  <ParSensiRemoveFixing>.*</ParSensiRemoveFixing>
  
</SensitivityAnalysis>
\end{minted}
\caption{Par Sensitivity excludes fixings}
\label{lst:par_sensi_excludes_fixings}
\end{longlisting}

Using DiscountCurves as an example, the full sensitivity specification including par conversion data is as follows:

\begin{longlisting}
  \begin{minted}[fontsize=\footnotesize]{xml}
    <DiscountCurve ccy="EUR">
      <ShiftType>Absolute</ShiftType>
      <ShiftSize>0.0001</ShiftSize>
      <ShiftTenors>2W,1M,3M,6M,9M,1Y,2Y,3Y,4Y,5Y,7Y,10Y,15Y,20Y,25Y,30Y</ShiftTenors>
      <ParConversion>
        <!--DEP, FRA, IRS, OIS, FXF, XBS -->
	<Instruments>OIS,OIS,OIS,OIS,OIS,OIS,OIS,OIS,OIS,OIS,OIS,OIS,OIS,OIS,OIS,OIS</Instruments>
	<SingleCurve>true</SingleCurve>
  <Conventions>
	  <Convention id="DEP">EUR-EURIBOR-CONVENTIONS</Convention>
	  <Convention id="IRS">EUR-6M-SWAP-CONVENTIONS</Convention>
	  <Convention id="OIS">EUR-OIS-CONVENTIONS</Convention>
	</Conventions>
      </ParConversion>
    </DiscountCurve>   
\end{minted}
\caption{Par sensitivity configuration}
\label{lst:par_sensitivity_config}
\end{longlisting}

Using CapFloors as an example:

\begin{longlisting}
  \begin{minted}[fontsize=\footnotesize]{xml}
		<CapFloorVolatility key="USD-SOFR">
			<ShiftType>Absolute</ShiftType>
			<ShiftSize>0.0001</ShiftSize>
			<ShiftScheme>Forward</ShiftScheme>
			<ShiftExpiries>1Y,2Y,3Y,5Y,10Y,15Y,20Y,30Y</ShiftExpiries>
			<ShiftStrikes>-0.0075,-0.005,-0.0025,-0.0015,0.0,0.0025,0.005,0.0075,0.01,0.015,0.02,0.025</ShiftStrikes>
			<Index>USD-SOFR</Index>
			<IsRelative>false</IsRelative>
			<ParConversion>
				<Instruments>CapFloor,CapFloor,CapFloor,CapFloor,CapFloor,CapFloor,CapFloor,CapFloor</Instruments>
				<DiscountCurve>USD-SOFR</DiscountCurve>
				<RateComputationPeriod>3M</RateComputationPeriod>
			</ParConversion>
		</CapFloorVolatility>
\end{minted}
\caption{Par sensitivity configuration}
\label{lst:par_sensitivity_config}
\end{longlisting}


ParConversion Fields:

\begin{itemize}
\item \textbf{Instruments} a comma seperated list of par instrument types, see below for the possible values.
\item \textbf{DiscountCurve} \textit{optional}: discount curve used for pricing the par instrument.
\item \textbf{RateComputationPeriod} \textit{optional}: required for OIS CapFloors, sepcify the period of the optionlet. 
\end{itemize}


Note
\begin{itemize}
\item The list of shift tenors needs to match the list of tenors matches the corresponding grid in the simulation (market) configuration
\item The length of list of (par) instruments needs to match the length of the list of shift tenors
\item Permissible codes for the assumed par instruments:
	\begin{itemize}
	\item DEP, FRA, IRS, OIS, TBS, FXF, XBS in the case of DiscountCurves 
	\item DEP, FRA, IRS, OIS, TBS in the case of IndexCurves
	\item DEP, FRA, IRS, OIS, TBS, XBS in the case of YieldCurves 
	\item ZIS, YYS for YYInflationIndexCurves, interpreted as Year-on-Year Inflation Swaps linked to Zero Inflation resp. YoY Inflation curves
	\item ZIS, YYS for YYCapFloorVolatilities, interpreted as Year-on-Year Inflation Cap Floor linked to Zero Inflation resp. YoY Inflation curves
	\item Any code for CreditCurves, interpreted as CDS
	\item Any code for ZeroInflationIndexCurves, interpreted as CPI Swaps linked to Zero Inflation curves
	\item Any code for CapFloorVolatilities, interpreted as flat Cap/Floor
	\end{itemize}
\item One convention needs to be referenced for each of the instrument codes	
\end{itemize}

