\subsubsection{Formula Based Leg Data}
\label{ss:formulalegdata}

The formula based leg data allows to use complex formulas to describe coupon payoffs. Its {\tt LegType} is {\tt
  FormulaBased}, and it has the data section {\tt FormulaBasedLegData}. It supports IBOR and CMS based payoffs with
quanto and digital features. The following example shows the definition of a coupon paying a capped / floored cross
currency EUR-GBP CMS Spread contingent on a USD CMS barrier.

The {\tt Index} field supports operations of the following kind:
\begin{itemize}
\item indices like IBOR and CMS indices, and constants as factors,
  spreads and/or cap/floor values;
\item basic operations: $+$, $-$, $\cdot$, $/$;
\item operators gtZero() (greater than zero) and geqZero() (greater than or equal zero) yielding $1$ if the argument is
  $>0$ (resp. $\geq 0$) and zero otherwise
\item functions: abs(), exp(), log(), min(), max(), pow()
\end{itemize}
%
In listing \ref{lst:FBLegdata}, we present a {\tt FormulaBasedLegData} example. 
%
\begin{listing}
\begin{minted}[fontsize=\footnotesize]{xml}
<LegData>
  <LegType>FormulaBased</LegType>
  <Payer>true</Payer>
  <Currency>EUR</Currency>
  <PaymentConvention>MF</PaymentConvention>
  <PaymentLag>2</PaymentLag>
  <PaymentCalendar>TARGET</PaymentCalendar>
  <DayCounter>A360</DayCounter>
   ...
  <FormulaBasedLegData>
    <Index>gtZero({USD-CMS-5Y}-0.03)*
              max(min(9.0*({EUR-CMS-10Y}-{GBP-CMS-2Y})+0.02,0.08),0.0)</Index>
    <IsInArrears>false</IsInArrears>
    <FixingDays>2</FixingDays>
  </FormulaBasedLegData>
   ...
</LegData>
\end{minted}
\caption{FormulaBasedLegData configuration.}
\label{lst:FBLegdata}
\end{listing}

This leg data type can be used in Swap and Bond trades.
