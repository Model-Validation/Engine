\subsubsection{ZeroCouponFixed Leg Data}
\label{ss:zerolegdata}

A Zero Coupon Fixed leg contains a series of Zero Coupon payments, i.e $(1 + r)^t \,N$ for a compounded coupon. The uninflated notional $N$ can be subtracted from the payment , i.e $((1 + r)^t-1) \,N$, see \lstinline!SubtractNotional! below.

Listing \ref{lst:zerolegdata} shows an example for a leg of type Zero Coupon Fixed. 

To create a leg with only one payment, the schedule must only contain the start and end date. Note that this can be achieved by setting the \emph{Tenor} to \emph{0D} or \emph{1T}  in a rules based Schedule. 

Note that the DayCounter in a Zero Coupon Fixed leg is used to compute $t$ in $(1+r)^t$ so that the series of zero coupon payments are calculated as $(1 + r)^t \,N$. Also note that the "1/1" DayCounter is treated as "Year" DayCounter on Zero Coupon Fixed legs, (otherwise the $t$ would always be 1).

For leg types other than Zero Coupon Fixed, the DayCounter is used to compute the “Accrual” (i.e. the accrual time period of a coupon) in a coupon payment calculated as: $N * Accrual * r$. However, the “Accrual” in the coupon formula is defaulted to 1 for the Zero Coupon Fixed leg type.

\begin{itemize}
\item Rates: The fixed real rate(s) of the leg. While this can be a single value, a vector of values or a dated vector of
  values. 
 
 Allowable values: Each rate element can take any  real number. The rate is
  expressed in decimal form, e.g. \emph{0.05} is a rate of 5\%.
\item Compounding [Optional]:  The method of compounding applied to the rate.

Allowable values: \emph{Simple} i.e. $(1 + r * t)$, or \emph{Compounded} i.e. $(1 + r)^t$. Defaults to \emph{Compounded} if left blank or omitted. 


\item SubtractNotional [Optional]:  Decides whether the notional is
  subtracted from the compounding factor,  i.e. $(1 + r * t)  - 1$
  respectively  $(1 + r)^t - 1$, or not, i.e. $(1 + r * t)$  respectively  $(1 + r)^t$

Note that if NotionalFinalExchange is set to  \emph{true} an additional final uninflated notional flow $N$ is added. So if NotionalFinalExchange is set to \emph{true}, and SubtractNotional is set to \emph{false}, there will be two final notional flows. It is recommended to omit NotionalFinalExchange causing it to default to \emph{false}, and solely use SubtractNotional to determine the final notional flow.

Allowable values: \emph{true},  \emph{Y} or \emph{false}, \emph{N}, defaults to \emph{true} if left blank or omitted. 

\end{itemize}

\begin{listing}[H]
%\hrule\medskip
\begin{minted}[fontsize=\footnotesize]{xml}
      <LegData>
        <LegType>ZeroCouponFixed</LegType>
        <Payer>false</Payer>
        <Currency>EUR</Currency>
        <Notionals>
          <Notional>10000000</Notional>
        </Notionals>
        <DayCounter>Year</DayCounter>
        <PaymentConvention>Following</PaymentConvention>
        <ScheduleData>
          <Rules>
            <StartDate>2016-07-18</StartDate>
            <EndDate>2021-07-18</EndDate>
            <Tenor>0D</Tenor>
            <Calendar>UK</Calendar>
            <Convention>ModifiedFollowing</Convention>
            <TermConvention>ModifiedFollowing</TermConvention>
            <Rule>Forward</Rule>
            <EndOfMonth/>
            <FirstDate/>
            <LastDate/>
          </Rules>
        </ScheduleData>
        <ZeroCouponFixedLegData>
            <Rates>
                <Rate>0.02</Rate>
            </Rates>
            <Compounding>Simple</Compounding>
            <SubtractNotional>false</SubtractNotional>
        </ZeroCouponFixedLegData>
      </LegData>
\end{minted}
\caption{ZeroCouponFixed leg data}
\label{lst:zerolegdata}
\end{listing}

