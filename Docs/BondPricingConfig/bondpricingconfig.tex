\documentclass[12pt, a4paper]{article}

%\usepackage[urlcolor=blue]{hyperref}

\usepackage[disable]{todonotes}
\usepackage{todonotes}

\usepackage{booktabs}
\usepackage{pbox}

\usepackage{listings}
\usepackage{amsmath}%
\usepackage{amsfonts}%
\usepackage{amssymb}%
\usepackage{graphicx}
\usepackage[miktex]{gnuplottex}
\ShellEscapetrue
\usepackage{epstopdf}
\usepackage{longtable}
\usepackage{floatrow}
\usepackage{minted}
\usepackage{textcomp}
\usepackage{color,soul}
\usepackage[font={small,it}]{caption}
\floatsetup[listing]{style=Plaintop}
\usepackage[utf8]{inputenc}

% Turn off indentation but allow \indent command to still work.
\newlength\tindent
\setlength{\tindent}{\parindent}
\setlength{\parindent}{0pt}
\renewcommand{\indent}{\hspace*{\tindent}}

\addtolength{\textwidth}{0.8in}
\addtolength{\oddsidemargin}{-.4in}
\addtolength{\evensidemargin}{-.4in}
\addtolength{\textheight}{1.6in}
\addtolength{\topmargin}{-.8in}

\usepackage{longtable,supertabular}
\usepackage{footnote}
\usepackage{listings}
\lstset{
  frame=top,frame=bottom,
  basicstyle=\ttfamily,
  language=XML,
  tabsize=2,
  belowskip=2\medskipamount
}

%\usepackage{float}
\usepackage{tabu}
\tabulinesep=1.0mm
\restylefloat{table}

\usepackage{siunitx}
\usepackage{hyperref}
\hypersetup{
	colorlinks=true, %set true if you want colored links
	linktoc=all,     %set to all if you want both sections and subsections linked
	linkcolor=blue,  %choose some color if you want links to stand out
}

%\usepackage[colorlinks=true]{hyperref}

\renewcommand\P{\ensuremath{\mathbb{P}}}
\newcommand\E{\ensuremath{\mathbb{E}}}
\newcommand\Q{\ensuremath{\mathbb{Q}}}
\newcommand\I{\mathds{1}}
\newcommand\F{\ensuremath{\mathcal F}}
\newcommand\V{\ensuremath{\mathbb{V}}}
\newcommand\YOY{{\rm YOY}}
\newcommand\Prob{\ensuremath{\mathbb{P}}}
\newcommand{\D}[1]{\mbox{d}#1}
\newcommand{\NPV}{\mathit{NPV}}
\newcommand{\CVA}{\mathit{CVA}}
\newcommand{\DVA}{\mathit{DVA}}
\newcommand{\FVA}{\mathit{FVA}}
\newcommand{\COLVA}{\mathit{COLVA}}
\newcommand{\FCA}{\mathit{FCA}}
\newcommand{\FBA}{\mathit{FBA}}
\newcommand{\KVA}{\mathit{KVA}}
\newcommand{\MVA}{\mathit{MVA}}
\newcommand{\PFE}{\mathit{PFE}}
\newcommand{\EE}{\mathit{EE}}
\newcommand{\EPE}{\mathit{EPE}}
\newcommand{\ENE}{\mathit{ENE}}
\newcommand{\PD}{\mathit{PD}}
\newcommand{\LGD}{\mathit{LGD}}
\newcommand{\DIM}{\mathit{DIM}}
\newcommand{\bs}{\textbackslash}
\newcommand{\REDY}{\color{red}Y}
\newcommand\enforce{\,\raisebox{0.8 ex}{$\substack{!\\\displaystyle=}$} \,}


\begin{document}

\title{Bond Pricing Configuration in ORE}
\author{Acadia - An LSEG Business}
\date{21 August 2024}
\maketitle

\newpage

%==========================================
\section*{Document Change History}
%==========================================

\begin{supertabular}{|l|l|p{7cm}|}
\hline
Date & Author & Comment \\
\hline
21 August 2024 & Peter Caspers & initial version\\
\hline
\end{supertabular}

\vspace{3cm}

\newpage

\tableofcontents
\vfill

\textcircled{c} 2024 Quaternion Risk Management Limited.  All rights reserved.
Quaternion\textsuperscript{\textregistered} is a trademark of Quaternion Risk Management Limited and is also registered
at the UK Intellectual Property Office and the U.S. Patent and Trademark Office.  All other trademarks are the property
of their respective owners. Open Source Risk Engine\textsuperscript{\textcircled{c}} (ORE) is sponsored by Quaternion
Risk Management Limited.

\newpage

%==========================================
\section{Summary}
%==========================================

This document describes the bond pricing configuration in ORE.

%==========================================
\section{Overview}\label{overview}
%==========================================

Bond pricing depends on a number of market data inputs

\begin{itemize}
\item ReferenceCurve: a rate curve determining the baseline discounting curve for bond valuation.
\item CreditCurve [optional]: a credit curve determining the default probabilty during the lifetime of a bond. This can
  e.g. be
\begin{itemize}
\item a CDS curve of the same issuer and seniority as the bond to proxy the credit risk
\item a sector-rating curve representing the systematic part of the bond's credit risk
\item a ``null'' - curve implying zero default risk for pricing, for the purpose of calculating default risk
  sensitivities or to apply stress scenarios.
\item omitted entirely, implying zero default risk. In this case no default risk sensitivities can be calculated though.
\end{itemize}
In all cases, the security spread can be used to represent additional credit / liquidity risk, see below.
\item SecuritySpread [optional]: a single additional discounting spread. This can generally be interpreted as part of
  the credit risk (generating additional default probabilty) or as part of the baseline discounting (liquidity
  spread). Assumed to be zero if not given. Exception: If a price quote is given and the security spread is configured
  in the security curve config, but no market data is given for the security spread, the security spread is implied from
  the price quote.
\item Recovery Rate [optional]: the recovery rate that applies in the event of a default of the bond. Assumed to be zero
  if not given.
\item PriceQuote [optional]: a liquid price for the bond, this can be used to imply a security spread such that the
  market price of the bond matches the theoretical pricing of the bond.
\item IncomeCurve [optional]: a repo curve, only relevant for bond forwards, falls back to reference curve if not given
\item VolatilityCurve [optional]: a bond volatility surface, only relevant for bond options
\end{itemize}

The above information is configured in the bond reference data and curve configurations. If the curve configurations are
not given as an input, they are auto-generated\footnote{this feature might not be available in the public version of
ore}.

Depending on the specific nature of the bond (vanilla fixed rate, floating rate, convertible, callable, ...) additional
market data might be needed to price a bond, e.g.

\begin{itemize}
\item projection curves for floating rate indices
\item equity curves, volatilities for convertible bonds
\item fx curves, volatilities for cross-currency bonds
\item swaption and cds volatilities to calibrate models to price callability 
\item ...
\end{itemize}

%==========================================
\section{Reference Data}\label{refdata}
%==========================================

Listing \ref{lst:bond_ref_data} shows and example of reference data of a bond.  The bond is identified via its security
id (ISIN:XS2257580857). The CreditCurveId has the same name in this case, the reference curve is set to EURIBOR-3M. No
income or vol curve is specified for the bond.

\begin{listing}[H]
\begin{minted}[fontsize=\footnotesize]{xml}
<ReferenceData>
  <ReferenceDatum id="ISIN:XS2257580857">
    <Type>ConvertibleBond</Type>
    <ConvertibleBondReferenceData>
      <BondData>
        <IssuerId>CELLNEX TELECOM SA</IssuerId>
        <CreditCurveId>ISIN:XS2257580857</CreditCurveId>
        <ReferenceCurveId>EUR-EURIBOR-3M</ReferenceCurveId>
        <IncomeCurveId/>
        <VolatilityCurveId/>
        ...
\end{minted}
\caption{Bond reference data (excerpt)}
\label{lst:bond_ref_data}
\end{listing}

%==========================================
\section{Curve Configurations}\label{curveconfigs}
%==========================================

\subsection{Credit Curve Config}

The credit curve represents the survival probability entering the bond pricing. It also provides a recovery rate that is
used as a fallback if the recovery rate from the security curve config is not available.

The curve config can either be explicitly set up or ore can auto-generate it. Listing
\ref{lst:autogen_default_curve_config} shows an auto-generated config for the bond from section \ref{refdata}:

\begin{itemize}
\item Type: The type is set to Null which means that the default probability of this curve is zero. It is set up
  nevertheless to enable the generation of credit curve sensitivities. No market data is required to build a curve of
  type Null.
\item RecoveryRate: The curve has a recovery rate quote. The wild card allows for recovery rate market data points
  referencing the security id and optionally seniority, currency, doc clause. Note: The recovery is not used to build
  the curve in this case, since the type is Null. However, the curve's recovery rate is used in other contexts. For
  example, it is the fallback value if the recovery rate from the security config is not available.
\end{itemize}

\begin{listing}[H]
\begin{minted}[fontsize=\footnotesize]{xml}
<DefaultCurve>
  <CurveId>ISIN:XS2257580857</CurveId>
  <CurveDescription>Autogenerated curve config for Bond Derivatives</CurveDescription>
  <Currency>EUR</Currency>
  <Configurations>
    <Configuration priority="0">
      <RecoveryRate>RECOVERY_RATE/RATE/ISIN:XS2257580857*</RecoveryRate>
      <Type>Null</Type>
      <DayCounter>Actual/365 (Fixed)</DayCounter>
      <DiscountCurve>Yield/EUR/EUR-EONIA</DiscountCurve>
      <Conventions/>
      <Extrapolation>true</Extrapolation>
      <BootstrapConfig> ... </BootstrapConfig>
      <AllowNegativeRates>false</AllowNegativeRates>
    </Configuration>
  </Configurations>
</DefaultCurve>
\end{minted}
\caption{Auto-generated default curve config}
\label{lst:autogen_default_curve_config}
\end{listing}

\subsection{Security Curve Config}

The security curve config points to additional market data used for bond pricing.

It can be explicitly set up or ore can auto-generate it. Listing \ref{lst:autogen_security_curve_config} shows an
auto-generated config for the bond from section \ref{refdata}:

\begin{itemize}
\item SpreadQuote: Points to the security spread market data point. If it specified in the curve config but not present
  in the market data feed and a PriceQuote is specified and available in the market data, the security spread value is
  implied from the price quote. Note: If no SpreadQuote is specified, no security spread will be implied, no matter if a
  price quote is available.
\item RecoveryRateQuote: Points to the recovery rate used for pricing. If this is not available in the market data, the
  credit curve recovery rate will be used as a fallback, or if no credit curve is specified either, zero will be used.
\item PriceQuote: If specified and available in the market data and if SpreadQuote is specified but not available in the
  market data, the security spread will be implied from the price quote.
\end{itemize}

\begin{listing}[H]
\begin{minted}[fontsize=\footnotesize]{xml}
<Security>
  <CurveId>ISIN:XS2257580857</CurveId>
  <CurveDescription>Autogenerated Config</CurveDescription>
  <SpreadQuote>BOND/YIELD_SPREAD/ISIN:XS2257580857</SpreadQuote>
  <RecoveryRateQuote>RECOVERY_RATE/RATE/ISIN:XS2257580857</RecoveryRateQuote>
  <PriceQuote>BOND/PRICE/ISIN:XS2257580857</PriceQuote>
</Security>
\end{minted}
\caption{Auto-generated default curve config}
\label{lst:autogen_security_curve_config}
\end{listing}

%==========================================
\section{Market and Sensitivity Configs}\label{curveconfigs}
%==========================================

\subsection{Todays market configuration}\label{t0_config}

The todays market configuration provides 

\begin{itemize}
\item a mapping from credit curve ids to default curve configurations
\item a mapping from security ids to security curve configurations
\end{itemize}

Listing \ref{lst:autogen_t0_config} shows an auto-generated example. Note that there are two mappings for the default
curve. The pricing engine builders will always attempt to retrieve the credit curve from the second label starting with
\verb+__SECCRCRV_+. This internal label contains both the security id and the credit curve id from the reference data of
this security. The builder will fallback to the usual first label only if the market object under the second label is
not available.

This allows to calculate security specific credit curve sensitivities even if several bonds share the same credit curve.

\begin{listing}[H]
\begin{minted}[fontsize=\footnotesize]{xml}
<DefaultCurves id="default">
  <DefaultCurve name="ISIN:XS2257580857">Default/EUR/ISIN:XS2257580857</DefaultCurve>
  <DefaultCurve name="__SECCRCRV_ISIN:XS2257580857_&amp;_ISIN:XS2257580857_&amp;_">
                                         Default/EUR/ISIN:XS2257580857</DefaultCurve>
</DefaultCurves>
<Securities id="default">
  <Security name="ISIN:XS2257580857">Security/ISIN:XS2257580857</Security>
</Securities>
\end{minted}
\caption{Auto-generated todays market config for credit curves and securities}
\label{lst:autogen_t0_config}
\end{listing}

Recovery rates are added to todays market for both default curves (the credit curve specific recovery rate) and
securities (the security specific recovery rate). In case of an overlap of names, one value will overwrite the other, so
it is important to ensure consistent values in this case.

\subsection{Simulation configuration}\label{sim_config}

Listing \ref{lst:autogen_sim_config} shows an auto-generated simulation market configuration. It lists the credit curves
and securities to enable in the simulation. Note the usage of the security specific credit curve label starting with
\verb+__SECCRCRV_+, see \ref{t0_config} for an explanation.

Similar to todays market, recovery rates are added for both default curves (the credit curve specific recovery rate) and
securities (the security specific recovery rate). In case of an overlap of names, one value will overwrite the other, so
it is again important to ensure consistent values in this case.

\begin{listing}[H]
\begin{minted}[fontsize=\footnotesize]{xml}
<DefaultCurves>
  <Names>
    <Name>ISIN:XS2257580857</Name>
    <Name>__SECCRCRV_ISIN:XS2257580857_&amp;_ISIN:XS2257580857_&amp;_</Name>
  </Names>
<Securities>
  <Simulate>true</Simulate>
  <Names>
    <Name>ISIN:XS2257580857</Name>
  </Names>
...
\end{minted}
\caption{Auto-generated simulation market config for credit curves and securities}
\label{lst:autogen_sim_config}
\end{listing}

\subsection{Sensitivity configuration}

Listing \ref{lst:autogen_sensi_config} shows an auto-generated sensitivity config for credit curve sensitivities. Again,
this uses the security specific credit curve label starting with \verb+__SECCRCRV_+, see \ref{t0_config} for an
explanation.

Note: The par conversion depends on the relevant recovery rate. As explained in \ref{sim_config}, recovery rates are
added to the sim market for credit curves (the curve specific recovery rate) and for securities (the security specific
recovery rate). 

In this example, the credit curve recovery rate will be used though, because the name refers to the security specific
credit curve starting with \verb+__SECCRCRV_+ and this name appears only in the default curve section of
\ref{lst:autogen_sim_config}, and not the securities section.

\begin{listing}[H]
\begin{minted}[fontsize=\footnotesize]{xml}
<CreditCurve name="__SECCRCRV_ISIN:XS2257580857_&amp;_ISIN:XS2257580857_&amp;_">
  <Currency>EUR</Currency>
  <ShiftType>Absolute</ShiftType>
  <ShiftSize>0.0001</ShiftSize>
  <ShiftScheme>Forward</ShiftScheme>
  <ShiftTenors>1Y, 2Y, 3Y, 5Y, 10Y</ShiftTenors>
  <ParConversion>
    <Instruments>CDS, CDS, CDS, CDS, CDS</Instruments>
    <SingleCurve>false</SingleCurve>
    <DiscountCurve>EUR-EONIA</DiscountCurve>
    <Conventions>
      <Convention id="CDS">CDS-STANDARD-CONVENTIONS</Convention>
    </Conventions>
  </ParConversion>
</CreditCurve>
\end{minted}
\caption{Auto-generated sensitivity config for credit curves}
\label{lst:autogen_sensi_config}
\end{listing}

%==========================================
% \begin{appendix}
%==========================================
% \end{appendix}

%\addcontentsline{toc}{section}{Appendices}

\begin{thebibliography}{1}

\bibitem{oreug} ORE User Guide, \url{http://www.opensourcerisk.org/documentation/}

\end{thebibliography}

\end{document}
